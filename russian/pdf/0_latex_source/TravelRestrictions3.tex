%\documentclass[twocolumn,journal]{IEEEtran}
\documentclass[onecolumn,journal]{IEEEtran}
\usepackage{amsfonts}
\usepackage{amsmath}
\usepackage{amsthm}
\usepackage{amssymb}
\usepackage{graphicx}
\usepackage[T1]{fontenc}
%\usepackage[english]{babel}
\usepackage{supertabular}
\usepackage{longtable}
\usepackage[usenames,dvipsnames]{color}
\usepackage{bbm}
%\usepackage{caption}
\usepackage{fancyhdr}
\usepackage{breqn}
\usepackage{fixltx2e}
\usepackage{capt-of}
%\usepackage{mdframed}
\setcounter{MaxMatrixCols}{10}
\usepackage{tikz}
\usetikzlibrary{matrix}
\usepackage{endnotes}
\usepackage{soul}
\usepackage{marginnote}
%\newtheorem{theorem}{Theorem}
\newtheorem{lemma}{Lemma}
%\newtheorem{remark}{Remark}
%\newtheorem{error}{\color{Red} Error}
\newtheorem{corollary}{Corollary}
\newtheorem{proposition}{Proposition}
\newtheorem{definition}{Definition}
\newcommand{\mathsym}[1]{}
\newcommand{\unicode}[1]{}
\newcommand{\dsum} {\displaystyle\sum}
\hyphenation{op-tical net-works semi-conduc-tor}
\usepackage{pdfpages}
\usepackage{enumitem}
\usepackage{multicol}
\usepackage[sort&compress]{natbib}
\usepackage[russian]{babel}

\headsep = 5pt
\textheight = 730pt
%\headsep = 8pt %25pt
%\textheight = 720pt %674pt
%\usepackage{geometry}

\bibliographystyle{unsrt}

\usepackage{float}

\usepackage{xcolor}
 
\usepackage[framemethod=TikZ]{mdframed}
%%%%%%%FRAME%%%%%%%%%%%
\usepackage[framemethod=TikZ]{mdframed}
\usepackage{framed}
    % \BeforeBeginEnvironment{mdframed}{\begin{minipage}{\linewidth}}
     %\AfterEndEnvironment{mdframed}{\end{minipage}\par}
% \usepackage[document]{ragged2e}

%	%\mdfsetup{%
%	%skipabove=20pt,
%	nobreak=true,
%	   middlelinecolor=black,
%	   middlelinewidth=1pt,
%	   backgroundcolor=purple!10,
%	   roundcorner=1pt}

\mdfsetup{%
	outerlinewidth=1,skipabove=20pt,backgroundcolor=yellow!50, outerlinecolor=black,innertopmargin=0pt,splittopskip=\topskip,skipbelow=\baselineskip, skipabove=\baselineskip,ntheorem,roundcorner=5pt}

\mdtheorem[nobreak=true,outerlinewidth=1,%leftmargin=40,rightmargin=40,
backgroundcolor=yellow!50, outerlinecolor=black,innertopmargin=0pt,splittopskip=\topskip,skipbelow=\baselineskip, skipabove=\baselineskip,ntheorem,roundcorner=5pt,font=\itshape]{result}{Result}


\mdtheorem[nobreak=true,outerlinewidth=1,%leftmargin=40,rightmargin=40,
backgroundcolor=yellow!50, outerlinecolor=black,innertopmargin=0pt,splittopskip=\topskip,skipbelow=\baselineskip, skipabove=\baselineskip,ntheorem,roundcorner=5pt,font=\itshape]{theorem}{Theorem}

\mdtheorem[nobreak=true,outerlinewidth=1,%leftmargin=40,rightmargin=40,
backgroundcolor=gray!10, outerlinecolor=black,innertopmargin=0pt,splittopskip=\topskip,skipbelow=\baselineskip, skipabove=\baselineskip,ntheorem,roundcorner=5pt,font=\itshape]{remark}{Remark}

\mdtheorem[nobreak=true,outerlinewidth=1,%leftmargin=40,rightmargin=40,
backgroundcolor=pink!30, outerlinecolor=black,innertopmargin=0pt,splittopskip=\topskip,skipbelow=\baselineskip, skipabove=\baselineskip,ntheorem,roundcorner=5pt,font=\itshape]{quaestio}{Quaestio}

\mdtheorem[nobreak=true,outerlinewidth=1,%leftmargin=40,rightmargin=40,
backgroundcolor=yellow!50, outerlinecolor=black,innertopmargin=5pt,splittopskip=\topskip,skipbelow=\baselineskip, skipabove=\baselineskip,ntheorem,roundcorner=5pt,font=\itshape]{background}{Background}

%TRYING TO INCLUDE Ppls IN TOC
\usepackage{hyperref}


\begin{document}
\title{\color{Brown} Ограничение перемещения для \\ сокращения распространения заболевания в обществе \\
\vspace{-0.35ex}}
\author{Аарон Грин, Чен Шен, Янир Бар-Ям \\ Институт Сложных Систем Новой Англии \\
 \today 
  \vspace{-14ex} \\ 

   
\bigskip
\bigskip

\textbf{}
 }
    
\maketitle


\flushbottom % Makes all text pages the same height

%\maketitle % Print the title and abstract box

%\tableofcontents % Print the contents section

\thispagestyle{empty} % Removes page numbering from the first page

%----------------------------------------------------------------------------------------
%	ARTICLE CONTENTS
%----------------------------------------------------------------------------------------

%\section*{Introduction} % The \section*{} command stops section numbering

%\addcontentsline{toc}{section}{\hspace*{-\tocsep}Introduction} % Adds this section to the table of contents with negative horizontal space equal to the indent for the numbered sections

%\tableofcontents 
%\section{ Introduction}
\renewcommand{\thefootnote}{\fnsymbol{footnote}}


\begin{multicols}{2}

Ограничение перемещений (Санитарный Кордон) жизненно необходимый интрумент в борьбе против COVID-19, который может позволить снизить распространение до нуля ($\#$CrushTheCurve) и восстановить нормальную активность. За счет сокращения перемещений (путеществий), результаты полученные благодаря локальным усилиям могут быть сохранены, потому что новые вспышки, возникающие из-за перемещающихся людей, предотвращаются. Без таких ограничений любые попытки остановить вспышку на местах будут сведены на нет за счет завезенных случаев. Ждать, пока все области достигнут низкого числа случаев, значительно отбросит сроки восстановленния экономической активности. Ограничения перемещений должны быть применены к людям, перемещающимся из зоны с высоким риском в любую другую, включая другие зоны выского риска.

Статус обозначенной области (зоны) должен быть определен одним из трех цветов: Зеленый, Желтый и Красный.
\begin{itemize}
\item Зеленая зона---нет новых локальных (внутри сообщества) случаев в течение 14 дней подряд. Все новые случаи, если они есть, появляются у людей, находящихся на самоизоляции с момента попадания в зону (путешествующие);
\item Желтая зона--нет новых случаев в течение 14 дней подряд, но есть новые случаи, найденные путем отслеживания контактных или если зона является смежной с красной.
\item Красная зона---новые случаи внутри сообществ, выявленные за последние 14 дней.
\end{itemize}
Зона представляет собой район, который естественным или искуственным образом отделен от соседних районов. Зона должна иметь только контролируемый траффик с соседними зонами. Если два географических района имеют общую границу, такую что она не может эффективно контролироваться, они должны считаться за одну зону. Зоны могут иметь вложенную структуру, где наибольшая единица это страна или штат. Зоны должны ограничивать число пересечений её границ, чтобы обеспечить соблюдение норм транспортного сообщения и карантина.

Базовые пограничные протоколы зон включают:
\begin{itemize}
\item Отсутствие излишних (не являющихся необходимыми) перемещений в любую зону из желтых и красных зон.
\item 14 дневный карантин для любого прибывшего человека в любую зону из желтой или красной зоны
\item Предоставление транзитного проезда, если действует полная изоляция (запрет воздушного транспорта, покидания автомобиля, покидания выделенной полосы движения) в момент перемещения через зону. По возможности предоставлять различные услуги в транзитных зонах, соблюдая все меры предосторожности.
\item Ввоз товаров осуществляется путем подвоза груза в специальную точку на границе для последующей передачи товаров под внутреннее управление.
\end{itemize}

Более полная инфраструктура пограничного контроля над зонами позволяет получить больше опций для перемещений, с учетом мер предотвращения распространения.

Временные пункты пограничного контроля необходимы и должны располагаться непосредственно перед входами и выходами в/из зоны. Такие пункты можно либо создать вновь либо переоборудовать имеющиеся зоны обслуживания или станции взвешивания. Если это имеющиеся станции, к ним нужно добавить средства гигиены и очистки воздуха. Это позволит защитить и путешествубщих и проживающее рядом сообщество. Следует оборудовать такие станции возможностью поесть и отдхнуть, чтобы минимизировать взаимодействие с зоной проезжающих мимо дальнобойщиков. Подобное самообслуживание помогут проезжающим из одной зеленой зоны в другую, если их маршрут должен проходить сквозь желтые/красные зоны. Скорая помощь с биоизоляцией отрицательным давлением должны быть на готове в таких зонах для экстренных случаев и для людей снаружи зоны, ищущих медицинскую помощь в медицинских учреждениях внутри зоны.

Зоны должны развернуть систему для путешествующих из желтых/красных зон, для заведомого уведомления местных властей о своем прибытии. Для этого требуется иметь возможность регистрировать имена прибывающих, номер транспортного средства и ориентировочное время прибытия онлайн. По прибытию в зону обслуживания, путешествующих отправляют к изоляционным учреждениям. В идеале такие учреждения должны быть близки к пограничному контролю и служить временным жильем для официальных лиц, работающих в зоне обслуживания. Благодаря этому будет происходить минимизация контакта с сообществом. Симптоматичная проверка (температура и опросник) должны быть стандартом во всех точках пересечения границ зон. По возможности ПЦР с обратной транскрипцией, быстрый тест при помощи мазка из носа должны быть выполнены. Результаты тестов, по возможности, должны быть добавлены в официальные документы путешествующего. Прибывающим без регистрации въезд запрещается.

Ограничения перемещений должны основываться на местных потребностях. Есть четыре аспекта перемещений: (1) Прибытие в зону, (2) перемещение работников сфер первой необходимости, (3) транзитное пересечение зон, (4) доставка товаров.

\section{Прибытие в зону}

В общем и целом, путешествующие прибывающие из желтой или красной зоны в любую другую (зеленую, желтую или красную) должны самоизолироваться под присмотром медицинского персонала на 14 дней. Прибывающих в зеленые зоны необязательно помещать на карантин, если они прибыли из других зон с низким риском. Чтобы избежать введения более строгих ограничений, путешествующим нужно иметь официальный или просто надежный документ, удостоверяющий маршрут перемещений (откуда, куда, транзитные зоны) и тесты с результатами (по возможности) из транзитных зон. Должна быть создана система для официальных маршрутных документов.

Стоит поощрять пациентов из желтых и красных зон обращаться в медицинские учреждения в своих, домашних зонах, а не перемещаться в другую за медицинским обслуживанием. Если им не удается получить медицинское обслуживание в домашней зоне, тогда они могут обратиться за помощью в специально выделенные клиники, при условии наличия в них соответствующих протоколов, позволяющих принять новых пациентов, которые могут быть заражены. Перемещение из зоны в находящийся по близости медицинский центр должно быть задокументировано.

\section{Перемещения сотрудников сфер первой необходимости}

Чтобы разделить распространение по разным зонам, люди регулярно путешествующие между зонами, по работе и вообще, должны делать следующее:

\begin{enumerate}
\item Временно сменить жилые и рабочие помещения, так чтобы они находились в одной зоне;
\item Следовать четким нормам по изоляции от местного населения в одной из зон. Это может означать проживание в одиночку или работу без контакта с сотрудниками;
\item Сменить должность/выполняемую работу, так чтобы удаленная работа была возможна;
\item Взять оплачиваемый или административный отпуск.
\end{enumerate}
Если вышеперечисленное не возможно, необходимы частые проверочные тесты. В случаях, когда работа ведется в пограничных регионах между двумя зонами с низким риском, для сотрудников сфер первой необходимости могут быть сделаны исключения. Если такие сотрудники имеют положительный тест на COVID-19, им следует переместиться на карантин в домашнюю зону.

В силу исключительных обстоятельств сотрудники сфер первой необходимости могут быть перемещены в зону, с одобренным местными властями карантином и рабочим планом, которые обеспечат низкий риск распространения в рабочих условиях.

\section{Транзитный траффик}

Транзитный аэропорт, автострада и основной траффик не должны блокироваться. Тем не менее, меры предосторожности должны быть приняты на входах в зону и пит стопах. Если зона разрешает транизитные перемещения, путешествующим из зон с более высоким риском, ей стоит отслеживать проходящие через неё транспортные средства, в целях сохранения режима и отслеживания. Необходимо выделить контролируемый маршрут с пит стопами, чтобы обеспечить безопасное перемещение через зону для резидентов.

Путешествующиим через зону стоит избегать излишних остановок по возможности. Если необходимо, носите маску, сохраняйте дистанцию и тщательно мойте руки. Добавьте пит стоп в маршрут, чтобы обозначить свое местоположение в целях отслеживания перемещений (для помощи властям).

\section{Доставка товаров}

Грузовики обязаны иметь четко обозначенное пункт назначения и маршрут. Необходимо развернуть карантин зону, в которую водители грузовиков будут отправляться после доставки товаров. Они либо возвращаются в отправную точку в тот же день либо ждут в карантинной зоне, пока не появится новая работа.
Упаковки должны проходить санитизацию или карантин на период, достаточный для деактивации любого вируса на фомитах и предотвращения распространения.

В общем и целом, для транспортировки в желтую или красную зону, водители грузовиков должны помещаться на карантин по возвращению в домашнюю зону. Грузовой транспорт без помещения на карантин по возвращению возможен, при условии соблюдения следующих условий:
\begin{enumerate}
\item Отсутствуют симптомы заболевания;
\item Водител не покидают транспортное средство, за исключением специально контролируемых пит стопов;
\item Погрузка и разгрузка производится заказчиком;
\item Водители покидают зону в течение 24 часов.
\end{enumerate}
Для поездок длительностью больше 24 часов водители должны, по возвращению, находиться на самоизоляции, пока не появится следующая работа. Если они хотят вернуться в домашнюю зону без ограничений, им необходимо пройти 14 дневный карантин или самоизоляцию заранее.

\end{multicols}
\begin{thebibliography}{20}

\bibitem{r1} \url{https://www.china-briefing.com/news/chinas-travel-restrictions-due-to-covid-19-an-explainer/}

\bibitem{r1} \url{https://www.craigak.com/sites/default/files/fileattachments/administration/page/10896/07-02.pdf}

\bibitem{r2} \url{https://www.gw-world.com/news/press-releases/news/update-coronavirus-covid-19/}

\bibitem{r3} \url{https://www.iatatravelcentre.com/international-travel-document-news/1580226297.htm}

\bibitem{r4} \url{https://www.china-briefing.com/news/wp-content/uploads/2020/04/CHINA-Travel-Entry-COVID-19-Policy-By-Province-As-of-27-March-2020.jpg}

\bibitem{r5} \url{https://www.wsj.com/articles/chinas-new-coronavirus-policies-disrupt-u-s-air-cargo-operations-11585859298}


\end{thebibliography}

% \bibliography{MyCollection.bib}



\end{document}