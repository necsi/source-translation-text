%\documentclass[twocolumn,journal]{IEEEtran}
\documentclass[onecolumn,journal]{IEEEtran}
\usepackage{amsfonts}
\usepackage{amsmath}
\usepackage{amsthm}
\usepackage{amssymb}
\usepackage{graphicx}
\usepackage[T1]{fontenc}
%\usepackage[english]{babel}
\usepackage{supertabular}
\usepackage{longtable}
\usepackage[usenames,dvipsnames]{color}
\usepackage{bbm}
%\usepackage{caption}
\usepackage{fancyhdr}
\usepackage{breqn}
\usepackage{fixltx2e}
\usepackage{capt-of}
%\usepackage{mdframed}
\setcounter{MaxMatrixCols}{10}
\usepackage{tikz}
\usetikzlibrary{matrix}
\usepackage{endnotes}
\usepackage{soul}
\usepackage{marginnote}
%\newtheorem{theorem}{Theorem}
\newtheorem{lemma}{Lemma}
%\newtheorem{remark}{Remark}
%\newtheorem{error}{\color{Red} Error}
\newtheorem{corollary}{Corollary}
\newtheorem{proposition}{Proposition}
\newtheorem{definition}{Definition}
\newcommand{\mathsym}[1]{}
\newcommand{\unicode}[1]{}
\newcommand{\dsum} {\displaystyle\sum}
\hyphenation{op-tical net-works semi-conduc-tor}
\usepackage{pdfpages}
\usepackage{enumitem}
\usepackage{multicol}
\usepackage[sort&compress]{natbib}
\usepackage[russian]{babel}

%\headsep = 5pt
%\textheight = 730pt
\headsep = 25pt
\textheight = 674pt
%\usepackage{geometry}

\bibliographystyle{unsrt}

\usepackage{float}

\usepackage{xcolor}
 
\usepackage[framemethod=TikZ]{mdframed}
%%%%%%%FRAME%%%%%%%%%%%
\usepackage[framemethod=TikZ]{mdframed}
\usepackage{framed}
    % \BeforeBeginEnvironment{mdframed}{\begin{minipage}{\linewidth}}
     %\AfterEndEnvironment{mdframed}{\end{minipage}\par}
% \usepackage[document]{ragged2e}

%	%\mdfsetup{%
%	%skipabove=20pt,
%	nobreak=true,
%	   middlelinecolor=black,
%	   middlelinewidth=1pt,
%	   backgroundcolor=purple!10,
%	   roundcorner=1pt}

\mdfsetup{%
	outerlinewidth=1,skipabove=20pt,backgroundcolor=yellow!50, outerlinecolor=black,innertopmargin=0pt,splittopskip=\topskip,skipbelow=\baselineskip, skipabove=\baselineskip,ntheorem,roundcorner=5pt}

\mdtheorem[nobreak=true,outerlinewidth=1,%leftmargin=40,rightmargin=40,
backgroundcolor=yellow!50, outerlinecolor=black,innertopmargin=0pt,splittopskip=\topskip,skipbelow=\baselineskip, skipabove=\baselineskip,ntheorem,roundcorner=5pt,font=\itshape]{result}{Result}


\mdtheorem[nobreak=true,outerlinewidth=1,%leftmargin=40,rightmargin=40,
backgroundcolor=yellow!50, outerlinecolor=black,innertopmargin=0pt,splittopskip=\topskip,skipbelow=\baselineskip, skipabove=\baselineskip,ntheorem,roundcorner=5pt,font=\itshape]{theorem}{Theorem}

\mdtheorem[nobreak=true,outerlinewidth=1,%leftmargin=40,rightmargin=40,
backgroundcolor=gray!10, outerlinecolor=black,innertopmargin=0pt,splittopskip=\topskip,skipbelow=\baselineskip, skipabove=\baselineskip,ntheorem,roundcorner=5pt,font=\itshape]{remark}{Remark}

\mdtheorem[nobreak=true,outerlinewidth=1,%leftmargin=40,rightmargin=40,
backgroundcolor=pink!30, outerlinecolor=black,innertopmargin=0pt,splittopskip=\topskip,skipbelow=\baselineskip, skipabove=\baselineskip,ntheorem,roundcorner=5pt,font=\itshape]{quaestio}{Quaestio}

\mdtheorem[nobreak=true,outerlinewidth=1,%leftmargin=40,rightmargin=40,
backgroundcolor=yellow!50, outerlinecolor=black,innertopmargin=5pt,splittopskip=\topskip,skipbelow=\baselineskip, skipabove=\baselineskip,ntheorem,roundcorner=5pt,font=\itshape]{background}{Background}

%TRYING TO INCLUDE Ppls IN TOC
\usepackage{hyperref}


\begin{document}
\title{\color{Brown} Снятие Карантина \\
\vspace{-0.35ex}}
\author{\large Зиви Бар-Ям, Чен Шен, Янир Бар-Ям \\ Институт Сложных Систем Новой Англии \\
 \today 
  \vspace{-10ex} \\ 

   
\bigskip
\bigskip

\textbf{}
 }
    
\maketitle


\flushbottom % Makes all text pages the same height

%\maketitle % Print the title and abstract box

%\tableofcontents % Print the contents section

\thispagestyle{empty} % Removes page numbering from the first page

%----------------------------------------------------------------------------------------
%	ARTICLE CONTENTS
%----------------------------------------------------------------------------------------

%\section*{Introduction} % The \section*{} command stops section numbering

%\addcontentsline{toc}{section}{\hspace*{-\tocsep}Introduction} % Adds this section to the table of contents with negative horizontal space equal to the indent for the numbered sections

%\tableofcontents 
%\section{ Introduction}
\renewcommand{\thefootnote}{\fnsymbol{footnote}}

\large
\begin{multicols}{2}

Прежде чем перезапустить экономику, надо удостовериться, что мы не запускаем очередной экономический коллапс. Преждевременное ослабление карантинных мер гарантирует потерю всей пользы. Даже небольшое преждевременное послабление может начать новое распространение, на усмирение которого уйдут недели.

Условия и процессы, которым необходимо следовать:
\begin{enumerate}
\item Ослабить ограничения локально, на основе географии изолированной зоны (не по типу индустрий, торговли или роду занятости)
\item Удостовериться, что ограничения перемещения (путешествий) предотвращают проникновение новых зараженных в регионы. Штрафы за нарушение уменьшат желание проникнуть тайком.
\item Остановить распространение внутри общества (путеществующие или ранее выявленные, находясь на карантне, не предотвращают выход из общей изоляции)
\item Достаточное тестирование дает возможность выявить регионы, в которых нет вируса. Даже после стремительного сокращения числа зараженных, обширное тестирование должно продолжаться как минимум 2 недели, чтобы предотвратить кластерную передачу, вызванную людьми с долгим инкубационным периодом, а также ложноположительными случаями.
\item Не должно быть новых случаев заразившихся локально людей, в течение самого длительного инкубационного периода - 14 дней.
\item Предоставить учреждения для изоляции и медицинского ухода, для вновь/повторно выявленных больных.
\item Продолжать отслеживание контактных.
\item Поэтапно (не одномоментно) ослаблять ограничения и продолжать мониторинг зараженных.
\item Удостовериться, что маски продолжают носить в течение нескольких недель после снятия изоляции.
\item В самом конце разрешить публичный транспорт и собрания в больших группах. Это позволит ослабить ограничения в учреждениях с высоким риском заражения и среди особо уязвимой части населения.
\end{enumerate}

Пока ограничения еще в силе, некоторые вещи все же возможны:
\begin{enumerate}
\item Выход на улицу в зоны, с малым числом людей.
\item Встречи с людьми на улице, но на расстоянии 6-9 метров (1.5 - 2 метра не достаточно). Более близкая дистанция возможна в особо ветренную погоду.
\item Можно водить машину и оставаться в машине в зонах с низкой плотностью людей.
\end{enumerate}

Открытие школ
\begin{enumerate}
\item Начните с встреч на улице, без контакта учителя с учеником один на один.
\item Организуйте встречи на улице небольшими группами, без контакта, в зонах с очень хорошей продуваемостью.
\item Организуйте возможность парных игровых активностей, предпочтительно на улице. Если в помощении, ограничьте контакт до двух семей, которые находились на карантине последние 14 дней.
\end{enumerate}

\end{multicols}
%\begin{thebibliography}{20}

%\bibitem{r1} a


%\end{thebibliography}

% \bibliography{MyCollection.bib}



\end{document}