%\documentclass[twocolumn,journal]{IEEEtran}
\documentclass[onecolumn,journal]{IEEEtran}
\usepackage{amsfonts}
\usepackage{amsmath}
\usepackage{amsthm}
\usepackage{amssymb}
\usepackage{graphicx}
\usepackage[T1]{fontenc}
%\usepackage[english]{babel}
\usepackage{supertabular}
\usepackage{longtable}
\usepackage[usenames,dvipsnames]{color}
\usepackage{bbm}
%\usepackage{caption}
\usepackage{fancyhdr}
\usepackage{breqn}
\usepackage{fixltx2e}
\usepackage{capt-of}
%\usepackage{mdframed}
\setcounter{MaxMatrixCols}{10}
\usepackage{tikz}
\usetikzlibrary{matrix}
\usepackage{endnotes}
\usepackage{soul}
\usepackage{marginnote}
%\newtheorem{theorem}{Theorem}
\newtheorem{lemma}{Lemma}
%\newtheorem{remark}{Remark}
%\newtheorem{error}{\color{Red} Error}
\newtheorem{corollary}{Corollary}
\newtheorem{proposition}{Proposition}
\newtheorem{definition}{Definition}
\newcommand{\mathsym}[1]{}
\newcommand{\unicode}[1]{}
\newcommand{\dsum} {\displaystyle\sum}
\hyphenation{op-tical net-works semi-conduc-tor}
\usepackage{pdfpages}
\usepackage{enumitem}
\usepackage{multicol}
\usepackage[russian]{babel}

\headsep = 5pt
\textheight = 730pt
%\headsep = 8pt %25pt
%\textheight = 720pt %674pt
%\usepackage{geometry}

\bibliographystyle{unsrt}

\usepackage{float}

\usepackage{xcolor}
 
\usepackage[framemethod=TikZ]{mdframed}
%%%%%%%FRAME%%%%%%%%%%%
\usepackage[framemethod=TikZ]{mdframed}
\usepackage{framed}
    % \BeforeBeginEnvironment{mdframed}{\begin{minipage}{\linewidth}}
     %\AfterEndEnvironment{mdframed}{\end{minipage}\par}
% \usepackage[document]{ragged2e}

%	%\mdfsetup{%
%	%skipabove=20pt,
%	nobreak=true,
%	   middlelinecolor=black,
%	   middlelinewidth=1pt,
%	   backgroundcolor=purple!10,
%	   roundcorner=1pt}

\mdfsetup{%
	outerlinewidth=1,skipabove=20pt,backgroundcolor=yellow!50, outerlinecolor=black,innertopmargin=0pt,splittopskip=\topskip,skipbelow=\baselineskip, skipabove=\baselineskip,ntheorem,roundcorner=5pt}

\mdtheorem[nobreak=true,outerlinewidth=1,%leftmargin=40,rightmargin=40,
backgroundcolor=yellow!50, outerlinecolor=black,innertopmargin=0pt,splittopskip=\topskip,skipbelow=\baselineskip, skipabove=\baselineskip,ntheorem,roundcorner=5pt,font=\itshape]{result}{Result}


\mdtheorem[nobreak=true,outerlinewidth=1,%leftmargin=40,rightmargin=40,
backgroundcolor=yellow!50, outerlinecolor=black,innertopmargin=0pt,splittopskip=\topskip,skipbelow=\baselineskip, skipabove=\baselineskip,ntheorem,roundcorner=5pt,font=\itshape]{theorem}{Theorem}

\mdtheorem[nobreak=true,outerlinewidth=1,%leftmargin=40,rightmargin=40,
backgroundcolor=gray!10, outerlinecolor=black,innertopmargin=0pt,splittopskip=\topskip,skipbelow=\baselineskip, skipabove=\baselineskip,ntheorem,roundcorner=5pt,font=\itshape]{remark}{Remark}

\mdtheorem[nobreak=true,outerlinewidth=1,%leftmargin=40,rightmargin=40,
backgroundcolor=pink!30, outerlinecolor=black,innertopmargin=0pt,splittopskip=\topskip,skipbelow=\baselineskip, skipabove=\baselineskip,ntheorem,roundcorner=5pt,font=\itshape]{quaestio}{Quaestio}

\mdtheorem[nobreak=true,outerlinewidth=1,%leftmargin=40,rightmargin=40,
backgroundcolor=yellow!50, outerlinecolor=black,innertopmargin=5pt,splittopskip=\topskip,skipbelow=\baselineskip, skipabove=\baselineskip,ntheorem,roundcorner=5pt,font=\itshape]{background}{Background}

%TRYING TO INCLUDE Ppls IN TOC
\usepackage{hyperref}


\begin{document}
\title{\color{Brown} COVID-19: Как Победить \\
\vspace{-0.35ex}}
\author{Чен Шен и Янир Бар-Ям \\ Институт Сложных Систем Новой Англии \\
 \today 
  \vspace{-14ex} \\ 

   
\bigskip
\bigskip

\textbf{}
 }
    
\maketitle


\flushbottom % Makes all text pages the same height

%\maketitle % Print the title and abstract box

%\tableofcontents % Print the contents section

\thispagestyle{empty} % Removes page numbering from the first page

%----------------------------------------------------------------------------------------
%	ARTICLE CONTENTS
%----------------------------------------------------------------------------------------

%\section*{Introduction} % The \section*{} command stops section numbering

%\addcontentsline{toc}{section}{\hspace*{-\tocsep}Introduction} % Adds this section to the table of contents with negative horizontal space equal to the indent for the numbered sections

%\tableofcontents 
%\section{ Introduction}
\renewcommand{\thefootnote}{\fnsymbol{footnote}}


\begin{multicols}{2}
COVID серьезный противник: скрытный, обманчивый, быстрый и жестокий. Введение карантина помогает, но этого не достаточно. Число случаев в США по-прежнему растет. Скорость снижения случаев в Италии по-прежнему медленная. Ситуация лучше в Австрии, но они начали раньше. Ситуация станет хуже, прежде чем улучшиться. Однако мы можем победить, действуя на опережение и в полную силу. Если делать все правильно, COVID будет остановлен за 5 недель. После тяжких потерь, набив всевожможные шишки, мы сможем вернуться к нормальной жизни. Вот как это сделать.

%Many countries have locked down but is that enough? 
%Here are the steps to win:
\begin{enumerate}
\item Задействуйте всех: Все уровни/аспекты правительства, сообщества, компании, и просто люди - всем придется выложиться по полной, чтобы остановить COVID. Даже небольшая ``течь" может потопить корабль. Воодушевляйте друг друга быть собранными, креативными и адаптивными. Вирус атакует нас экспоненциально, но мы можем сделать тоже самое с вирусом.
\item Изоляция: Разделяйте людей для предотвращения распространения. Дело не в том чтобы загонять людей по домам, а в том чтобы сохранять дистанцию. Гулять можно, если люди не кучкуются. Примерно 80\% распространения происходит внутри семей. Если есть риск заражения, домашние могут сократить его временно разделяясь. Карантин можно будет снять через 5 недель, поскольку снижение случаев может быть экспоненциальным как и рост. Будет сложно, но не бойтесь худшего: мы выдержали пандемию на подъеме заболеваемости, но можем сокрушить её на спаде.
\item Карантины: Нужно организовать помещения для тех, у кого заболевание протекает в легкой и средней форме, чтобы они не могли заразить своих домочадцев. Выделите больницы эксклюзивно для COVID и не-COVID, чтобы сократить сопутствующие потери и перекрестное инфицирование.
\item Носите маски в общественных местах: Кашель, чихание, выдыхание - все распространяет вирус. Маски блокируют распространение, когда люди вынуждены находиться в одном пространстве. Протирайте поверхности дезинфицирующими салфетками/алкоголем и используйте перчатки.
\item Ограничения перемещений (путешествий): Это остановит вспышки, сделает отслеживание контактных возможным и сохранит локальные достижения для лучшего эффекта---поддержание движения вперед, предотвращая откат назад, иначе процесс будет бесконечным. Странам стоит ввести 14 дневное ограничение на перемещение между регионами. Транспортировка необходимых товаров или людей должна следовать "бесконтактному" протоколу для перемещений в зону или меж зонами.
\item Услуги первой необходимости должны быть безопасными: Компаниям, предоставляющим товары и услуги первой необходимости стоит организовать доставку, поднос товаров к машине для загрузки и другие методы сокращения риска для сотрудников и клиентов.
\item Тестирование, тестирование, тестирование: Мы боремся с чем-то, чего не видим. Если мы знаем, кто заражен мы можем их изолировать. При этом не все зараженные проявляют симптомы простуды. Люди с симптомами и без, могут распространять инфекцию, только тестирование позволит видеть картину целиком. За счет конкретного тестирования в конкретных зонах мы можем понять где сосредоточиться, а где можно ослабить ограничения. Однако: так делать можно только при наличии ограничений передвижения, иначе путешествующие начнут переносить инфекцию между зонами.
\item Рекомендации для здоровья: Прежде всего нужно поддерживать состояние здоровья, чтобы предотвратить превращение легкой формы заболевания в тяжелую. Укрепляйте имунную систему при помощи простых действий - пейте достаточное количество воды, спите хорошо, занимайтесь спортом и дышите свежим воздухом. При обычной для простуды температуре тела - около $38^\circ$C или около того, дайте телу использовать эту температуру, чтобы побороть заразу. Выше $39.5^\circ$C используйте медикаменты, чтобы уменьшить жар и обратитесь ко врачу~(\url{https://www.ncbi.nlm.nih.gov/pmc/articles/PMC4786079/}).
\item Поддержите медицинские службы: Больницы и медработники работают с повышенной нагрузкой, будет становиться только хуже, пока наши действия не остановят распространение. Дайте им инструмент, необходимый для спасения жизней: медицинское оборудование, средства личной защиты, предметы первой необходимости. Распознайте их заслуги и окажите им поддержку и помощь, необходимую для преодоления постоянных испытаний.
\end{enumerate}

\end{multicols}


%\begin{thebibliography}{20}
%\end{thebibliography}


% \bibliography{MyCollection.bib}


\end{document}