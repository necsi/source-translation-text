%\documentclass[twocolumn,journal]{IEEEtran}
\documentclass[onecolumn,journal]{IEEEtran}
\usepackage{amsfonts}
\usepackage{amsmath}
\usepackage{amsthm}
\usepackage{amssymb}
\usepackage{graphicx}
\usepackage[T1]{fontenc}
%\usepackage[english]{babel}
\usepackage{supertabular}
\usepackage{longtable}
\usepackage[usenames,dvipsnames]{color}
\usepackage{bbm}
%\usepackage{caption}
\usepackage{fancyhdr}
\usepackage{breqn}
\usepackage{fixltx2e}
\usepackage{capt-of}
%\usepackage{mdframed}
\setcounter{MaxMatrixCols}{10}
\usepackage{tikz}
\usetikzlibrary{matrix}
\usepackage{endnotes}
\usepackage{soul}
\usepackage{marginnote}
%\newtheorem{theorem}{Theorem}
\newtheorem{lemma}{Lemma}
%\newtheorem{remark}{Remark}
%\newtheorem{error}{\color{Red} Error}
\newtheorem{corollary}{Corollary}
\newtheorem{proposition}{Proposition}
\newtheorem{definition}{Definition}
\newcommand{\mathsym}[1]{}
\newcommand{\unicode}[1]{}
\newcommand{\dsum} {\displaystyle\sum}
\hyphenation{op-tical net-works semi-conduc-tor}
\usepackage{pdfpages}
\usepackage{enumitem}
\usepackage{multicol}
\usepackage[sort&compress]{natbib}
\usepackage[utf8]{inputenc}


%\headsep = 5pt
%\textheight = 730pt
\headsep = 25pt
\textheight = 674pt
%\usepackage{geometry}

\bibliographystyle{unsrt}

\usepackage{float}

\usepackage{xcolor}
 
\usepackage[framemethod=TikZ]{mdframed}
%%%%%%%FRAME%%%%%%%%%%%
\usepackage[framemethod=TikZ]{mdframed}
\usepackage{framed}
    % \BeforeBeginEnvironment{mdframed}{\begin{minipage}{\linewidth}}
     %\AfterEndEnvironment{mdframed}{\end{minipage}\par}
% \usepackage[document]{ragged2e}

%	%\mdfsetup{%
%	%skipabove=20pt,
%	nobreak=true,
%	   middlelinecolor=black,
%	   middlelinewidth=1pt,
%	   backgroundcolor=purple!10,
%	   roundcorner=1pt}

\mdfsetup{%
	outerlinewidth=1,skipabove=20pt,backgroundcolor=yellow!50, outerlinecolor=black,innertopmargin=0pt,splittopskip=\topskip,skipbelow=\baselineskip, skipabove=\baselineskip,ntheorem,roundcorner=5pt}

\mdtheorem[nobreak=true,outerlinewidth=1,%leftmargin=40,rightmargin=40,
backgroundcolor=yellow!50, outerlinecolor=black,innertopmargin=0pt,splittopskip=\topskip,skipbelow=\baselineskip, skipabove=\baselineskip,ntheorem,roundcorner=5pt,font=\itshape]{result}{Result}


\mdtheorem[nobreak=true,outerlinewidth=1,%leftmargin=40,rightmargin=40,
backgroundcolor=yellow!50, outerlinecolor=black,innertopmargin=0pt,splittopskip=\topskip,skipbelow=\baselineskip, skipabove=\baselineskip,ntheorem,roundcorner=5pt,font=\itshape]{theorem}{Theorem}

\mdtheorem[nobreak=true,outerlinewidth=1,%leftmargin=40,rightmargin=40,
backgroundcolor=gray!10, outerlinecolor=black,innertopmargin=0pt,splittopskip=\topskip,skipbelow=\baselineskip, skipabove=\baselineskip,ntheorem,roundcorner=5pt,font=\itshape]{remark}{Remark}

\mdtheorem[nobreak=true,outerlinewidth=1,%leftmargin=40,rightmargin=40,
backgroundcolor=pink!30, outerlinecolor=black,innertopmargin=0pt,splittopskip=\topskip,skipbelow=\baselineskip, skipabove=\baselineskip,ntheorem,roundcorner=5pt,font=\itshape]{quaestio}{Quaestio}

\mdtheorem[nobreak=true,outerlinewidth=1,%leftmargin=40,rightmargin=40,
backgroundcolor=yellow!50, outerlinecolor=black,innertopmargin=5pt,splittopskip=\topskip,skipbelow=\baselineskip, skipabove=\baselineskip,ntheorem,roundcorner=5pt,font=\itshape]{background}{Background}

%TRYING TO INCLUDE Ppls IN TOC
\usepackage{hyperref}


\begin{document}
\title{\color{Brown} Opening Up \\
\vspace{-0.35ex}}
\author{\large Zvi Bar-Yam, Chen Shen, Yaneer Bar-Yam \\ New England Complex Systems Institute \\
\vspace{+0.35ex}
\small{\textit{(translated by A. P. Rossi, P. Bonavita})}\\
\today 
  \vspace{-10ex} \\ 

   
\bigskip
\bigskip

\textbf{}
 }
    
\maketitle


\flushbottom % Makes all text pages the same height

%\maketitle % Print the title and abstract box

%\tableofcontents % Print the contents section

\thispagestyle{empty} % Removes page numbering from the first page

%----------------------------------------------------------------------------------------
%	ARTICLE CONTENTS
%----------------------------------------------------------------------------------------

%\section*{Introduction} % The \section*{} command stops section numbering

%\addcontentsline{toc}{section}{\hspace*{-\tocsep}Introduction} % Adds this section to the table of contents with negative horizontal space equal to the indent for the numbered sections

%\tableofcontents 
%\section{ Introduction}
\renewcommand{\thefootnote}{\fnsymbol{footnote}}

\large
\begin{multicols}{2}

Prima di far ripartire l'economia bisogna assicurarsi che questo non causi un nuovo collasso economico. 
Un rilassamento prematuro delle restrizioni garantirà la perdita di tutto quello che abbiamo guadagnato. Un rilassamento prematuro, anche se breve, causerà nuove trasmissioni che non possono essere eliminate in poche settimane.

Condizioni e procedure da seguire:

\begin{enumerate}
  \item Rilassare le restrizioni localmente, per aree geografiche isolate (non per tipologia di industria, settore o mansione)
  \item Assicurarsi che le restrizioni di viaggio prevengano l'ingresso di nuovi casi. Multe o rimpatrii possono aiutare a ridurre la motivazione nel tentare il rientro.
  \item Fermare la trasmissione nella comunità (viaggiatori o contatti con casi precedenti che sono in già quarantena quando cadono malati non ostacolano la riapertura)
  \item Assicurarsi che una sufficiente capacità di test consenta di identificare regioni libere dal virus. Anche dopo la ripida caduta dei casi, controlli diffusi vanno portati avanti per almeno due settimane, per prevenire trasmissione di \textit{cluster} causata da persone con un tempo di incubazione lungo o risultati falsi negativi ai test.
  \item Non devono esserci casi trasmessi localmente nel periodo di ultima incubazione di 14 giorni.
  \item Assicurare strutture per l'isolamento e le cure mediche dei casi positibi identificati.
  \item Organizzare il tracciamento dei contatti
  \item Diverse misure vanno prese per un graduale rilassamento delle restrizioni e per il monotoraggio dei nuovi casi.
  \item Assicurarsi che le mascherine siano utilizzate per svariate settimane dopo la riapertura
  \item Le ultime misure da prendere nella riapertura sono di evitare, sia per i trasporti pubblici e che per gli assembramenti, che si verifichino episodi di super-diffusione. Come ultima cosa, si possono rilassare le restrizioni per le istituzioni ad alto rischi e la popolazione vulnerabile.
\end{enumerate}

Tuttavia, mentre le restrizioni sono ancora presenti, alcune cose  sono possibili (compatibilmente con le normative locali):
\begin{enumerate}
  \item Stare all'aperto in aree dove gli incontri con altre persone sono molto rari.
  \item Incontrare una o due persone all'aperto e stando a una distanza di 6-9 metri (2 metri non sono abbastanza). Distanze piu' ravvicinate sono possibile se in presenza di vento.
  \item Guidare e stare nella propria auto in aree a bassa densità di presenza umana.
\end{enumerate}

Nell'apertura delle scuole
\begin{enumerate}
  \item Cominciare con incontri all'aperto senza contatto diretto tra docenti e studenti
  \item ORganizzare piccoli incontri di gruppo all'aperto senza contatto, in aree con ottima ventilazione.
  \item Organizzare appuntamenti per il gioco con due scolari, preferibilmente all'aperto. Se al chiuso, restringere i contatti tra due famiglie che sono state in isolamento preventivo di 14 giorni.
\end{enumerate}

\end{multicols}
%\begin{thebibliography}{20}

%\bibitem{r1} a


%\end{thebibliography}

% \bibliography{MyCollection.bib}



\end{document}
