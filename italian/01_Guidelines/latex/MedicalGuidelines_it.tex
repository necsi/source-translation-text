%\documentclass[twocolumn,journal]{IEEEtran}
\documentclass[onecolumn,journal]{IEEEtran}
\usepackage{amsfonts}
\usepackage{amsmath}
\usepackage{amsthm}
\usepackage{amssymb}
\usepackage{graphicx}
\usepackage[T1]{fontenc}
%\usepackage[english]{babel}
\usepackage{supertabular}
\usepackage{longtable}
\usepackage[usenames,dvipsnames]{color}
\usepackage{bbm}
%\usepackage{caption}
\usepackage{fancyhdr}
\usepackage{breqn}
\usepackage{fixltx2e}
\usepackage{capt-of}
%\usepackage{mdframed}
\setcounter{MaxMatrixCols}{10}
\usepackage{tikz}
\usetikzlibrary{matrix}
\usepackage{endnotes}
\usepackage{soul}
\usepackage{marginnote}
%\newtheorem{theorem}{Theorem}
\newtheorem{lemma}{Lemma}
%\newtheorem{remark}{Remark}
%\newtheorem{error}{\color{Red} Error}
\newtheorem{corollary}{Corollary}
\newtheorem{proposition}{Proposition}
\newtheorem{definition}{Definition}
\newcommand{\mathsym}[1]{}
\newcommand{\unicode}[1]{}
\newcommand{\dsum} {\displaystyle\sum}
\hyphenation{op-tical net-works semi-conduc-tor}
\usepackage{pdfpages}
\usepackage{enumitem}
\usepackage{multicol}
\usepackage[utf8]{inputenc}

\renewcommand{\thefootnote}{\fnsymbol{footnote}}
\usepackage{contour}
\usepackage{ulem}

\renewcommand{\ULdepth}{1.8pt}
\contourlength{0.8pt}

\newcommand{\myuline}[1]{%
  \uline{\phantom{#1}}%
  \llap{\contour{white}{#1}}%
}
\usepackage{enumitem}

\headsep = 5pt
\textheight = 730pt
%\headsep = 8pt %25pt
%\textheight = 720pt %674pt
%\usepackage{geometry}

\bibliographystyle{unsrt}

\usepackage{float}

\usepackage{xcolor}

\usepackage[framemethod=TikZ]{mdframed}
%%%%%%%FRAME%%%%%%%%%%%
\usepackage[framemethod=TikZ]{mdframed}
\usepackage{framed}
    % \BeforeBeginEnvironment{mdframed}{\begin{minipage}{\linewidth}}
     %\AfterEndEnvironment{mdframed}{\end{minipage}\par}
% \usepackage[document]{ragged2e}

%	%\mdfsetup{%
%	%skipabove=20pt,
%	nobreak=true,
%	   middlelinecolor=black,
%	   middlelinewidth=1pt,
%	   backgroundcolor=purple!10,
%	   roundcorner=1pt}

\mdfsetup{%
	outerlinewidth=1,skipabove=20pt,backgroundcolor=yellow!50, outerlinecolor=black,innertopmargin=0pt,splittopskip=\topskip,skipbelow=\baselineskip, skipabove=\baselineskip,ntheorem,roundcorner=5pt}

\mdtheorem[nobreak=true,outerlinewidth=1,%leftmargin=40,rightmargin=40,
backgroundcolor=yellow!50, outerlinecolor=black,innertopmargin=0pt,splittopskip=\topskip,skipbelow=\baselineskip, skipabove=\baselineskip,ntheorem,roundcorner=5pt,font=\itshape]{result}{Result}


\mdtheorem[nobreak=true,outerlinewidth=1,%leftmargin=40,rightmargin=40,
backgroundcolor=yellow!50, outerlinecolor=black,innertopmargin=0pt,splittopskip=\topskip,skipbelow=\baselineskip, skipabove=\baselineskip,ntheorem,roundcorner=5pt,font=\itshape]{theorem}{Theorem}

\mdtheorem[nobreak=true,outerlinewidth=1,%leftmargin=40,rightmargin=40,
backgroundcolor=gray!10, outerlinecolor=black,innertopmargin=0pt,splittopskip=\topskip,skipbelow=\baselineskip, skipabove=\baselineskip,ntheorem,roundcorner=5pt,font=\itshape]{remark}{Remark}

\mdtheorem[nobreak=true,outerlinewidth=1,%leftmargin=40,rightmargin=40,
backgroundcolor=pink!30, outerlinecolor=black,innertopmargin=0pt,splittopskip=\topskip,skipbelow=\baselineskip, skipabove=\baselineskip,ntheorem,roundcorner=5pt,font=\itshape]{quaestio}{Quaestio}

\mdtheorem[nobreak=true,outerlinewidth=1,%leftmargin=40,rightmargin=40,
backgroundcolor=yellow!50, outerlinecolor=black,innertopmargin=5pt,splittopskip=\topskip,skipbelow=\baselineskip, skipabove=\baselineskip,ntheorem,roundcorner=5pt,font=\itshape]{background}{Background}

%TRYING TO INCLUDE Ppls IN TOC
\usepackage{hyperref}


\begin{document}
\title{\color{Brown} Linee guida speciali per operatori sanitari durante la Pandemia Covid-19\\
\vspace{-0.35ex}}
\author{Paige Voltaire, Chen Shen and Yaneer Bar-Yam \\ New England Complex Systems Institute \\
\vspace{+0.35ex}
\small{\textit{(tradotto by S. Vitale, G. Fantasia})}\\
 \today
  \vspace{-14ex} \\


\bigskip
\bigskip

\textbf{}
 }

\maketitle


\flushbottom % Makes all text pages the same height

%\maketitle % Print the title and abstract box

%\tableofcontents % Print the contents section

\thispagestyle{empty} % Removes page numbering from the first page

%----------------------------------------------------------------------------------------
%	ARTICLE CONTENTS
%----------------------------------------------------------------------------------------

%\section*{Introduction} % The \section*{} command stops section numbering

%\addcontentsline{toc}{section}{\hspace*{-\tocsep}Introduction} % Adds this section to the table of contents with negative horizontal space equal to the indent for the numbered sections

%\tableofcontents
%\section{ Introduction}

\begin{multicols}{2}

Dottori, infermieri, medici e assistenti medici: siete al fronte in una guerra contro la pandemia Covid-19. Data la vostra posizione e importanza, facciamo affidamento a voi per lo svolgimento al meglio delle vostre funzioni, e per seguire poche, semplici ma stringenti linee guida per rallentare la diffusione di questo virus – così salvando ancora più vite di quanto non stiate già facendo.

\begin{enumerate}
\item \textbf{Ovviamente, indossare adeguati dispositivi di protezione personale}, ove disponibili. Documentarsi sui metodi appropriati per indossare e rimuovere gli specifici dispositivi in sicurezza. Considerare che i rischi maggiori si corrono durante la rimozione dei dispositivi. Riutilizzare i dispositivi di sicurezza personale ove indicato. Vi sono diversi metodi di disinfezione approvati dal CDC-Center for Disease Controls and Prevention, ivi incluso l’utilizzo di raggi UV e dell’ozono. Rivolgetevi ai vostri superiori e Dirigenti Medici con riferimento ai processi di sterilizzazione/disinfezione per il riutilizzo dei dispositivi di sicurezza personale. Forniture, dispositivi e supporto stanno per essere distribuiti, ove già non lo fossero stati.

\item \textbf{Turni di lavoro/risposo}: come Nazione e come società, noi ci affidiamo a voi per svolgere i vostri compiti medici al vostro meglio. Affinchè ciò sia possibilie, è necessario \textbf{dormire a sufficienza}. Questa è una necessità assoluta. Tentare, di concerto con i Dirigenti Medici, Caposala e Responsabili, di stabilire e implementare un sistema con turni obbligatori di sonno e riposo per ciascun lavoratore o team. La strategia raccomandata in questa emergenza è un \myuline{massimo di 18 ore di lavoro} e un \myuline{minimo di 12 ore di riposo obbligatorio}.

\item Le vie d’accesso agli ospedali, incluso corsie e ascensori non sono uno spazio sicuro e richiedono un livello ulteriore di protezione. L’individuazione di aree salubri designate per lo staff è estremamente utile per l’arrivo, l’uscita e le pause. Quando la pressione è molto alta, le difficoltà nell’indossare e spogliarsi dei dispositivi di protezione personale per bere, mangiare e andare in bagno dovrebbero essere ridotte al minimo.

\item Appropriate procedures should be standardized for medical personnel to put on and take off their protective equipment. Separate zones should be identified. Make flowcharts of different zones, provide full-length mirrors and observe the walking routes strictly.  %I hope the US and other countries wouldn't get to this point, but due to the difficulty in putting on/off the PPE, in the initial period, many doctors are using adults diaper to avoid going to the bathroom, which increases odds of infection a lot.

\item Non possiamo permetterci di avere personale sanitario deprivato di sonno, con il sistema immunitario compromesso, eccessivamente affaticato o demoralizzato. Quando ciò succede, vengono commessi errori, iniezioni sbagliate, errori di dosaggio, disaccordi e contestazioni, i pazienti ne risentono, i lavoratori si ammalano diventando così rapidamente pazienti. Ciò può portare a un completo collasso del locale sistema ospedaliero. Prendere il giusto tempo per rilassarsi e dormire beneficerà voi, i vostri pazienti, I colleghi e tutti quanti dentro e fuori l’Ospedale.

\item Le ore di lavoro consecutive dovrebbero essere ulteriormente ridotte quando nuovo personale si unisce allo sforzo dello staff. Ad esempio, quando i dottori/infermieri sono giunti a Wuhan da altre regioni, i dottori hanno lavorato 8 ore al giorno e gli infermieri 6 ore al giorno, trattandosi di un lavoro estenuante e intensivo. Azioni eroiche, seppur apprezzabili, conducono a un più alto tasso di mortalità. Pertanto, il riposo obbligatorio è essenziale.

\item \textbf{Distanziamento sociale}: a causa di questa situazione senza precedenti, riteniamo di dover (tristemente) consigliare di non far rientro a casa, dove potreste entrare in contatto con i vostri cari e i membri della famiglia. Spenderete la maggior parte del vostro tempo in un ambiente contagioso a causa di Covid-19 e ricco di altri patogeni Al momento, è troppo rischioso mantenere la propria precedente routine. Tutti i contatti non necessari sono altamente sconsigliati. Al momento, è verosimile che voi potreste contagiare altri nella vostra famiglia o casa, e ciascuno dei contagiati potrebbe diffondere il virus ad altri gruppi, regioni e così via. Ciò comporterebbe facilmente il vanificarsi di tutti gli altri metodi utilizzati per “appiattire la curva”. \textbf{Se possibile, dovreste mantenervi lontani da tutte le persone che non sono assolutamente necessarie alla vostra professione o vita}.

\item Si prega di considerare l’uso di hotel, letti o stanze di ospedale non utilizzati, o altri alloggi necessari. Se si vive soli, o si conosce qualcuno che vive solo, potrebbe essere utile chiedere alloggio per sé e/o per alcuni colleghi per qualche tempo. Qualora fosse necessario tornare a casa, si prega di mantenere l’isolamento dagli altri, indossare una maschera, lavarsi con acqua calda e sapone, e mettere i propri vestiti sporchi in una sacca dell’immondizia o ermeticamente chiusa.

\item \textbf{Organizzare i team}, in genere di 3-5 persone al fine di agevolare le operazioni e di gestire l’incremento del numero di pazienti, al fine di fornire cure di livello elevato e di supportarsi a vicenda all’insorgere dei primi segnali di malattia, infortuni o eccessivo affaticamento. Di seguito alcuni esempi di team che sono adatti a questa crisi: squadre di intubazione, squadre di posizionamento prono, squadre di trattamento extra-cardiaco.

\item 10)	Le istituzioni sanitarie dovrebbero limitare gli estranei (visitatori, famiglie, lavoratori terzi) per ridurre l’esposizione tanto degli stessi quanto dello staff. Personale medico a rischio, che non può lavorare a contatto con i pazienti, può aiutare i team di comunicazione a contattare le famiglie e consentire ai colleghi di concentrarsi sul lavoro clinico.

%Below is a suggestion found to be useful in the US military for organizing medical units.  It is meant to allow for easy of operations and to maximize the quantity and quality of patient care.  Further, these teams are optimal to travel in, lodge in the same space, check and support each other for signs of illness, injury or being over worked.

%\begin{enumerate}[resume]
%\item \textbf{Suggested team structures}: it would be optimal to split into health care teams of 3 or 5 people.  This structure allows for a hierarchy of 1-3 Nurses/Paramedics/Medical Assistants, or 1-2 doctors/PAs/Respiratory therapists/Nurse Practitioners---if available.  This way there is a full functional medical unit, that is fast, adaptable, and highly capable for most all patient care for several people.

%\section*{Example: Team of 5}

%\centering
%\centering  \textbf{1} Doctor (MD/DO), \textbf{1} Respiratory Therapist (RT), \\
%\centering \textbf{1} Paramedic, \textbf{1} Registered Nurse (RN), \\
%\centering and \textbf{1} LPN (Licensed Practical Nurse)

$^*$\small{\textit{Versione rivista e corretta da: Dr. Christian DePaola and Dr. Margit Kaufman}}

\end{enumerate}
\end{multicols}


%\begin{thebibliography}{20}
%\end{thebibliography}


% \bibliography{MyCollection.bib}


\end{document}
