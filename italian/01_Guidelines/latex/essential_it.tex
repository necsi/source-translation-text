%\documentclass[twocolumn,journal]{IEEEtran}
\documentclass[onecolumn,journal]{IEEEtran}
\usepackage{amsfonts}
\usepackage{amsmath}
\usepackage{amsthm}
\usepackage{amssymb}
\usepackage{graphicx}
\usepackage[T1]{fontenc}
%\usepackage[english]{babel}
\usepackage{supertabular}
\usepackage{longtable}
\usepackage[usenames,dvipsnames]{color}
\usepackage{bbm}
%\usepackage{caption}
\usepackage{fancyhdr}
\usepackage{breqn}
\usepackage{fixltx2e}
\usepackage{capt-of}
%\usepackage{mdframed}
\setcounter{MaxMatrixCols}{10}
\usepackage{tikz}
\usetikzlibrary{matrix}
\usepackage{endnotes}
\usepackage{soul}
\usepackage{marginnote}
%\newtheorem{theorem}{Theorem}
\newtheorem{lemma}{Lemma}
%\newtheorem{remark}{Remark}
%\newtheorem{error}{\color{Red} Error}
\newtheorem{corollary}{Corollary}
\newtheorem{proposition}{Proposition}
\newtheorem{definition}{Definition}
\newcommand{\mathsym}[1]{}
\newcommand{\unicode}[1]{}
\newcommand{\dsum} {\displaystyle\sum}
\hyphenation{op-tical net-works semi-conduc-tor}
\usepackage{pdfpages}
\usepackage{enumitem}
\usepackage{multicol}
\usepackage[utf8]{inputenc}


\headsep = 5pt
\textheight = 730pt
%\headsep = 8pt %25pt
%\textheight = 720pt %674pt
%\usepackage{geometry}

\bibliographystyle{unsrt}

\usepackage{float}

 \usepackage{xcolor}

\usepackage[framemethod=TikZ]{mdframed}
%%%%%%%FRAME%%%%%%%%%%%
\usepackage[framemethod=TikZ]{mdframed}
\usepackage{framed}
    % \BeforeBeginEnvironment{mdframed}{\begin{minipage}{\linewidth}}
     %\AfterEndEnvironment{mdframed}{\end{minipage}\par}


%	%\mdfsetup{%
%	%skipabove=20pt,
%	nobreak=true,
%	   middlelinecolor=black,
%	   middlelinewidth=1pt,
%	   backgroundcolor=purple!10,
%	   roundcorner=1pt}

\mdfsetup{%
	outerlinewidth=1,skipabove=20pt,backgroundcolor=yellow!50, outerlinecolor=black,innertopmargin=0pt,splittopskip=\topskip,skipbelow=\baselineskip, skipabove=\baselineskip,ntheorem,roundcorner=5pt}

\mdtheorem[nobreak=true,outerlinewidth=1,%leftmargin=40,rightmargin=40,
backgroundcolor=yellow!50, outerlinecolor=black,innertopmargin=0pt,splittopskip=\topskip,skipbelow=\baselineskip, skipabove=\baselineskip,ntheorem,roundcorner=5pt,font=\itshape]{result}{Result}


\mdtheorem[nobreak=true,outerlinewidth=1,%leftmargin=40,rightmargin=40,
backgroundcolor=yellow!50, outerlinecolor=black,innertopmargin=0pt,splittopskip=\topskip,skipbelow=\baselineskip, skipabove=\baselineskip,ntheorem,roundcorner=5pt,font=\itshape]{theorem}{Theorem}

\mdtheorem[nobreak=true,outerlinewidth=1,%leftmargin=40,rightmargin=40,
backgroundcolor=gray!10, outerlinecolor=black,innertopmargin=0pt,splittopskip=\topskip,skipbelow=\baselineskip, skipabove=\baselineskip,ntheorem,roundcorner=5pt,font=\itshape]{remark}{Remark}

\mdtheorem[nobreak=true,outerlinewidth=1,%leftmargin=40,rightmargin=40,
backgroundcolor=pink!30, outerlinecolor=black,innertopmargin=0pt,splittopskip=\topskip,skipbelow=\baselineskip, skipabove=\baselineskip,ntheorem,roundcorner=5pt,font=\itshape]{quaestio}{Quaestio}

\mdtheorem[nobreak=true,outerlinewidth=1,%leftmargin=40,rightmargin=40,
backgroundcolor=yellow!50, outerlinecolor=black,innertopmargin=5pt,splittopskip=\topskip,skipbelow=\baselineskip, skipabove=\baselineskip,ntheorem,roundcorner=5pt,font=\itshape]{background}{Background}

%TRYING TO INCLUDE Ppls IN TOC
\usepackage{hyperref}


\begin{document}
\title{\color{Brown} Linee guida essenziali sul Coronavirus \\
\vspace{-0.35ex}}
\author{Chen Shen and Yaneer Bar-Yam \\ New England Complex Systems Institute \\
\vspace{+0.35ex}
\small{\textit{(tradotto by A. P. Rossi, P. Bonavita})}\\
 \today
  \vspace{-14ex} \\


\bigskip
\bigskip

\textbf{}
 }

\maketitle


\flushbottom % Makes all text pages the same height

%\maketitle % Print the title and abstract box

%\tableofcontents % Print the contents section

\thispagestyle{empty} % Removes page numbering from the first page

%----------------------------------------------------------------------------------------
%	ARTICLE CONTENTS
%----------------------------------------------------------------------------------------

%\section*{Introduction} % The \section*{} command stops section numbering

%\addcontentsline{toc}{section}{\hspace*{-\tocsep}Introduction} % Adds this section to the table of contents with negative horizontal space equal to the indent for the numbered sections

%\tableofcontents
%\section{ Introduction}
\renewcommand{\thefootnote}{\fnsymbol{footnote}}




\begin{multicols}{2}

\section*{I. Everyone}

\begin{itemize}

\item Riconoscere che questo non è normale, che il comportamento normale deve essere modificato per sopravvivere, proteggere e sostenere la vita

\item Nell'80\% dei casi la malattia e' lieve, nel 20\% grave con necessità i ospedalizzazzione, per il 10\% dei case necessita di terapia intensiva e nel 2-4\% dei casi porta alla morte. Anche casi leggeri sono contagiosi e possono portare a casi gravi in altri, con la diffusione dell'infezione. Ognuno ha la responsabilità di proteggere famiglia, amici, la società intera da più danni.

\item Minimizzare il contatto, mantenere le azioni all'essenziale e quelle normali ove possibile, massimizzare l'aiuto agli altri

\item Una risposta ideale ed immediata può non essere possibile, ma ogni azione che riduce l'esposizione all'infezione riduce sia il proprio rishio personale, sia l'epidemia, e miglioramenti possono avvenire col tempo e attraverso la collaborazione

\end{itemize}

\section*{II. Auto-isolamento}

\begin{itemize}
\item Tenersi a 2 metri di distanza dagli altri, non toccare superfici condivise/pubbliche non sicure

\item Usare sapone, disinfettante, guanti o buste di plastica

\item Tenersi informati sui sintomi tipici, le istruzioni locali per test e cure, essere in grado di agire il più veloce possibile qualora dovessero manifestarsi sintomi

\item Approntare controlli di routine con un amico o parente affidabile (ci sono casi in cui i sintomi colpiscono molto rapidamente)
\end{itemize}

\section*{III. Famiglia e amici}

\begin{itemize}
\item Condividere spazi sicuri con famiglia e amici che sono d'accordo nell'isolarsi dagli altri

\item Coordinate con gli altri negli spazi sicuri l'aiuto reciproco.

\item Individui con sintomi o identificati come casi positivi devono isolarsi separatamente da famiglia e amici
\end{itemize}


\section*{IV. Comunità}

\begin{itemize}
\item Organizzare consegna di beni di necessità senza contatto

\item Certificare gli individui sani. Verificare i sintomi nei membri, controllare ed isolare

\item Provvedere a cure senza contatto per coloro in isolamento

\item Espandere i servizi degli spazi sicuri condivisi

\end{itemize}



\section*{V. Aziende}

\begin{itemize}
\item Massimizzare il lavoro da casa per permettere auto-isolamento e promuovere spazi sicuri
\item Mantenere le funzioni essenziali e ridurre l'impatto su tutte le funzioni usando luogi di lavoro in spazi sicuri

\end{itemize}

\section*{VI. Strutture sanitarie}

\begin{itemize}
\item Partizionare le strutture per prevenire il contagio. Approntare ampi spazi temporanei separati per differenti livelli di cura
\end{itemize}

\section*{VI. Amministrazioni, Governi, Commercio}

\begin{itemize}
\item Organizzare controlli rapidi per identificare individui sani e individui che necessitano di isolamento
\item Isolare le comunità con trasmissione attiva, supportarle con servizi essenziali
\end{itemize}


%
% In aree a rischio elevato laddove il governo non prende misure adeguate, proteggere la famiglia, o la collettività e’ difficile. La diffusione di un incendio Affinché un incendio si diffonda, è necessario che gli elementi combustibili vengano in qualche modo in contatto tra loro. Analogamente, il contagio di COVID-19 necessita di una catena di individui suscettibili. La soluzione e’: (1) ridurre i contatti tra la famiglia e gli altri individui, e provvedere ai bisogni essenziali, quando il rischio aumenta. (2) creare uno spazio sicuro che protegge chi si trova al suo interno, con un accordo condiviso di non essere in contatto fisico non protetto con altri e con superfici che sono toccate da altri.
%
%
% Lo spazio sicuro contribuisce a ridurre il contagio anche perché chi e’ al suo interno non partecipa alla trasmissione della malattia. I membri di uno spazio sicuro si possono unire ai membri di altri spazi sicuri per espandere gradualmente lo spazio e crearne di nuovi. Seguono le linee guida per le famiglie.
%
% Ridurre il contatto tra la famiglia e gli altri:
%
% \begin{itemize}
%
% \item Leggere accuratamente le nostre linee guida per gli individui e condividerle con i familiari. Discutere con loro come ridurre i loro contatti con gli altri
%
% \item Rendere le riunioni familiari virtuali. La presente epidemia o sarà sconfitta, o diventerà diffusa. Nel primo caso, in pochi mesi torneremo alla normalità Nel caso di diffusione, azioni differenti saranno necessarie.
%
% \item Assicuratevi che voi ed i membri della vostra famiglia disponiate dei necessari beni di consumo, inclusa una scorta di medicine. Considerate i membri vulnerabili della famiglia, inclusi gli anziani, ma anche chiunque sopra i 50 anni di eta’, e coloro che soffrono di malattie croniche, come a rischio per il contatto con altri. Cercate di ridurre i loro contatti, e di supportarli in modo tale da farli restare a casa il più possibile e non dover recarsi in spazi affollati.
%
% \item Valutare il trasferimento temporaneo di persone da abitazioni condivise (case di riposo, strutture abitative assistite, case di degenza, etc.) a sistemazioni più’ isolate, ad esempio case private, o strutture più piccole
%
% \item Laddove non fosse possibile ridurre i contatti tra gli ospiti, discutere con i responsabili delle strutture la necessità di incrementare i livelli di precauzione contro il virus.
% \item Evitare assembramenti e posti affollati, tra cui eventi pubblici e ristoranti specialmente se in spazi chiusi
%
% \end{itemize}
%
% Creare Spazi Sicuri in situazioni ad elevato rischio:
%
% \begin{itemize}
%
% \item L’obiettivo principale di stabilire uno spazio sicuro tra un gruppo di persone e’ quello di creare un'unità’ isolata che riduca i contatti fisici con altre individui esterni all'unità’ al minimo indispensabile, al tempo stesso essendo in grado di auto-sostenersi e auto-supportarsi.
%
% \item Non c’e’ motivo per cui i singoli individui debbano aspettare la decisione di misure di sicurezza ‘top-down’ promulgate dai governi. In assenza di interventi drastici e sistematici, gli spazi sicuri autogestiti e organizzati ‘dal basso’ possono essere d’aiuto. Con il crescere del loro numero, le zone sicure possono rallentare o addirittura spegnere i focolai d’infezione locali.
%
% \item Gli Spazi Sicuri possono essere costituiti da famiglie o gruppi che decidano di condividere una struttura abitativa. Più strutture abitative possono essere combinate, ad esempio spostandosi tra l’una e l’altra (a piedi o in macchina), se sono stabiliti e seguiti stringenti protocolli di sicurezza. Per avviare con successo uno Spazio Sicuro, OGNI partecipante deve accettare e rispettare la promessa di minimizzare i contatti fisici esterni al gruppo. Devono essere inoltre stabilite istruzioni chiare su come comportarsi e cooperare. I membri di uno Spazio Sicuro devono essere aperti e trasparenti sulle proprie condizioni di salute e la propria cronologia di viaggio, ed essere reciprocamente responsabile della salute degli altri membri.
%
% \item Per far impegnare gli individui a condividere uno spazio condiviso, alcuni aspetti vanno organizzati con le aziende, le scuole, la famiglia e gli amici. Potrebbe rendersi necessario, stare a casa con l'approvazione dell'azienda, o assentarsi dal lavoro.
%
% \item Pianificare un periodo esteso (almeno una o più’ settimane) in uno spazio sicuro deve essere fatto in anticipo, incluso il reperimento di beni di consumo, ma durante l'approvvigionamento bisogna porre particolare attenzione, data la potenziale esposizione al un gran numero di persone. Formulare delle strategie di sopravvivenza può essere utili in questo ambito. Avere un piano relativo alla gestione dei bisogni è vitale perché la ricerca di approvvigionamenti comporta necessariamente del rischio.
%
% \item Quando possibile, organizzare la consegna di beni, incluso il cibo, per limitare i trasferimenti verso i negozi. Bisogna fare attenzione perché’ i beni consegnati sono stati toccati da qualcuno. A meno che non ci sia un accordo con il fornitore riguardo all’uso di guanti, lavare e disinfettare i beni ricevuti e’ consigliato  nelle aree di trasmissione attiva.
%
% \item Per attività essenziali, inclusi gli acquisti, durante le quali un contatto fisico esterno è inevitabile, i membri della famiglia devono pianificare in anticipo per agire in modo efficiente e minimizzare la durata e l’estensione del contatto. Uscire e tornare nello spazio sicuro necessita di precauzioni. Usare protezione personale appropriata, inclusi guanti e materiale monouso (fazzoletti) per prendere o manipolare oggetti che non devono essere toccati, disinfettante o alcool per le mani, e mascherine. Ritornare allo spazio sicuro necessità di lavarsi e disinfettarsi, preferibilmente prima del rientro.
%
% \item Promuovere la comunicazione interna e la mutua attenzione è essenziale per mantenere i membri dello spazio sicuro in una relazione positiva ed in buona salute mentale. Riconoscere che l’emergenza corrente richiede azioni straordinarie e sacrifici e’ fondamentale. Questo può mitigare, anche se non sostituire completamente, l’importanza del supporto reciproco.
%
% \item I membri dello spazio sicuro devono ottenere informazioni sulle azioni da intraprendere in caso uno o più membri mostrino sintomi di infezione. Le azioni variano in base al paese/luogo, e devono essere peraltro dinamiche secondo la situazione. I membri devono informarsi reciprocamente all’interno del gruppo a riguardo del piano contingente più aggiornato e le informazioni sui contatti avuti. Nel caso che un membro mostri sintomi tipici, gli altri devono agire velocemente per aiutarlo/a a farsi testare, e praticare l’isolamento precauzionale nel periodo precedente all’ottenimento dei risultati.
%
% \end{itemize}
%
% Quando una epidemia progredisce, decisioni difficili dovranno inevitabilmente essere prese a riguardo dell’uscita da uno spazio sicuro per aiutare membri della famiglia o amici che non sono in uno spazio sicuro. Gli individui devono essere preparati a fare queste decisioni.
%
% In un periodo di rischio elevato, ci saranno inevitabilmente azioni prese per sbaglio che potrebbero compromettere la sicurezza. Per evitare reazioni eccessive a un evento singolo, è importante capire che ogni singolo atto ha una piccola probabilità di creare danni. Tuttavia, quando azioni multiple vengono prese, il rischio aumenta drammaticamente. Assicurarsi che le relative lezioni vengano apprese è più importante di accusare, incolpare o punire.


\end{multicols}



% \bibliography{MyCollection.bib}


\end{document}
