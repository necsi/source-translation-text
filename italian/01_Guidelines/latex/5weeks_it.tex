%\documentclass[twocolumn,journal]{IEEEtran}
\documentclass[onecolumn,journal]{IEEEtran}
\usepackage{amsfonts}
\usepackage{amsmath}
\usepackage{amsthm}
\usepackage{amssymb}
\usepackage{graphicx}
\usepackage[T1]{fontenc}
%\usepackage[english]{babel}
\usepackage{supertabular}
\usepackage{longtable}
\usepackage[usenames,dvipsnames]{color}
\usepackage{bbm}
\usepackage{caption}
\usepackage{fancyhdr}
\usepackage{breqn}
\usepackage{fixltx2e}
\usepackage{capt-of}
%\usepackage{mdframed}
\setcounter{MaxMatrixCols}{10}
\usepackage{tikz}
\usetikzlibrary{matrix}
\usepackage{endnotes}
\usepackage{soul}
\usepackage{marginnote}
%\newtheorem{theorem}{Theorem}
\newtheorem{lemma}{Lemma}
%\newtheorem{remark}{Remark}
%\newtheorem{error}{\color{Red} Error}
\newtheorem{corollary}{Corollary}
\newtheorem{proposition}{Proposition}
\newtheorem{definition}{Definition}
\newcommand{\mathsym}[1]{}
\newcommand{\unicode}[1]{}
\newcommand{\dsum} {\displaystyle\sum}
\hyphenation{op-tical net-works semi-conduc-tor}
\usepackage{pdfpages}
\usepackage{enumitem}
\usepackage{multicol}
\usepackage[utf8]{inputenc}


\headsep = 5pt
\textheight = 730pt
%\headsep = 8pt %25pt
%\textheight = 720pt %674pt
%\usepackage{geometry}

\bibliographystyle{unsrt}

\usepackage{float}

 \usepackage{xcolor}

\usepackage[framemethod=TikZ]{mdframed}
%%%%%%%FRAME%%%%%%%%%%%
\usepackage[framemethod=TikZ]{mdframed}
\usepackage{framed}
    % \BeforeBeginEnvironment{mdframed}{\begin{minipage}{\linewidth}}
     %\AfterEndEnvironment{mdframed}{\end{minipage}\par}


%	%\mdfsetup{%
%	%skipabove=20pt,
%	nobreak=true,
%	   middlelinecolor=black,
%	   middlelinewidth=1pt,
%	   backgroundcolor=purple!10,
%	   roundcorner=1pt}

\mdfsetup{%
	outerlinewidth=1,skipabove=20pt,backgroundcolor=yellow!50, outerlinecolor=black,innertopmargin=0pt,splittopskip=\topskip,skipbelow=\baselineskip, skipabove=\baselineskip,ntheorem,roundcorner=5pt}

\mdtheorem[nobreak=true,outerlinewidth=1,%leftmargin=40,rightmargin=40,
backgroundcolor=yellow!50, outerlinecolor=black,innertopmargin=0pt,splittopskip=\topskip,skipbelow=\baselineskip, skipabove=\baselineskip,ntheorem,roundcorner=5pt,font=\itshape]{result}{Result}


\mdtheorem[nobreak=true,outerlinewidth=1,%leftmargin=40,rightmargin=40,
backgroundcolor=yellow!50, outerlinecolor=black,innertopmargin=0pt,splittopskip=\topskip,skipbelow=\baselineskip, skipabove=\baselineskip,ntheorem,roundcorner=5pt,font=\itshape]{theorem}{Theorem}

\mdtheorem[nobreak=true,outerlinewidth=1,%leftmargin=40,rightmargin=40,
backgroundcolor=gray!10, outerlinecolor=black,innertopmargin=0pt,splittopskip=\topskip,skipbelow=\baselineskip, skipabove=\baselineskip,ntheorem,roundcorner=5pt,font=\itshape]{remark}{Remark}

\mdtheorem[nobreak=true,outerlinewidth=1,%leftmargin=40,rightmargin=40,
backgroundcolor=pink!30, outerlinecolor=black,innertopmargin=0pt,splittopskip=\topskip,skipbelow=\baselineskip, skipabove=\baselineskip,ntheorem,roundcorner=5pt,font=\itshape]{quaestio}{Quaestio}

\mdtheorem[nobreak=true,outerlinewidth=1,%leftmargin=40,rightmargin=40,
backgroundcolor=yellow!50, outerlinecolor=black,innertopmargin=5pt,splittopskip=\topskip,skipbelow=\baselineskip, skipabove=\baselineskip,ntheorem,roundcorner=5pt,font=\itshape]{background}{Background}

%TRYING TO INCLUDE Ppls IN TOC
\usepackage{hyperref}


\begin{document}
\title{\color{Brown} Perchè 5 settimane di contenimento possono fermare COVID-19 \\
\vspace{-0.35ex}}
\author{Chen Shen and Yaneer Bar-Yam \\ New England Complex Systems Institute \\
\vspace{+0.35ex}
\small{\textit{(tradotto da A. P. Rossi, P. Bonavita})}\\
 \today
  \vspace{-14ex} \\


\bigskip
\bigskip

\textbf{}
 }

\maketitle


\flushbottom % Makes all text pages the same height

%\maketitle % Print the title and abstract box

%\tableofcontents % Print the contents section

\thispagestyle{empty} % Removes page numbering from the first page

%----------------------------------------------------------------------------------------
%	ARTICLE CONTENTS
%----------------------------------------------------------------------------------------

%\section*{Introduction} % The \section*{} command stops section numbering

%\addcontentsline{toc}{section}{\hspace*{-\tocsep}Introduction} % Adds this section to the table of contents with negative horizontal space equal to the indent for the numbered sections

%\tableofcontents
%\section{ Introduction}
\renewcommand{\thefootnote}{\fnsymbol{footnote}}




\begin{multicols}{2}

Durante un contenimento rigido le persone restano a casa, a parte le uscite per procurarsi cibo ed altri beni essenziali, accedere alle cure mediche o condurre lavoro essenziale al funzionamento della società. I governi devono offrire supporto economico e sociale ai cittadini bisognosi.

Nel corso delle prime due settimane di lockdown, coloro che sono già infetti mostreranno sintomi. Questo "periodo di incubazione" tipicamente è di 3-5 giorni, ma può arrivare fino a due settimane. Gli individui contagiati in maniera lieve o guariranno da soli o andranno in cerca di cure mediche. Le uniche persone che possono essere contagiate in questa fase sono coloro che coabitano con un individuo già infetto. Siccome sappiamo quali individui sono infetti, o per i test o per i sintomi mostrati, sappiamo chi può venire infettato e possiamo isolarli così che non infettino a loro volta altre persone (questo è quello che si chiama tracciatura dei contatti). 

Durante le 3-4 settimane seguenti, ogni nuova persona infetta tra i familiari e tra coloro che vivono assieme ad individui infetti guarirà o cercherà cure mediche. Una volta isolati, non potranno infettare ulteriori individui. Il numero di casi scenderà rapidamente. Verso la fine del contenimento, i casi di COVID-19 saranno una piccola frazione del numero originale. Questo è esattamente quello che è successo in Cina.

Il contenimento fornisce anche tempo per aumentare in modo sostanziale le scorte di kit di test per il COVID-19, e la loro capacità produttiva. Se il numero di infezioni si riduce di molto usando il contenimento e cominciano i controlli di massa, COVID-19 può essere controllato dopo 5 settimane senza dover ricorrere a misure estreme di distanziamento sociale. Isolare gli individui malati e i loro contatti più prossimi sarà sufficiente. Questo è quanto è stato fatto per controllare l'epidemia per i casi registrati a Singapore.

Il caso dell'Italia può servire da avvertimento per chi volesse tentare un contenimento "soft". Le misure di contenimento in Italia sono state non sufficientemente rigide. Alcuni hanno aggirato le restrizioni di movimento e hanno continuato a diffondere COVID-19. La malattia ha continuato a crescere esponenzialmente. L'italia sta irrigidendo le sue procedure di contenimento per prevenire ulteriore diffusione. La Danimarca, che ha implementato un contenimento più completo e chiuso i propri confini, ha avuto maggiore successo nella limitazione dell'epidemia.

\end{multicols}

\begin{figure}[H]
\begin{centering}
\captionsetup{justification=centering}
\includegraphics[scale=0.85]{ChinaDynamics.png}
\caption{Dinamica dell'epidemia in Cina che mostra il momento del contenimento e il numero di casi per ogni provincia.}
\end{centering}
\end{figure}




% \bibliography{MyCollection.bib}


\end{document}
