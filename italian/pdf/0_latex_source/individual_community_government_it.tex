%\documentclass[twocolumn,journal]{IEEEtran}
\documentclass[onecolumn,journal]{IEEEtran}
\usepackage{amsfonts}
\usepackage{amsmath}
\usepackage{amsthm}
\usepackage{amssymb}
\usepackage{graphicx}
\usepackage[T1]{fontenc}
%\usepackage[english]{babel}
\usepackage{supertabular}
\usepackage{longtable}
\usepackage[usenames,dvipsnames]{color}
\usepackage{bbm}
%\usepackage{caption}
\usepackage{fancyhdr}
\usepackage{breqn}
\usepackage{fixltx2e}
\usepackage{capt-of}
%\usepackage{mdframed}
\setcounter{MaxMatrixCols}{10}
\usepackage{tikz}
\usetikzlibrary{matrix}
\usepackage{endnotes}
\usepackage{soul}
\usepackage{marginnote}
%\newtheorem{theorem}{Theorem}
\newtheorem{lemma}{Lemma}
%\newtheorem{remark}{Remark}
%\newtheorem{error}{\color{Red} Error}
\newtheorem{corollary}{Corollary}
\newtheorem{proposition}{Proposition}
\newtheorem{definition}{Definition}
\newcommand{\mathsym}[1]{}
\newcommand{\unicode}[1]{}
\newcommand{\dsum} {\displaystyle\sum}
\hyphenation{op-tical net-works semi-conduc-tor}
\usepackage{pdfpages}
\usepackage{enumitem}
\usepackage{multicol}
\usepackage[utf8]{inputenc}


\headsep = 5pt
\textheight = 730pt
%\headsep = 8pt %25pt
%\textheight = 720pt %674pt
%\usepackage{geometry}

\bibliographystyle{unsrt}

\usepackage{float}

 \usepackage{xcolor}

\usepackage[framemethod=TikZ]{mdframed}
%%%%%%%FRAME%%%%%%%%%%%
\usepackage[framemethod=TikZ]{mdframed}
\usepackage{framed}
    % \BeforeBeginEnvironment{mdframed}{\begin{minipage}{\linewidth}}
     %\AfterEndEnvironment{mdframed}{\end{minipage}\par}


%	%\mdfsetup{%
%	%skipabove=20pt,
%	nobreak=true,
%	   middlelinecolor=black,
%	   middlelinewidth=1pt,
%	   backgroundcolor=purple!10,
%	   roundcorner=1pt}

\mdfsetup{%
	outerlinewidth=1,skipabove=20pt,backgroundcolor=yellow!50, outerlinecolor=black,innertopmargin=0pt,splittopskip=\topskip,skipbelow=\baselineskip, skipabove=\baselineskip,ntheorem,roundcorner=5pt}

\mdtheorem[nobreak=true,outerlinewidth=1,%leftmargin=40,rightmargin=40,
backgroundcolor=yellow!50, outerlinecolor=black,innertopmargin=0pt,splittopskip=\topskip,skipbelow=\baselineskip, skipabove=\baselineskip,ntheorem,roundcorner=5pt,font=\itshape]{result}{Result}


\mdtheorem[nobreak=true,outerlinewidth=1,%leftmargin=40,rightmargin=40,
backgroundcolor=yellow!50, outerlinecolor=black,innertopmargin=0pt,splittopskip=\topskip,skipbelow=\baselineskip, skipabove=\baselineskip,ntheorem,roundcorner=5pt,font=\itshape]{theorem}{Theorem}

\mdtheorem[nobreak=true,outerlinewidth=1,%leftmargin=40,rightmargin=40,
backgroundcolor=gray!10, outerlinecolor=black,innertopmargin=0pt,splittopskip=\topskip,skipbelow=\baselineskip, skipabove=\baselineskip,ntheorem,roundcorner=5pt,font=\itshape]{remark}{Remark}

\mdtheorem[nobreak=true,outerlinewidth=1,%leftmargin=40,rightmargin=40,
backgroundcolor=pink!30, outerlinecolor=black,innertopmargin=0pt,splittopskip=\topskip,skipbelow=\baselineskip, skipabove=\baselineskip,ntheorem,roundcorner=5pt,font=\itshape]{quaestio}{Quaestio}

\mdtheorem[nobreak=true,outerlinewidth=1,%leftmargin=40,rightmargin=40,
backgroundcolor=yellow!50, outerlinecolor=black,innertopmargin=5pt,splittopskip=\topskip,skipbelow=\baselineskip, skipabove=\baselineskip,ntheorem,roundcorner=5pt,font=\itshape]{background}{Background}

%TRYING TO INCLUDE Ppls IN TOC
\usepackage{hyperref}


\begin{document}
\title{\color{Brown}  Linee guida per gli individui, le comunità ed i governi \\
Tradotto da: Early Outbreak Response Guidelines Version 2 \\
\vspace{-0.35ex}}
\author{Chen Shen and Yaneer Bar-Yam \\ New England Complex Systems Institute \\
\vspace{+0.35ex}
\small{\textit{(translated by A. P. Rossi, P. Bonavita, E. Lamberti})}\\
 \today
  \vspace{-8ex} \\
%\bigskip
\textbf{}
 }

\maketitle

%\vspace{-1ex}
%\flushbottom % Makes all text pages the same height

%\maketitle % Print the title and abstract box

%\tableofcontents % Print the contents section

\thispagestyle{empty} % Removes page numbering from the first page

%----------------------------------------------------------------------------------------
%	ARTICLE CONTENTS
%----------------------------------------------------------------------------------------

%\section*{Introduction} % The \section*{} command stops section numbering

%\addcontentsline{toc}{section}{\hspace*{-\tocsep}Introduction} % Adds this section to the table of contents with negative horizontal space equal to the indent for the numbered sections

%\tableofcontents
%\section{ Introduction}

%\section*{Overview}

L’epidemia di Coronavirus originata a Wuhan presenta circa il 20\% di casi gravi ed una mortalità del 2\%. Il periodo di incubazione tipico e’ di 3 giorni ma può arrivare a 14 giorni, e sono stati riportati casi di 24 o 27 giorni. La malattia è altamente contagiosa, con un aumento delle infezioni del +50\% ogni giorno (tasso di infezione R0 di circa 3-4), a meno di misure straordinarie. Se diventasse una pandemia o una malattia endemica cambierebbe la vita di chiunque, in tutto il mondo. E’ imperativo agire per confinare e fermare l’epidemia, non accettando la sua diffusione. Forniamo queste linee guide per l’azione di singoli individui,  comunità, amministrazioni e governi

\begin{multicols}{2}
\section*{Linee guida individuali e per la comunità}
\begin{itemize}
\item Prendersi cura della propria salute e condividere responsabilmente per la salute del vicinato, con coscienza e disciplina
\item Praticare distanziamento sociale
\item Evitare di toccare superfici in spazi pubblici o condivisi
\item Evitare assembramenti
\item Evitare contatti diretti con altri, lavarsi le mani regolarmente, indossare maschere protettive in ambienti chiusi con altri che potrebbero essere infetti
\item Coprire starnuti e tosse
\item Monitorare la temperatura o altri sintomi iniziali di infezione (tosse, starnuti, raffreddore, mal di gola)
\item Praticare isolamento volontario se hai sintomi iniziali
\item Se i sintomi continuano a svilupparsi, organizzare trasporto sicuro presso strutture sanitarie seguendo le raccomandazioni governative; evitare trasporti pubblici, indossare maschere protettive
\item In aree a rischio elevato occuparsi delle necessità dei membri della comunità
\item Collaborare con gli altri per creare zone e comunità sicure. Discutere sulla sicurezza con famiglia ed amici, parlare delle linee guida per la sicurezza, individuare chi le segue, creare regole condivise, continuare ad occuparsi e condividere i bisogni degli altri, le preoccupazioni e le opportunità
\item Essere scettici sulle dicerie e non diffondere disinformazione
\end{itemize}

\vspace{2ex}

\section*{Linee guida per amministrazioni locali}
\begin{itemize}
\item Qualora le regioni o i paesi confinanti abbiano infezioni attive, introdurre controlli per i sintomi ai confini
\item Condurre quarantene di 14 giorni per ogni individuo a rischio che si reca in aree libere da infezione
\item In aree di rischio elevato coordinare gruppi di vicinato per monitorare la comunità porta a porta per sintomi utilizzando termometri ad infrarosso e dispositivi di protezione
\item Gruppi di vicinato porta a porta devono anche identificare persone che hanno bisogno di supporto

\end{itemize}

\vspace{2ex}

\section*{Linee guida governative}
\begin{itemize}
\item Preparare in anticipo risorse strategiche come mascherine, dispositivi di protezione, kit per test e stabilire linee di distribuzione
\item Identificare aree dove le infezioni sono confermate o sospettate
\item Fermare i trasporti non essenziali tra le aree infette e quelle libere da infezione
\item Isolare gli individui con infezione sospetta o confermata separatamente per cura in strutture designate con risorse mediche adeguate, inclusi dispositivi di protezione
\item Persone con sintomi devono utilizzare un protocollo specifico per essere trasportate nelle strutture mediche per i test, evitando trasporti pubblici o taxi
\item Mettere in quarantena e testare tutti i casi sospetti nelle vicinanze di un caso identificato
\item Promuovere conoscenza pubblica di:
  \begin{itemize}
  \item Sintomi tipici e possibili mezzi di trasmissione
  \item Enfatizzare l’alto tasso di contagio e i sintomi generalmente lievi, per incoraggiare le persone a chiedere supporto medico
  \item Incoraggiare una migliore igiene personale, incluso il lavaggio frequente delle mani, l’uso di mascherine di protezione in aree pubbliche ed l’evitare contatto tra gli individu
  \end{itemize}
\item Impedire assembramenti pubblici
\item Prestare particolare attenzione per la prevenzione e il monitoraggio della salute delle persone che entrano/escono da strutture confinate ad alta densità abitativa, come prigioni, ospedali, centri di riabilitazione, ospizi, dormitori e ostelli
\item Promuovere la responsabilità delle comunità nelle aree infette
\item In ogni quartiere/comunità, selezionare un gruppo di persone la cui mansione quotidiana richiede frequente contatto umano. Monitorare le loro condizioni giornalmente per facilitare la scoperta di infezioni e prevenire il contagio
\item Portare avanti campagne di comunicazione e distribuzione di risorse nelle aree remote
\item Coordinare con la comunita’ internazionale e l’OMS per condividere informazioni sui casi di infezione identificati, la storia di viaggio dei pazienti, le cure, le strategie di prevenzione, la mancanza di equipaggiamento medico
\item Pianificare il trattamento anche di pazienti con sintomi simili a COVID-19 e che non sono infettati
\item In aree di trasmissione attiva:
  \begin{itemize}
  \item Chiudere i centri di culto, le universita, scuole ed aziende
  \item Tenere le persone a casa e fornire supporto per le loro necessità, da venire recapitate senza contatti personali
  \item Effettuare controlli porta a porta per identificare individui con sintomi iniziali e necessità, con i necessari dispositivi di protezione, anche con il supporto delle comunità locali
  \end{itemize}
\end{itemize}

\end{multicols}

\vspace{2ex}
Per ulteriori informazioni sui resoconti medici e sociali, consultare:
\begin{itemize}
\item WHO: \url{https://www.who.int/emergencies/diseases/novel-coronavirus-2019/technical-guidance}
\item Linee guida di Singapore: \url{https://www.moh.gov.sg/covid-19}
\end{itemize}






% \bibliography{MyCollection.bib}
\bibliography{references.bib}

\end{document}
