%\documentclass[twocolumn,journal]{IEEEtran}
\documentclass[onecolumn,journal]{IEEEtran}
\usepackage{ctex}
\usepackage{amsfonts}
\usepackage{amsmath}
\usepackage{amsthm}
\usepackage{amssymb}
\usepackage{graphicx}
\usepackage[T1]{fontenc}
%\usepackage[english]{babel}
\usepackage{supertabular}
\usepackage{longtable}
\usepackage[usenames,dvipsnames]{color}
\usepackage{bbm}
%\usepackage{caption}
\usepackage{fancyhdr}
\usepackage{breqn}
\usepackage{fixltx2e}
\usepackage{capt-of}
%\usepackage{mdframed}
\setcounter{MaxMatrixCols}{10}
\usepackage{tikz}
\usetikzlibrary{matrix}
\usepackage{endnotes}
\usepackage{soul}
\usepackage{marginnote}
%\newtheorem{theorem}{Theorem}
\newtheorem{lemma}{Lemma}
%\newtheorem{remark}{Remark}
%\newtheorem{error}{\color{Red} Error}
\newtheorem{corollary}{Corollary}
\newtheorem{proposition}{Proposition}
\newtheorem{definition}{Definition}
\newcommand{\mathsym}[1]{}
\newcommand{\unicode}[1]{}
\newcommand{\dsum} {\displaystyle\sum}
\hyphenation{op-tical net-works semi-conduc-tor}
\usepackage{pdfpages}
\usepackage{enumitem}
\usepackage{multicol}
\usepackage[utf8]{inputenc}

\headsep = 5pt
\textheight = 730pt
%\headsep = 8pt %25pt
%\textheight = 720pt %674pt
%\usepackage{geometry}

\bibliographystyle{unsrt}

\usepackage{float}

 \usepackage{xcolor}

\usepackage[framemethod=TikZ]{mdframed}
%%%%%%%FRAME%%%%%%%%%%%
\usepackage[framemethod=TikZ]{mdframed}
\usepackage{framed}
    % \BeforeBeginEnvironment{mdframed}{\begin{minipage}{\linewidth}}
     %\AfterEndEnvironment{mdframed}{\end{minipage}\par}


%	%\mdfsetup{%
%	%skipabove=20pt,
%	nobreak=true,
%	   middlelinecolor=black,
%	   middlelinewidth=1pt,
%	   backgroundcolor=purple!10,
%	   roundcorner=1pt}

\mdfsetup{%
	outerlinewidth=1,skipabove=20pt,backgroundcolor=yellow!50, outerlinecolor=black,innertopmargin=0pt,splittopskip=\topskip,skipbelow=\baselineskip, skipabove=\baselineskip,ntheorem,roundcorner=5pt}

\mdtheorem[nobreak=true,outerlinewidth=1,%leftmargin=40,rightmargin=40,
backgroundcolor=yellow!50, outerlinecolor=black,innertopmargin=0pt,splittopskip=\topskip,skipbelow=\baselineskip, skipabove=\baselineskip,ntheorem,roundcorner=5pt,font=\itshape]{result}{Result}


\mdtheorem[nobreak=true,outerlinewidth=1,%leftmargin=40,rightmargin=40,
backgroundcolor=yellow!50, outerlinecolor=black,innertopmargin=0pt,splittopskip=\topskip,skipbelow=\baselineskip, skipabove=\baselineskip,ntheorem,roundcorner=5pt,font=\itshape]{theorem}{Theorem}

\mdtheorem[nobreak=true,outerlinewidth=1,%leftmargin=40,rightmargin=40,
backgroundcolor=gray!10, outerlinecolor=black,innertopmargin=0pt,splittopskip=\topskip,skipbelow=\baselineskip, skipabove=\baselineskip,ntheorem,roundcorner=5pt,font=\itshape]{remark}{Remark}

\mdtheorem[nobreak=true,outerlinewidth=1,%leftmargin=40,rightmargin=40,
backgroundcolor=pink!30, outerlinecolor=black,innertopmargin=0pt,splittopskip=\topskip,skipbelow=\baselineskip, skipabove=\baselineskip,ntheorem,roundcorner=5pt,font=\itshape]{quaestio}{Quaestio}

\mdtheorem[nobreak=true,outerlinewidth=1,%leftmargin=40,rightmargin=40,
backgroundcolor=yellow!50, outerlinecolor=black,innertopmargin=5pt,splittopskip=\topskip,skipbelow=\baselineskip, skipabove=\baselineskip,ntheorem,roundcorner=5pt,font=\itshape]{background}{Background}

%TRYING TO INCLUDE Ppls IN TOC
\usepackage{hyperref}


\begin{document}
\title{\color{Brown}  企业应对新冠病毒指南
\vspace{-0.35ex}}
\author{Chen Shen 和 Yaneer Bar-Yam \\ 新英格兰复杂系统研究所 \\
\vspace{+0.35ex}
\small{\textit({翻译:Yanxia Fei  校对:David Xie})}\\
 \number 2020 年 \number 3 月 \number 2 日
  \vspace{-8ex} \\
%\bigskip
\textbf{}
 }

\maketitle

%\vspace{-1ex}
%\flushbottom % Makes all text pages the same height

%\maketitle % Print the title and abstract box

%\tableofcontents % Print the contents section

\thispagestyle{empty} % Removes page numbering from the first page

%----------------------------------------------------------------------------------------
%	ARTICLE CONTENTS
%----------------------------------------------------------------------------------------

%\section*{Introduction} % The \section*{} command stops section numbering

%\addcontentsline{toc}{section}{\hspace*{-\tocsep}Introduction} % Adds this section to the table of contents with negative horizontal space equal to the indent for the numbered sections

%\tableofcontents
%\section{ Introduction}

%\section*{Overview}


\begin{multicols}{2}

% \section

以下是企业防范新冠病毒传播时可以采取的一系列行动。我们还专门为零售业和酒店服务业提供了建议。


\section*{总方针}
\begin{itemize}
\item 提高员工及其家人对新冠病毒传播和预防的认识
\item 制定组织策略以减少传播,并一丝不苟地落实实施
\item 确保员工知道,即使他们症状轻微,也不应再到工作场所或者与他人面对面开会,保障员工不会因为请病假而受到处罚。设立一套应对各种情况的报告系统
\item 确保员工知道,即使他们症状轻微,也不应再到工作场所或者与他人面对面开会,保障员工不会因为请病假而受到处罚。设立一套应对各种情况的报告系统
\item 与当地的医疗机构合作,尽早为员工安排新冠病毒的快速检测
\item 在疫情恶化前就为员工准备好他们自己无法获取的必要物资(洗手液、酒精、口罩、红外线非接触式额温计)
\item 加固组织中最薄弱的环节,减少漏洞
\end{itemize}

\section*{会议、出差和访客}
\begin{itemize}
\item 用线上会议替代面对面的线下会议
\item 有可能的话安排员工在居家工作
\item 限制前往较高风险区域(红区、橙区乃至黄区)出差
\item 取消不必要的出差
\item 改变经营方式,把看似必要的出差转变成不必要
\item 限制访客,并制定相应的政策,根据访客来源地的疫情状况和对方公司应对新冠病毒的政策来询问访客和取消拜访计划。访客到达时检测其有无出现相关症状
\end{itemize}

\section*{办公场所}
\begin{itemize}
\item 提倡灵活安排工作时间,安排错峰上班和轮班,从而降低工作场所的员工密度。在一定时间内,办公场所的人口密度应该小于 50%
\item 雇主应要求从有确诊病例地区回来或者在出差过程中不确定有无接触到感染者的员工,返回办公室前进行 14 天的自我隔离。员工也需密切注意自身健康状况,出现症状后应及时报告并就医
\item 应该在办公场所入口处安排相关员工用红外非接触式额温计为进入人员测温
\item 每天为员工测量体温,并为不能避免接触其他同事的员工发放口罩 \footnote{1.目前针对口罩的使用存在争议。我们注意到:(1)任何轻症患者都应该避免与他人接触,在不得不与他人接触的公共或私人场合都应该佩戴口罩。(2)应该接受他人在公共场所应佩戴口罩,以防受感染者因为犹豫是否合适或者害怕被指责而不敢在公共场所佩戴口罩。(3)虽然口罩不能完全保障健康个体的安全,并且因为相关医护人员与医疗机构更优先需要口罩,所以其他人可能买不到口罩,但是对那些无法避免与潜在感染者接触的人而言,佩戴口罩会大大降低感染风险。(4)对 50 岁以上或者有基础疾病的人群,以及疫情风险升高地区的普遍人群而言,感染病毒都将对他们造成巨大的损伤,因此最好预先佩戴口罩。}
\item 改径车辆进入大楼的通道,提倡进入办公场所前就洗手,并在办公室入口处放置洗手液
\item 安排员工不要聚集在一起乘坐电梯。乘坐电梯的人数不应超过电梯承载量的一半
\item 确保每个员工的工位之间都保持至少 3 英尺的距离,每个人的办公区域不小于 25 平方英尺。员工数量较多的办公场所应进一步拉大员工之间的空间距离
\item 对公共区域、交通繁忙地段、经常有人触摸的表面进行消毒
\item 如果必须使用空调,请避免循环室内空气。每周清洁、消毒、更换空调的关键组件和滤网
\item 分散就餐,就餐时人与人保持 3 英尺距离,避免面对面就坐。分开放置餐具并经常消毒。必须经常检查餐厅食堂工作人员的健康状况
\item 鼓励外卖,而不是外出就餐。协助员工叫外卖,在不限制进出的卫生区域内设立无接触外卖取货点
\item 把员工的通勤方式纳入考虑,并据此提供建议,包括避免乘坐公共交通,或者注意卫生习惯,不触碰疫情高风险地区公共场所的物体表面,勤洗手,在疫情风险增加的地区佩戴口罩
\item 应对新冠肺炎疫情、保障工作场所安全的一系列政策措施必须明确落实责任
\end{itemize}
%
\section*{零售业和酒店服务业}
\begin{itemize}
\item 人员接触频率较高的行业,其经营状况可能因疫情而受到严重影响。尽早采取有效的干预也许能缓和风险,但除非全社会都采取相应的措施,要不然风险无法降低到零
\item 必须向员工反复强调,即使出现的是轻微的感冒症状,也必须避免与他人接触
\item 清楚记录员工每天的接触者。这么做的话,如果发现感染,企业就可以向所有可能暴露在病毒下的接触者发出警报,从而最大程度地降低风险,并减轻对员工和客户的伤害
\item 应该开发并实施无接触的经营方式:
  \begin{itemize}
    \item 开展窗口取餐和送餐服务,确保排队等候的顾客之间保持足够的间距
    \item 提供路过式服务
    \item 提供无接触外卖服务
  \end{itemize}
\end{itemize}


\end{multicols}

\vspace{2ex}







% \bibliography{MyCollection.bib}
\bibliography{references.bib}

\end{document}
