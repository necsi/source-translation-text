%\documentclass[twocolumn,journal]{IEEEtran}
\documentclass[onecolumn,journal]{IEEEtran}
\usepackage{ctex}
\usepackage{amsfonts}
\usepackage{amsmath}
\usepackage{amsthm}
\usepackage{amssymb}
\usepackage{graphicx}
\usepackage[T1]{fontenc}
%\usepackage[english]{babel}
\usepackage{supertabular}
\usepackage{longtable}
\usepackage[usenames,dvipsnames]{color}
\usepackage{bbm}
%\usepackage{caption}
\usepackage{fancyhdr}
\usepackage{breqn}
\usepackage{fixltx2e}
\usepackage{capt-of}
%\usepackage{mdframed}
\setcounter{MaxMatrixCols}{10}
\usepackage{tikz}
\usetikzlibrary{matrix}
\usepackage{endnotes}
\usepackage{soul}
\usepackage{marginnote}
%\newtheorem{theorem}{Theorem}
\newtheorem{lemma}{Lemma}
%\newtheorem{remark}{Remark}
%\newtheorem{error}{\color{Red} Error}
\newtheorem{corollary}{Corollary}
\newtheorem{proposition}{Proposition}
\newtheorem{definition}{Definition}
\newcommand{\mathsym}[1]{}
\newcommand{\unicode}[1]{}
\newcommand{\dsum} {\displaystyle\sum}
\hyphenation{op-tical net-works semi-conduc-tor}
\usepackage{pdfpages}
\usepackage{enumitem}
\usepackage{multicol}
\usepackage[utf8]{inputenc}


\headsep = 5pt
\textheight = 730pt
%\headsep = 8pt %25pt
%\textheight = 720pt %674pt
%\usepackage{geometry}

\bibliographystyle{unsrt}

\usepackage{float}

 \usepackage{xcolor}

\usepackage[framemethod=TikZ]{mdframed}
%%%%%%%FRAME%%%%%%%%%%%
\usepackage[framemethod=TikZ]{mdframed}
\usepackage{framed}
    % \BeforeBeginEnvironment{mdframed}{\begin{minipage}{\linewidth}}
     %\AfterEndEnvironment{mdframed}{\end{minipage}\par}


%	%\mdfsetup{%
%	%skipabove=20pt,
%	nobreak=true,
%	   middlelinecolor=black,
%	   middlelinewidth=1pt,
%	   backgroundcolor=purple!10,
%	   roundcorner=1pt}

\mdfsetup{%
	outerlinewidth=1,skipabove=20pt,backgroundcolor=yellow!50, outerlinecolor=black,innertopmargin=0pt,splittopskip=\topskip,skipbelow=\baselineskip, skipabove=\baselineskip,ntheorem,roundcorner=5pt}

\mdtheorem[nobreak=true,outerlinewidth=1,%leftmargin=40,rightmargin=40,
backgroundcolor=yellow!50, outerlinecolor=black,innertopmargin=0pt,splittopskip=\topskip,skipbelow=\baselineskip, skipabove=\baselineskip,ntheorem,roundcorner=5pt,font=\itshape]{result}{Result}


\mdtheorem[nobreak=true,outerlinewidth=1,%leftmargin=40,rightmargin=40,
backgroundcolor=yellow!50, outerlinecolor=black,innertopmargin=0pt,splittopskip=\topskip,skipbelow=\baselineskip, skipabove=\baselineskip,ntheorem,roundcorner=5pt,font=\itshape]{theorem}{Theorem}

\mdtheorem[nobreak=true,outerlinewidth=1,%leftmargin=40,rightmargin=40,
backgroundcolor=gray!10, outerlinecolor=black,innertopmargin=0pt,splittopskip=\topskip,skipbelow=\baselineskip, skipabove=\baselineskip,ntheorem,roundcorner=5pt,font=\itshape]{remark}{Remark}

\mdtheorem[nobreak=true,outerlinewidth=1,%leftmargin=40,rightmargin=40,
backgroundcolor=pink!30, outerlinecolor=black,innertopmargin=0pt,splittopskip=\topskip,skipbelow=\baselineskip, skipabove=\baselineskip,ntheorem,roundcorner=5pt,font=\itshape]{quaestio}{Quaestio}

\mdtheorem[nobreak=true,outerlinewidth=1,%leftmargin=40,rightmargin=40,
backgroundcolor=yellow!50, outerlinecolor=black,innertopmargin=5pt,splittopskip=\topskip,skipbelow=\baselineskip, skipabove=\baselineskip,ntheorem,roundcorner=5pt,font=\itshape]{background}{Background}

%TRYING TO INCLUDE Ppls IN TOC
\usepackage{hyperref}


\begin{document}
\title{\color{Brown} 大规模检测可以阻止新冠病毒爆发 \\
\vspace{-0.35ex}}
\author{Chen Shen 和 Yaneer Bar-Yam \\ 新英格兰复杂系统研究所 \\
\vspace{+0.35ex}
\small{\textit({翻译:David Xie  校对:Yanxia Fei})}\\
 \number 2020 年 \number 3 月 \number 6 日
  \vspace{-14ex} \\


\bigskip
\bigskip

\textbf{}
 }

\maketitle


\flushbottom % Makes all text pages the same height

%\maketitle % Print the title and abstract box

%\tableofcontents % Print the contents section

\thispagestyle{empty} % Removes page numbering from the first page

%----------------------------------------------------------------------------------------
%	ARTICLE CONTENTS
%----------------------------------------------------------------------------------------

%\section*{Introduction} % The \section*{} command stops section numbering

%\addcontentsline{toc}{section}{\hspace*{-\tocsep}Introduction} % Adds this section to the table of contents with negative horizontal space equal to the indent for the numbered sections

%\tableofcontents
%\section{ Introduction}
\renewcommand{\thefootnote}{\fnsymbol{footnote}}




\begin{multicols}{2}

  为了阻止爆发,我们必须阻止传播。关键策略是辨识患该疾病的个人并隔离他们,以使他人不被感染。如果检测手段有缺陷,部分没有感染的人也会被要求隔离。他们将成为检测结果为假阳性的个体。这会增加社会负担, 但仍能阻止病毒的爆发。另一方面,我们也容许一定数量的假阴性测试结果。只要假阴性数量足够小,随着时间的推移,新增病例就会下降,病毒爆发趋势会呈指数级下降。每个患者感染的个体数量越多(感染率或 R0),可容许的假阴性率就越低。

  鉴别潜在感染者的方法越精确越好,因为这样可以减少被隔离的人数。另一方面,即使更多的人被隔离,漏诊的病例越少,爆发就会越快消失,患者和死亡病例就越少。

我们阻止爆发的各种手段与这个总体框架有何关系呢?这里有一些例子:

\begin{itemize}
  \item \textbf{自我报告和诊断:} 使用这种检测手段,个人首先必须确定自己出现了需要就医的症状,然后向医生报告。医生将对个人进行诊断,一旦确诊(会有部分假阳性和假阴性结果),患者就将被隔离。那些出现症状却不自我报告的人即为假阴性。那些被误诊为感染者的人即为假阳性。通常,最严峻的困难是让自行判断而对症状产生误判所导致的假阴性:即那些已经感染了但没有辨识出来并且不进行自我报告的人。出现这种情况可能是因为症状不典型或不特殊,或者直接危及生命。也有可能是,个人可能怀疑自己已经感染,但是出于个人、经济、社会或职业方面的原因,他们不选择及时问诊,或者得不到接受诊断和隔离的机会。(其他问题还包括,流程中的不同阶段,从送去检测,正在检测,再到隔离的过程中是否会导致新感染者的出现。例如在运输过程中或在医疗机构中发生感染,以及有无严格隔离。)

  \item \textbf{接触史追踪:} 使用这种检测手段,可以识别与确诊者接触过的个体(根据自我报告和诊断方法)。观察他们是否出现症状,或直接进行隔离。即使他们没有感染(包括许多没有真正被感染的人,即假阳性),对他们进行隔离也有助于阻止爆发。

  \item \textbf{封锁——社区地域辨识:} 使用这种检测手段,把已感染人群所在的社区地域范围内的所有成员都视为潜在感染者,并进行隔离。这么做会隔离许多假阳性个体,但可以阻止疫情爆发。

  \item \textbf{邻里之间的一般症状检测:} 使用这种检测手段,进一步检测已感染人群所在社区地域范围内所有成员是否有发热(可能与其他疾病有关)等可能感染病毒后出现的症状。如果发现出现相应症状的个体,把他们视为潜在感染者加以隔离。与封锁相比,这一手段的优势在于能减少隔离人数,从而降低社会成本。与进行具体诊断相比,这一手段的优势是可以隔离更多已感染的个体。这一方法曾有效地遏制了埃博拉疫情在利比里亚和塞拉利昂的爆发 [1]。

  \item \textbf{大规模特定检测:} 使用这种检测手段,即对广泛的人群进行 DNA 或其他特定检测,也可能集中应用在某个特定的地理区域内,以识别潜在感染者并对其实行隔离。如果检测结果足够精确并广泛应用,就可以阻止爆发。

  \item \textbf{有针对性的随机抽样:} 使用这种检测手段,对密切往来的人群进行诊断或 DNA 检测。例如监狱、宿舍、旅馆、疗养院、康复设施、精神病房、医疗机构或退休社区。在那些区域,只要有一人感染,即使尚未表现出症状或还在等待阳性检测结果,仍然可能导致许多人被感染。在这种情况下,整个社区需要被隔离(对每个人进行独立隔离,而不是一群人集中隔离),以防止病毒进一步传播。


\end{itemize}

更广意地说,我们发现,通过使用各种检测手段,包括针对症状、地理位置的检测,或分子检测,如果该方法能鉴别被感染的人群,即使检测结果中有假阳性,只要假阴性的数量足够少,该检测手段就可以用来阻止传染。

除了执行检测之外,一个关键的问题是,如何能尽早判断某个人是否属于需要被隔离的群体,以及判断的早晚对于他们还能感染多少人的影响。如果在他们变得具有传染性之前就把他们检测出来,或者他们只在很短一段时间内可能对他人产生传染,那么这样就足以阻止传染,从而阻止疫情爆发。病毒携带者在感染其他人期间就类似于检测结果呈假阴性的患者。这段时间的感染人数会增加,也会降低为阻止疫情爆发而进行的检测的效率。在这种情形下,无论何种检测手段,包括针对症状、地理位置的检测或分子检测,对于有效地阻止爆发、降低患病及死亡人数而言,尽早并快速地实施任何手段都至关重要。

在应对 COVID-19 新冠肺炎疫情(爆发于中国武汉)的过程中,有一项使用鼻拭子或咽拭子进行特定DNA检测的技术,可以迅速识别出感染者。隔离检测结果呈阳性的人,则可以大大降低传染率。该项检测需要几天才能得到结果。更快的检测手段正在研发。

疫情爆发初期,得到检测的人数严重不足。在这种情形下,必须以社区范围进行隔离。这种隔离手段在中国,尤其在武汉得到采用,效果显著,为大规模的惯常接触史追踪工作(670,000人)[2] 提供了支持。随后在韩国,他们在实施封锁的同时 [3],也在开展更大规模的检测。在某些地方,人们无需下车就可方便地接受检测 [4]。最近有迹象表明,韩国的疫情已得到控制 [5]。

目前在世界上许多地方,包括美国在内,检测试剂盒不足以支持开展大规模检测。这限制了我们使用这种方法的能力。尽管如此,原则上可以以不高的造价快速地生产检测试剂盒,然后进行大规模检测以鉴别感染者。这样就可以降低使用封锁等其他手段的必要性。一旦大量检测得以实施,按照预期,大规模的特定检测就可以达到阻止爆发的效果。

我们应该做什么?减缓或阻止新冠病毒爆发的措施涉及多种因素。个人、家庭和社区应采取预防措施,避免人员接触,降低自身和他人被感染的可能性。同时,医疗机构应该放宽对能接受检测人员的诊断标准,并提供跨地理区域的便捷来往方式来加强检测的力度和广度。条件允许的话,公司或非政府组织也可以在便于人抵达的地点(甚至挨家挨户地)提供检测服务,以便快速识别感染者。中国在疫情爆发的最后阶段就是这样做的 [6]。

\section*{参考资料}

[1] \href{https://necsi.edu/how-community-response-stopped-ebola}{How community response stopped ebola}

[2] \href{https://www.who.int/docs/default-source/coronaviruse/who-china-joint-mission-on-covid-19-final-report.pdf}{Report of the WHO-China Joint Mission on Coronavirus Disease 2019 (COVID-19)}

[3] \href{http://english.chosun.com/site/data/html_dir/2020/02/24/2020022401353.html}{Daegu in Lockdown as Coronavirus Infections Soar}

[4] \href{https://www.cnn.com/2020/03/02/asia/coronavirus-drive-through-south-korea-hnk-intl/index.html}{South Korea pioneers coronavirus drive-through testing station}

[5] \href{https://twitter.com/yaneerbaryam/status/1235734017699430401?s=20}{BREAKING: Coronavirus Update. Significant decline in daily new cases in South Korea--positive sign of gaining control.}

[6] \href{https://www.courthousenews.com/china-goes-door-to-door-in-wuhan-seeking-infections/}{China Goes Door to Door in Wuhan, Seeking Infections}




\end{multicols}



% \bibliography{MyCollection.bib}


\end{document}
