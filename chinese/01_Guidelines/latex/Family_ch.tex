%\documentclass[twocolumn,journal]{IEEEtran}
\documentclass[onecolumn,journal]{IEEEtran}
\usepackage{ctex}
\usepackage{amsfonts}
\usepackage{amsmath}
\usepackage{amsthm}
\usepackage{amssymb}
\usepackage{graphicx}
\usepackage[T1]{fontenc}
%\usepackage[english]{babel}
\usepackage{supertabular}
\usepackage{longtable}
\usepackage[usenames,dvipsnames]{color}
\usepackage{bbm}
%\usepackage{caption}
\usepackage{fancyhdr}
\usepackage{breqn}
\usepackage{fixltx2e}
\usepackage{capt-of}
%\usepackage{mdframed}
\setcounter{MaxMatrixCols}{10}
\usepackage{tikz}
\usetikzlibrary{matrix}
\usepackage{endnotes}
\usepackage{soul}
\usepackage{marginnote}
%\newtheorem{theorem}{Theorem}
\newtheorem{lemma}{Lemma}
%\newtheorem{remark}{Remark}
%\newtheorem{error}{\color{Red} Error}
\newtheorem{corollary}{Corollary}
\newtheorem{proposition}{Proposition}
\newtheorem{definition}{Definition}
\newcommand{\mathsym}[1]{}
\newcommand{\unicode}[1]{}
\newcommand{\dsum} {\displaystyle\sum}
\hyphenation{op-tical net-works semi-conduc-tor}
\usepackage{pdfpages}
\usepackage{enumitem}
\usepackage{multicol}
\usepackage[utf8]{inputenc}


\headsep = 5pt
\textheight = 730pt
%\headsep = 8pt %25pt
%\textheight = 720pt %674pt
%\usepackage{geometry}

\bibliographystyle{unsrt}

\usepackage{float}

 \usepackage{xcolor}

\usepackage[framemethod=TikZ]{mdframed}
%%%%%%%FRAME%%%%%%%%%%%
\usepackage[framemethod=TikZ]{mdframed}
\usepackage{framed}
    % \BeforeBeginEnvironment{mdframed}{\begin{minipage}{\linewidth}}
     %\AfterEndEnvironment{mdframed}{\end{minipage}\par}


%	%\mdfsetup{%
%	%skipabove=20pt,
%	nobreak=true,
%	   middlelinecolor=black,
%	   middlelinewidth=1pt,
%	   backgroundcolor=purple!10,
%	   roundcorner=1pt}

\mdfsetup{%
	outerlinewidth=1,skipabove=20pt,backgroundcolor=yellow!50, outerlinecolor=black,innertopmargin=0pt,splittopskip=\topskip,skipbelow=\baselineskip, skipabove=\baselineskip,ntheorem,roundcorner=5pt}

\mdtheorem[nobreak=true,outerlinewidth=1,%leftmargin=40,rightmargin=40,
backgroundcolor=yellow!50, outerlinecolor=black,innertopmargin=0pt,splittopskip=\topskip,skipbelow=\baselineskip, skipabove=\baselineskip,ntheorem,roundcorner=5pt,font=\itshape]{result}{Result}


\mdtheorem[nobreak=true,outerlinewidth=1,%leftmargin=40,rightmargin=40,
backgroundcolor=yellow!50, outerlinecolor=black,innertopmargin=0pt,splittopskip=\topskip,skipbelow=\baselineskip, skipabove=\baselineskip,ntheorem,roundcorner=5pt,font=\itshape]{theorem}{Theorem}

\mdtheorem[nobreak=true,outerlinewidth=1,%leftmargin=40,rightmargin=40,
backgroundcolor=gray!10, outerlinecolor=black,innertopmargin=0pt,splittopskip=\topskip,skipbelow=\baselineskip, skipabove=\baselineskip,ntheorem,roundcorner=5pt,font=\itshape]{remark}{Remark}

\mdtheorem[nobreak=true,outerlinewidth=1,%leftmargin=40,rightmargin=40,
backgroundcolor=pink!30, outerlinecolor=black,innertopmargin=0pt,splittopskip=\topskip,skipbelow=\baselineskip, skipabove=\baselineskip,ntheorem,roundcorner=5pt,font=\itshape]{quaestio}{Quaestio}

\mdtheorem[nobreak=true,outerlinewidth=1,%leftmargin=40,rightmargin=40,
backgroundcolor=yellow!50, outerlinecolor=black,innertopmargin=5pt,splittopskip=\topskip,skipbelow=\baselineskip, skipabove=\baselineskip,ntheorem,roundcorner=5pt,font=\itshape]{background}{Background}

%TRYING TO INCLUDE Ppls IN TOC
\usepackage{hyperref}


\begin{document}
\title{\color{Brown} 家庭指南 \\
\vspace{-0.35ex}}
\author{Chen Shen 和 Yaneer Bar-Yam \\ 新英格兰复杂系统研究所 \\
\vspace{+0.35ex}
\small{\textit({翻译:Yanxia Fei  校对:David Xie})}\\
 \number 2020 年 \number 3 月 \number 7 日
  \vspace{-14ex} \\


\bigskip
\bigskip

\textbf{}
 }

\maketitle


\flushbottom % Makes all text pages the same height

%\maketitle % Print the title and abstract box

%\tableofcontents % Print the contents section

\thispagestyle{empty} % Removes page numbering from the first page

%----------------------------------------------------------------------------------------
%	ARTICLE CONTENTS
%----------------------------------------------------------------------------------------

%\section*{Introduction} % The \section*{} command stops section numbering

%\addcontentsline{toc}{section}{\hspace*{-\tocsep}Introduction} % Adds this section to the table of contents with negative horizontal space equal to the indent for the numbered sections

%\tableofcontents
%\section{ Introduction}
\renewcommand{\thefootnote}{\fnsymbol{footnote}}




\begin{multicols}{2}


高风险地区的政府如果不采取相应的行动,那么要在这些地方保护所有家庭成员或一群人将变得很有挑战。火势的蔓延离不开一连串可燃物的助燃。同样,COVID-19 疫情扩散源于感染了一系列易感个体。阻断扩散的解决方案如下:(1)减少家庭成员与其他人之间的接触,并提供基本所需;随着风险的增加,(2)创建安全空间,按照共享协议保护其中的居民:在没有适当防护的条件下,避免与他人进行身体接触,也不要直接触摸其他人摸过的物体表面。


创建安全空间也可以抑制传染,因为安全空间中的人并没有参与传播病毒。一个安全空间中的成员可以与其他安全空间的成员组合,小心翼翼地扩展已有的安全空间或创建新的空间。以下是我们为家庭提供的指南。

减少家庭成员与其他人的接触:

\begin{itemize}

\item 仔细阅读我们提供的个人指南,并与家庭成员分享。与家人讨论如何减少与其他人的接触。

\item 把线下家庭聚会转移到线上。目前的疫情要么能得到控制,要么会继续广泛扩散。如果是前一种情况,那么几个月后一切都将恢复正常。如果是后一种情况,那么我们需要采取不同的行动。

\item 确保您和家人拥有包括处方药在内的必要物品。对于家庭中有较高感染风险的成员,包括老年人、50岁 以上的任何人,以及患有慢性疾病的家人,要衡量他们与外界接触的风险。减少他们与外界的接触,为他们提供支持,让他们能够待在家里,不需要前往公共场所。

\item 考虑安排居住在集体宿舍(退休社区、辅助生活设施等)的个人暂时搬到私人住宅或小型组织机构等独立住房内。

\item 如果无法减少接触,请与负责管理集体住宿设施的人员交谈,加强预防疫情传播的措施。
\item 避免公共集会或前往餐厅等公共场所,尤其需要避免前往密闭的公共空间。

\end{itemize}

在高风险地区创建安全空间:

\begin{itemize}

\item 创建安全空间的主要目的是让一群人组成一个单独的单位,把与安全空间外的其他人发生身体接触的机会降到最低,同时这个空间能够自立自给。

\item 个人不必等待政府指导的自上而下的安全行动。在缺乏积极、系统的干预措施的情况下,自发组织、自下而上的安全空间也能帮助个人。逐步扩大安全区域的范围,这可以减缓甚至阻断疫情在当地的扩散。

\item 可以从共享单一住所的家庭或人群开始创建安全空间。如果制定了安全协议并严格遵守,多个安全空间便可以组合在一起,并允许在这些区域之间通行(例如步行或开车)。为了成功地建立起安全空间,每个参与者都必须同意尽可能地减少外部实体接触,并严格遵循这一原则。安全协议还必须为如何采取行动和合作给出明确的指示。同一安全空间内的成员应该开诚布公地告知其他人自己的旅行史和健康状况,也为其他人的健康负责。

\item 为了使个人能够致力于维护共享的安全空间,大家可能还需要安排好工作、学校、家人和朋友。必要时请在企业雇主的允许下或者请假居家。

\item 按照在安全空间内工作、学习和生活一小段时间(至少一周或一周以上)提前规划好补充生活用品等事宜。采购必需品时需格外注意,因为可能接触到其他人群。在这种情况下,采用生存主义策略可能会有所帮助。有意识地提前安排好生活必需品的采购计划至关重要,因为每次外出购物都涉及一定的风险。

\item 条件允许的话,请以外卖形式订购食物和其他物品,从而减少外出采购。接收快递时务必小心,因为任何物品肯定都经由他人之手处理。在疫情仍在扩散的地区,除非供应商规定快递员必须戴好手套,否则收到物品后请先主动清洗或消毒。

\item 必须外出(比如采购)时,不可避免地会接触其他人,所以安全空间内的成员应该提前计划,高效行动,尽量缩短外出时间和减少接触范围。离开和返回安全空间时都需要格外小心。请适当做好个人防护,戴好手套或使用一次性物品(纸巾)抓取或处理不应直接触摸的物品,使用消毒洗手液或酒精清洁双手,还需要佩戴口罩。尽量在返回安全空间前或在入口处就进行清洗或消毒。

\item 促进内部沟通,相互照顾,使安全空间内的成员保持积极的关系和心理健康。成员们需要认识到,为了应对当前的紧急状况,每个人都有必要采取非常行动和做出一定的牺牲。疫情可能会有所缓和,但相互支持的重要性在任何时候都无可取代。

\item 安全空间内的成员应该要了解一个或多个成员出现感染症状时需要采取什么行动。具体行动因国家/州/具体位置而异,并且需要根据现状的变化随时调整行动。成员应该向安全空间内的每个人普及最新的应急计划和联系信息。任何成员表现出典型症状的话,其他人都应该迅速采取行动,帮助他/她申请检测,并在得到检测结果前进行预防隔离。

\item 随着疫情不断扩散,安全空间内的成员将不可避免地遇到以下情况,即是否需要退出安全空间去帮助不在安全空间内的家人和朋友。要做出这些决定很艰难,个人应该为此做好准备。

\item 在高风险时期,有人可能会错误地采取危及安全的行为。为了避免人们对单一事件过度反应,必须认识到任何单一行为造成的伤害都很有限。但是,多个个人行为叠加之后,风险就会显著增加。请确保吸取教训,这比指控、指责或惩罚更重要。

\end{itemize}



\end{multicols}



% \bibliography{MyCollection.bib}


\end{document}
