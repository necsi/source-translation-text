%\documentclass[twocolumn,journal]{IEEEtran}
\documentclass[onecolumn,journal]{IEEEtran}
\usepackage{ctex}
\usepackage{amsfonts}
\usepackage{amsmath}
\usepackage{amsthm}
\usepackage{amssymb}
\usepackage{graphicx}
\usepackage[T1]{fontenc}
%\usepackage[english]{babel}
\usepackage{supertabular}
\usepackage{longtable}
\usepackage[usenames,dvipsnames]{color}
\usepackage{bbm}
%\usepackage{caption}
\usepackage{fancyhdr}
\usepackage{breqn}
\usepackage{fixltx2e}
\usepackage{capt-of}
%\usepackage{mdframed}
\setcounter{MaxMatrixCols}{10}
\usepackage{tikz}
\usetikzlibrary{matrix}
\usepackage{endnotes}
\usepackage{soul}
\usepackage{marginnote}
%\newtheorem{theorem}{Theorem}
\newtheorem{lemma}{Lemma}
%\newtheorem{remark}{Remark}
%\newtheorem{error}{\color{Red} Error}
\newtheorem{corollary}{Corollary}
\newtheorem{proposition}{Proposition}
\newtheorem{definition}{Definition}
\newcommand{\mathsym}[1]{}
\newcommand{\unicode}[1]{}
\newcommand{\dsum} {\displaystyle\sum}
\hyphenation{op-tical net-works semi-conduc-tor}
\usepackage{pdfpages}
\usepackage{enumitem}
\usepackage{multicol}
\usepackage[utf8]{inputenc}


\headsep = 5pt
\textheight = 730pt
%\headsep = 8pt %25pt
%\textheight = 720pt %674pt
%\usepackage{geometry}

\bibliographystyle{unsrt}

\usepackage{float}

 \usepackage{xcolor}

\usepackage[framemethod=TikZ]{mdframed}
%%%%%%%FRAME%%%%%%%%%%%
\usepackage[framemethod=TikZ]{mdframed}
\usepackage{framed}
    % \BeforeBeginEnvironment{mdframed}{\begin{minipage}{\linewidth}}
     %\AfterEndEnvironment{mdframed}{\end{minipage}\par}


%	%\mdfsetup{%
%	%skipabove=20pt,
%	nobreak=true,
%	   middlelinecolor=black,
%	   middlelinewidth=1pt,
%	   backgroundcolor=purple!10,
%	   roundcorner=1pt}

\mdfsetup{%
	outerlinewidth=1,skipabove=20pt,backgroundcolor=yellow!50, outerlinecolor=black,innertopmargin=0pt,splittopskip=\topskip,skipbelow=\baselineskip, skipabove=\baselineskip,ntheorem,roundcorner=5pt}

\mdtheorem[nobreak=true,outerlinewidth=1,%leftmargin=40,rightmargin=40,
backgroundcolor=yellow!50, outerlinecolor=black,innertopmargin=0pt,splittopskip=\topskip,skipbelow=\baselineskip, skipabove=\baselineskip,ntheorem,roundcorner=5pt,font=\itshape]{result}{Result}


\mdtheorem[nobreak=true,outerlinewidth=1,%leftmargin=40,rightmargin=40,
backgroundcolor=yellow!50, outerlinecolor=black,innertopmargin=0pt,splittopskip=\topskip,skipbelow=\baselineskip, skipabove=\baselineskip,ntheorem,roundcorner=5pt,font=\itshape]{theorem}{Theorem}

\mdtheorem[nobreak=true,outerlinewidth=1,%leftmargin=40,rightmargin=40,
backgroundcolor=gray!10, outerlinecolor=black,innertopmargin=0pt,splittopskip=\topskip,skipbelow=\baselineskip, skipabove=\baselineskip,ntheorem,roundcorner=5pt,font=\itshape]{remark}{Remark}

\mdtheorem[nobreak=true,outerlinewidth=1,%leftmargin=40,rightmargin=40,
backgroundcolor=pink!30, outerlinecolor=black,innertopmargin=0pt,splittopskip=\topskip,skipbelow=\baselineskip, skipabove=\baselineskip,ntheorem,roundcorner=5pt,font=\itshape]{quaestio}{Quaestio}

\mdtheorem[nobreak=true,outerlinewidth=1,%leftmargin=40,rightmargin=40,
backgroundcolor=yellow!50, outerlinecolor=black,innertopmargin=5pt,splittopskip=\topskip,skipbelow=\baselineskip, skipabove=\baselineskip,ntheorem,roundcorner=5pt,font=\itshape]{background}{Background}

%TRYING TO INCLUDE Ppls IN TOC
\usepackage{hyperref}


\begin{document}
\title{\color{Brown}  个人、社区和政府 \linebreak 应对早期爆发疫情的指南 第三版 \\
\vspace{-0.35ex}}
\author{Chen Shen 和 Yaneer Bar-Yam \\ 新英格兰复杂系统研究所 \\
\vspace{+0.35ex}
\small{\textit({翻译:Yanxia Fei 校对:David Xie})}\\
 \number 2020 年  \number 3 月 \number 2 日
  \vspace{-8ex} \\
%\bigskip
\textbf{}
 }

\maketitle

%\vspace{-1ex}
%\flushbottom % Makes all text pages the same height

%\maketitle % Print the title and abstract box

%\tableofcontents % Print the contents section

\thispagestyle{empty} % Removes page numbering from the first page

%----------------------------------------------------------------------------------------
%	ARTICLE CONTENTS
%----------------------------------------------------------------------------------------

%\section*{Introduction} % The \section*{} command stops section numbering

%\addcontentsline{toc}{section}{\hspace*{-\tocsep}Introduction} % Adds this section to the table of contents with negative horizontal space equal to the indent for the numbered sections

%\tableofcontents
%\section{ Introduction}

%\section*{Overview}

从中国武汉开始爆发的新冠病毒肺炎疫情中,重症和危重症比例约为 20\%,病亡率约为 2\%。感染新冠病毒后的常见潜伏期为 3 天,但可能延长至 14 天,另外还有潜伏期长达 24 和 27 天的病例报告。新冠病毒具有极强的传染性,导致每天会新增 50\% 的确诊病例(基本传染数 R0 约为 3—4 )。如果新冠肺炎疫情发展成大流行或地方性流行病,全世界每个人的生活都会就此改变。我们必须采取行动限制和阻止疫情爆发,不再任由疫情蔓延。我们为个人、社区和政府提供以下行动指南。 

\begin{multicols}{2}
\section*{个人和社区指南}
\begin{itemize}
\item 对自己的健康负责,有意识和守规则地对邻里的健康负责
\item 增大社交距离
\item 避免触摸公共场所或共享空间内的表面
\item 避免聚集 
\item 避免与他人直接接触,定期洗手,并在与疑似感染者近距离接触时佩戴口罩
\item 咳嗽和打喷嚏时遮掩口鼻
\item 监测体温或者其他早期感染症状(干咳、打喷嚏、流涕、咽痛)
\item 如果您出现了早期症状,请进行自我隔离
\item 如果症状持续发展,请安排安全的交通方式,前往政府建议的医疗机构及时就医;请避免乘坐公共交通,同时请全程佩戴口罩
\item 在较高风险地区,用无接触的方式为社区成员提供必需品;可以把物资放在屋外
\item 与他人合作建立起安全区或安全社区。与家人和朋友讨论安全保障,交流安全准则,了解正在遵循安全准则的人,制定共同策略,跟进并与彼此分享需求、担忧和机会
\item 对谣言保持警惕,不散布错误的信息
\end{itemize}

\vspace{2ex}

\section*{社区和政府指南}
\begin{itemize}
\item 位于疫情高发社区或国家附近的区域,在社区进出口或入境处设立新冠病毒肺炎检疫站
\item 引导具有较高感染风险的个人在进入无感染区后进行 14 天的隔离
\item 在疫情风险较高地区,组织协调社区团队使用红外测温仪和穿戴个人防护设备,挨家挨户进行症状检测
\item 社区团队挨家挨户进行检查时,还应该留意哪些住户需要支持服务

\end{itemize}

\vspace{2ex}

\section*{政府指南}
\begin{itemize}
\item 提前准备好口罩、个人防护装备和测试试剂盒等战略物资,构建分发渠道
\item 确认有确诊或疑似病例的区域
\item 暂停有感染区域和未感染区域之间不必要的运输
\item 把疑似病例与确诊病例分开安排在指定机构隔离,隔离机构需要配备足够的医疗资源,包括个人防护装备
\item 有症状者应该遵循专门设计好的流程前往指定医疗机构接受检测,避免乘坐公共交通工具或出租车
\item 隔离和检测确诊病例密切接触者中的所有疑似病例
\item 提高公众意识:
  \begin{itemize}
  \item 典型的症状和可能的传播途径
  \item 强调新冠病毒的高传染率和感染后普遍症状较轻,鼓励个人及时就医
  \item 鼓励改善个人卫生习惯,包括经常洗手、在公共场所佩戴口罩,以及避免人与人之间的接触
  \end{itemize}
\item 停止公共聚会
\item 特别注意预防或监测进出高密度密闭设施的人群,例如监狱、医疗机构、康复中心和看护机构、疗养院、退休社区、宿舍和旅馆等
\item 提高有确诊病例地区的社区责任
\item 留意每个邻里/社区中从事频繁接触他人的工作的人群。每天监测他们的身体状况,帮助发现感染病例和阻止传播
\item 与偏远地区进行沟通,为他们分配资源
\item 配合国际社会和世界卫生组织的协调工作,共享有关病例诊断、患者旅行史、治疗方案、预防策略和医疗用品短缺等信息
\item 为那些表现出类似新冠肺炎症状的非新冠病毒肺炎患者安排治疗
\item 在疫情传播活跃地区:
  \begin{itemize}
  \item 关闭宗教场所、大学、中小学和企业
  \item 限制人们外出,以无接触的递送方式提供必需品支持
  \item 挨家挨户上门排查出现早期症状和需要支持服务的居民,排查人员必须穿戴好 个人防护装备,由社区统筹安排
  \end{itemize}
\end{itemize}

\end{multicols}

\vspace{2ex}
更多关于医疗和社会应对的信息,请见:
\begin{itemize}
\item 世界卫生组织:\url{https://www.who.int/emergencies/diseases/novel-coronavirus-2019/technical-guidance}
\item 新加坡 COVID-19:\url{https://www.moh.gov.sg/covid-19}
\end{itemize}






% \bibliography{MyCollection.bib}
\bibliography{references.bib}

\end{document}
