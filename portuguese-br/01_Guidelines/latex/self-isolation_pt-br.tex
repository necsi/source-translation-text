%\documentclass[twocolumn,journal]{IEEEtran}
\documentclass[onecolumn,journal]{IEEEtran}
\usepackage{amsfonts}
\usepackage{amsmath}
\usepackage{amsthm}
\usepackage{amssymb}
\usepackage{graphicx}
\usepackage[T1]{fontenc}
%\usepackage[english]{babel}
\usepackage{supertabular}
\usepackage{longtable}
\usepackage[usenames,dvipsnames]{color}
\usepackage{bbm}
%\usepackage{caption}
\usepackage{fancyhdr}
\usepackage{breqn}
\usepackage{fixltx2e}
\usepackage{capt-of}
%\usepackage{mdframed}
\setcounter{MaxMatrixCols}{10}
\usepackage{tikz}
\usetikzlibrary{matrix}
\usepackage{endnotes}
\usepackage{soul}
\usepackage{marginnote}
%\newtheorem{theorem}{Theorem}
\newtheorem{lemma}{Lemma}
%\newtheorem{remark}{Remark}
%\newtheorem{error}{\color{Red} Error}
\newtheorem{corollary}{Corollary}
\newtheorem{proposition}{Proposition}
\newtheorem{definition}{Definition}
\newcommand{\mathsym}[1]{}
\newcommand{\unicode}[1]{}
\newcommand{\dsum} {\displaystyle\sum}
\hyphenation{op-tical net-works semi-conduc-tor}
\usepackage{pdfpages}
\usepackage{enumitem}
\usepackage{multicol}
\usepackage[utf8]{inputenc}

\headsep = 5pt
\textheight = 730pt
%\headsep = 8pt %25pt
%\textheight = 720pt %674pt
%\usepackage{geometry}

\bibliographystyle{unsrt}

\usepackage{float}

 \usepackage{xcolor}

\usepackage[framemethod=TikZ]{mdframed}
%%%%%%%FRAME%%%%%%%%%%%
\usepackage[framemethod=TikZ]{mdframed}
\usepackage{framed}
    % \BeforeBeginEnvironment{mdframed}{\begin{minipage}{\linewidth}}
     %\AfterEndEnvironment{mdframed}{\end{minipage}\par}


%	%\mdfsetup{%
%	%skipabove=20pt,
%	nobreak=true,
%	   middlelinecolor=black,
%	   middlelinewidth=1pt,
%	   backgroundcolor=purple!10,
%	   roundcorner=1pt}

\mdfsetup{%
	outerlinewidth=1,skipabove=20pt,backgroundcolor=yellow!50, outerlinecolor=black,innertopmargin=0pt,splittopskip=\topskip,skipbelow=\baselineskip, skipabove=\baselineskip,ntheorem,roundcorner=5pt}

\mdtheorem[nobreak=true,outerlinewidth=1,%leftmargin=40,rightmargin=40,
backgroundcolor=yellow!50, outerlinecolor=black,innertopmargin=0pt,splittopskip=\topskip,skipbelow=\baselineskip, skipabove=\baselineskip,ntheorem,roundcorner=5pt,font=\itshape]{result}{Result}


\mdtheorem[nobreak=true,outerlinewidth=1,%leftmargin=40,rightmargin=40,
backgroundcolor=yellow!50, outerlinecolor=black,innertopmargin=0pt,splittopskip=\topskip,skipbelow=\baselineskip, skipabove=\baselineskip,ntheorem,roundcorner=5pt,font=\itshape]{theorem}{Theorem}

\mdtheorem[nobreak=true,outerlinewidth=1,%leftmargin=40,rightmargin=40,
backgroundcolor=gray!10, outerlinecolor=black,innertopmargin=0pt,splittopskip=\topskip,skipbelow=\baselineskip, skipabove=\baselineskip,ntheorem,roundcorner=5pt,font=\itshape]{remark}{Remark}

\mdtheorem[nobreak=true,outerlinewidth=1,%leftmargin=40,rightmargin=40,
backgroundcolor=pink!30, outerlinecolor=black,innertopmargin=0pt,splittopskip=\topskip,skipbelow=\baselineskip, skipabove=\baselineskip,ntheorem,roundcorner=5pt,font=\itshape]{quaestio}{Quaestio}

\mdtheorem[nobreak=true,outerlinewidth=1,%leftmargin=40,rightmargin=40,
backgroundcolor=yellow!50, outerlinecolor=black,innertopmargin=5pt,splittopskip=\topskip,skipbelow=\baselineskip, skipabove=\baselineskip,ntheorem,roundcorner=5pt,font=\itshape]{background}{Background}

%TRYING TO INCLUDE Ppls IN TOC
\usepackage{hyperref}


\begin{document}
\title{\color{Brown}  Diretrizes para Auto-Isolamento Físico
\vspace{-0.35ex}}
\author{Chen Shen e Yaneer Bar-Yam \\ New England Complex Systems Institute \\
\vspace{+0.35ex}
\small{\textit{(traduzido por Lucas Pontes})}\\
 \today
  \vspace{-8ex} \\
%\bigskip
\textbf{}
 }

\maketitle

%\vspace{-1ex}
%\flushbottom % Makes all text pages the same height

%\maketitle % Print the title and abstract box

%\tableofcontents % Print the contents section

\thispagestyle{empty} % Removes page numbering from the first page

%----------------------------------------------------------------------------------------
%	ARTICLE CONTENTS
%----------------------------------------------------------------------------------------

%\section*{Introduction} % The \section*{} command stops section numbering

%\addcontentsline{toc}{section}{\hspace*{-\tocsep}Introduction} % Adds this section to the table of contents with negative horizontal space equal to the indent for the numbered sections

%\tableofcontents
%\section{ Introduction}

%\section*{Overview}


\begin{multicols}{2}

% \section

Essa diretriz é direcionada a indivíduos que testaram positivo para COVID-19, mas que apresentam sintomas leves ou inexistentes. Os sintomas leves incluem febre baixa, fadiga leve, tosse, mas sem sintomas de pneumonia (como falta de ar), e sem doenças crônicas associadas.
 
Esse documento também pode servir de orientação para indivíduos com sintomas moderados ou até em pior estado, que estão presentes em locais onde os recursos médicos estejam sobrecarregados. Contudo, é importante lembrar-se de que os conselhos fornecidos aqui são direcionados para indivíduos com sintomas leves/sem sintomas.
 
Caso os sintomas se desenvolvam, incluindo dificuldades respiratórias e/ou febre alta, procure atendimento médico imediatamente.


\section*{Aspectos Gerais}
\begin{itemize}
\item  Esteja ciente que a maioria dos casos é resolvida naturalmente, com retorno total à saúde. A quarentena é uma necessidade temporária, normalmente com duração de 14 dias.

\item Se adoecer, é altamente recomendável que fique sozinho, pelo menos em um cômodo à parte.

\end{itemize}

\section*{No caso de estar morando sozinho}
\begin{itemize}
\item Acompanhe com atenção o estado de sua saúde. Mantenha um registro escrito de informações acerca de sua saúde com letra legível e em um local óbvio. O registro deve incluir: data e hora do registro, frequência cardíaca, sangue, saturação de oxigênio no sangue (caso possua um oxímetro), sintomas, refeições, remédios tomados e dosagem.

\item  Tenha a mão as informações locais sobre o que fazer caso seus sintomas progridam. Grave contatos importantes na discagem rápida. Combine contatos telefônicos de checagem de estado de saúde (pelo menos diariamente) com membros da família ou amigos. Dê a eles todas as informações de contato de emergência, bem como todas as informações necessárias para entrar em sua casa, caso você esteja incapacitado.

\item Cuide da sua saúde mantendo-se constantemente hidratado, pratique uma dieta equilibrada e durma horas regulares. Mantenha ou desenvolva atividades divertidas ou educacionais, incluindo leitura, jogos solitários ou online, ou outras interações pela internet.

\item Lave as mãos com freqüência, com sabão/desinfetante, por pelo menos, 20 segundos de cada vez.

\item É crucial manter regular a ventilação da sala de estar.

\item Lave lençóis, toalhas e roupas com freqüência. Separe o que usar do que está sendo usado por outros.

\item Evite o contato com seu animal de estimação e outros animais. Se não for possível, use máscara [1] e lave as mãos antes e depois de interagir com os animais de estimação.

\item Isole-se de qualquer contato físico com outras pessoas, mas mantenha contato com familiares e amigos por texto, telefone, vídeo bate-papo ou por outros meios eletrônicos. Isso é importante por várias razões, incluindo a manutenção de uma perspectiva positiva.

\item Se as notícias sobre o desenrolar do surto lhe causam ansiedade, tente não se concentrar nele, para evitar mais problemas de saúde mental.

\item Execute uma rotina diária relativamente consistente. Se possível, mantenha uma quantidade moderada ou até pequena de exercícios físicos.

\item Organize com amigos, família ou autoridades locais a logística da entrega de itens essenciais, incluindo refeições diárias. Acompanhe todos os itens essenciais e esteja ciente da data de validade desses produtos, de modo a se antecipar e a informar seus fornecedores com antecedência. É recomendável que a entrega dos itens seja feita sem contato físico. Use máscara [1] e luvas caso seja necessário interagir com o entregador.

\item Consulte as autoridades locais e os profissionais de saúde sobre a duração e as condições para encerrar o auto-isolamento. Limpe, lave sistematicamente as superfícies, tecidos e objetos com que teve contato. Tenha cuidado mesmo após o auto-isolamento.

\item Em áreas suburbanas ou rurais, onde é possível sair e entrar em casa sem ter contato com outras pessoas, é possível fazer caminhadas isoladas de outras pessoas em espaços abertos. Não se esqueça de que você está em quarentena e não pode interagir com outras pessoas, nem mesmo entrar em espaços onde é provável que outras pessoas estejam.
\end{itemize}

\section*{No caso de dividir sua moradia com outros}
\begin{itemize}
\item As pessoas que compartilham a casa com você são consideradas "contatos próximos" e devem seguir as diretrizes locais em relação a contatos próximos, incluindo evitar contato desnecessário com outras pessoas.
\item  Os outros residentes da casa devem evitar receber visitantes, especialmente pessoas vulneráveis (idosos ou pessoas com doença crônica). Qualquer visitante deve ser informado sobre a presença de um indivíduo em quarentena na casa.
\item Defina claramente (se possível com auxílio visual), as diferentes Zonas da residência. O quarto e as instalações utilizadas pelo paciente são considerados uma Zona Vermelha. Áreas conectadas à Zona Vermelha, por exemplo, a sala de estar, são consideradas Zonas Amarelas. Outros quartos mais a parte são Zonas Verdes.
\item O paciente deve seguir rigorosamente a forma correta de espirrar e tossir, espirrando em tecidos descartáveis que sejam descartados com segurança, ou em roupas (por exemplo, manga de camisa) que sejam lavadas em breve.
\item Instale um mecanismo de comunicação na casa para que o paciente possa informar os outros residentes quando precisar sair do quarto.
\item O paciente deve desinfetar as Zonas Vermelhas regularmente. Os demais moradores devem desinfetar as Zonas Amarelas e, de preferência, as Zonas Verdes também, regularmente.
\item O paciente deve se limitar à Zona Vermelha, minimizando a entrada nas Zonas Amarela, e evitando completamente a Zona Verde. Objetos em contato próximo com o paciente também devem seguir esta regra. Quando fora da Zona Vermelha, o paciente deve usar luvas e máscara [1].
\item As possíveis rotas de transmissão da doença para os outros moradores, mesmo que o doente ou a pessoa em suspeita esteja isolado, são:
\begin{itemize}
\item instalações compartilhadas: cozinha, banheiro, etc.
\item material doméstico compartilhado: toalha, copos, utensílios, etc.
\item alimentos, bebidas, etc.
\item áreas tocadas: maçaneta da porta, superfície da mesa, controle remoto, interruptores de luz, etc (estes devem ser desinfetados pelo menos uma vez ao dia).
\end{itemize}

\item O paciente deve higienizar qualquer instalação que compartilhe com os demais após seu uso, especialmente o banheiro. Mantenha a tampa do vaso fechada quando não estiver sendo usada.

\item O paciente deve ter um saco de lixo / lixeira separado para descartar luvas, máscaras, tecidos, etc.

\item Se possível, o morador saudável deve ajudar no despache de resíduos descartados pelo enfermo ou suspeito de portar a enfermidade, para minimizar a necessidade de este sair da Zona Vermelha.
\end{itemize}

\section*{Observações finais}

[1] Apesar do uso de máscaras ser debatido, observamos que: (1) A Qualquer pessoa que tenha sintomas leves deve evitar o contato com outras pessoas e usar máscara enquanto estiver em contato público ou privado com outras pessoas. (2) O uso da máscara deve ser aceito em locais públicos para impedir que os doentes hesitem ou se sintam estigmatizados usando uma máscara. (3) Embora as máscaras não garantam segurança para um indivíduo saudável, e sua disponibilidade possa ser limitada devido à maior prioridade nas condições médicas, o uso de máscaras nos casos em que não se pode evitar o contato físico com outras pessoas reduz drasticamente o risco de infecção. (4) Para os indivíduos com alta suscetibilidade à doença (acima de 50 anos ou com condições de saúde preexistentes), assim como para aqueles que estão em áreas de risco elevado, o alto custo de ser infectado justifca o uso da máscara.


\end{multicols}

\vspace{2ex}







% \bibliography{MyCollection.bib}
\bibliography{references.bib}

\end{document}

