%\documentclass[twocolumn,journal]{IEEEtran}
\documentclass[onecolumn,journal]{IEEEtran}
\usepackage{amsfonts}
\usepackage{amsmath}
\usepackage{amsthm}
\usepackage{amssymb}
\usepackage{graphicx}
\usepackage[T1]{fontenc}
%\usepackage[english]{babel}
\usepackage{supertabular}
\usepackage{longtable}
\usepackage[usenames,dvipsnames]{color}
\usepackage{bbm}
%\usepackage{caption}
\usepackage{fancyhdr}
\usepackage{breqn}
\usepackage{fixltx2e}
\usepackage{capt-of}
%\usepackage{mdframed}
\setcounter{MaxMatrixCols}{10}
\usepackage{tikz}
\usetikzlibrary{matrix}
\usepackage{endnotes}
\usepackage{soul}
\usepackage{marginnote}
%\newtheorem{theorem}{Theorem}
\newtheorem{lemma}{Lemma}
%\newtheorem{remark}{Remark}
%\newtheorem{error}{\color{Red} Error}
\newtheorem{corollary}{Corollary}
\newtheorem{proposition}{Proposition}
\newtheorem{definition}{Definition}
\newcommand{\mathsym}[1]{}
\newcommand{\unicode}[1]{}
\newcommand{\dsum} {\displaystyle\sum}
\hyphenation{op-tical net-works semi-conduc-tor}
\usepackage{pdfpages}
\usepackage{enumitem}
\usepackage{multicol}
\usepackage[utf8]{inputenc}


\headsep = 5pt
\textheight = 730pt
%\headsep = 8pt %25pt
%\textheight = 720pt %674pt
%\usepackage{geometry}

\bibliographystyle{unsrt}

\usepackage{float}

 \usepackage{xcolor}

\usepackage[framemethod=TikZ]{mdframed}
%%%%%%%FRAME%%%%%%%%%%%
\usepackage[framemethod=TikZ]{mdframed}
\usepackage{framed}
    % \BeforeBeginEnvironment{mdframed}{\begin{minipage}{\linewidth}}
     %\AfterEndEnvironment{mdframed}{\end{minipage}\par}


%	%\mdfsetup{%
%	%skipabove=20pt,
%	nobreak=true,
%	   middlelinecolor=black,
%	   middlelinewidth=1pt,
%	   backgroundcolor=purple!10,
%	   roundcorner=1pt}

\mdfsetup{%
	outerlinewidth=1,skipabove=20pt,backgroundcolor=yellow!50, outerlinecolor=black,innertopmargin=0pt,splittopskip=\topskip,skipbelow=\baselineskip, skipabove=\baselineskip,ntheorem,roundcorner=5pt}

\mdtheorem[nobreak=true,outerlinewidth=1,%leftmargin=40,rightmargin=40,
backgroundcolor=yellow!50, outerlinecolor=black,innertopmargin=0pt,splittopskip=\topskip,skipbelow=\baselineskip, skipabove=\baselineskip,ntheorem,roundcorner=5pt,font=\itshape]{result}{Result}


\mdtheorem[nobreak=true,outerlinewidth=1,%leftmargin=40,rightmargin=40,
backgroundcolor=yellow!50, outerlinecolor=black,innertopmargin=0pt,splittopskip=\topskip,skipbelow=\baselineskip, skipabove=\baselineskip,ntheorem,roundcorner=5pt,font=\itshape]{theorem}{Theorem}

\mdtheorem[nobreak=true,outerlinewidth=1,%leftmargin=40,rightmargin=40,
backgroundcolor=gray!10, outerlinecolor=black,innertopmargin=0pt,splittopskip=\topskip,skipbelow=\baselineskip, skipabove=\baselineskip,ntheorem,roundcorner=5pt,font=\itshape]{remark}{Remark}

\mdtheorem[nobreak=true,outerlinewidth=1,%leftmargin=40,rightmargin=40,
backgroundcolor=pink!30, outerlinecolor=black,innertopmargin=0pt,splittopskip=\topskip,skipbelow=\baselineskip, skipabove=\baselineskip,ntheorem,roundcorner=5pt,font=\itshape]{quaestio}{Quaestio}

\mdtheorem[nobreak=true,outerlinewidth=1,%leftmargin=40,rightmargin=40,
backgroundcolor=yellow!50, outerlinecolor=black,innertopmargin=5pt,splittopskip=\topskip,skipbelow=\baselineskip, skipabove=\baselineskip,ntheorem,roundcorner=5pt,font=\itshape]{background}{Background}

%TRYING TO INCLUDE Ppls IN TOC
\usepackage{hyperref}


\begin{document}
\title{\color{Brown}  Para Indivíduos, Comunidades e Governos \\
Diretrizes de resposta antecipada ao surto - Versão 3 \\
\vspace{-0.35ex}}
\author{Chen Shen e Yaneer Bar-Yam \\ do New England Complex Systems Institute \\
\vspace{+0.35ex}
\small{\textit{(traduzido por Lucas Pontes})}\\
 \today
  \vspace{-8ex} \\
%\bigskip
\textbf{}
 }

\maketitle

%\vspace{-1ex}
%\flushbottom % Makes all text pages the same height

%\maketitle % Print the title and abstract box

%\tableofcontents % Print the contents section

\thispagestyle{empty} % Removes page numbering from the first page

%----------------------------------------------------------------------------------------
%	ARTICLE CONTENTS
%----------------------------------------------------------------------------------------

%\section*{Introduction} % The \section*{} command stops section numbering

%\addcontentsline{toc}{section}{\hspace*{-\tocsep}Introduction} % Adds this section to the table of contents with negative horizontal space equal to the indent for the numbered sections

%\tableofcontents
%\section{ Introduction}

%\section*{Overview}

O surto de coronavírus que se originou em Wuhan tem cerca de 20\% de casos graves e 2\% de mortes. Um período de incubação típico é de 3 dias, mas pode se estender para 14 dias, e há relatos de 24 e 27 dias. O coronavírus é altamente contagioso com um aumento diário de 50\% em novos casos (taxa de infecção R0 de cerca de 3-4), a menos que intervenções extraordinárias sejam feitas. Caso o Corona se torne uma pandemia ou endemia generalizada, isso mudará a vida de todos no mundo. É imperativo agir para limitar e parar o surto e não aceitar a sua propagação. Fornecemos essas diretrizes para ações individuais, comunitárias e governamentais.

\begin{multicols}{2}
\section*{Diretrizes para indivíduos e comunidades}
\begin{itemize}
\item Assuma a responsabilidade por sua própria saúde e tome responsabilidade pela saúde de seus vizinhos, com consciência e disciplina
\item Pratique distanciamento físico
\item Evite tocar superfícies em espaços públicos ou compartilhados
\item Evite grupos e aglomerações de pessoas
\item Evite contato direto com outras pessoas, lave as mãos regularmente e use máscaras [1] quando se aproximar de outras pessoas que podem estar infectadas
\item Cubra a tosse/espirro
\item Monitore a temperatura ou outros sintomas iniciais de infecção (tosse, espirro, coriza, dor de garganta)
\item Coloque-se em auto-isolamento se você apresentar os sintomas iniciais
\item Se os sintomas persistirem, providencie transporte seguro para instalações médicas, seguindo as recomendações do governo: evite o transporte público, e use máscaras [1]
\item Nas áreas de maior risco, entregue os suprimentos aos membros da comunidade sem contato pessoal; os suprimentos podem ser deixados do lado de fora de casa para que sejam pegos após seu afastamento
\item Colabore com outras pessoas para criar zonas / comunidades seguras. Converse com seus familiares e amigos sobre como se manterem seguros, fale sobre diretrizes de segurança, saiba quem está seguindo diretrizes seguras, defina políticas compartilhadas, acompanhe e compartilhe as necessidades, preocupações e oportunidades uns com os outros.
\item Analise criticamente os rumores e não espalhe informações falsas

\end{itemize}

\vspace{2ex}

\section*{Diretrizes para comunidades e governos locais}
\begin{itemize}

\item Próximo a comunidades ou países com infecções ativas, faça verificações nas fronteiras quanto a sintomas
\item Realize quarentenas de 14 dias para indivíduos em risco que entram em áreas livres de infecção
\item Em áreas de risco elevado, coordene equipes da vizinhança para monitoramento de porta em porta da comunidade, em busca de sintomas usando termômetros de infravermelho e equipamentos de proteção individual (EPIs)
\item As equipes da vizinhança que passarem de porta em porta também devem identificar os indivíduos que precisam de serviços de apoio.

\end{itemize}

\vspace{2ex}

\section*{Diretrizes para governos}
\begin{itemize}
\item Prepare antecipadamente recursos estratégicos como máscaras [1], EPIs, e kits de teste, e estabeleça rotas de distribuição para esses recursos
\item Identifique as áreas em que há casos confirmados ou suspeitos de infecção
\item Interrompa o transporte não essencial entre as áreas infectadas e não infectadas.
\item Isole os indivíduos com infecções suspeitas e confirmados separadamente, para que sejam atendidos em instalações designadas com recursos médicos adequados, incluindo equipamentos de proteção pessoal (EPIs)
\item Pessoas com sintomas deve fazer uso de um processo especialmente concebido para serem levados para as instalações de saúde designadas para testes, evitando o transporte público ou privado (táxi, Uber)
\item Coloque em quarentena e teste todos os casos suspeitos nos arredores de um caso identificado
\item Promova a conscientização pública acerca dos sintomas típicos e possíveis meios de transmissão:
  \begin{itemize}
  \item Enfatize a alta taxa de contágio da doença e seus sintomas iniciais, para incentivar as pessoas a procurar atendimento médico
  \item Incentive uma melhor higiene individual, incluindo lavar as mãos com frequência, uso de máscaras [1] em áreas públicas e evitar o contato entre indivíduos
  \end{itemize}
\item Interrompa reuniões públicas
\item Preste atenção especial para prevenir a entrada ou saída, ou monitorar a saúde, das pessoas em instalações confinadas de alta densidade, como prisões, instalações médicas, clínicas de reabilitação e de vida assistida, asilos, comunidades de aposentados, dormitórios e albergues.
\item Promova as responsabilidades da comunidade em áreas infectadas;
\item Em cada bairro/comunidade, identifique o grupo de pessoas cujo trabalho diário envolva contato humano frequente. Monitore a condição desses diariamente para ajudar a detectar infecções e prevenir o contágio.
\item Envolva-se na comunicação e distribuição de recursos para áreas remotas.
\item Mantenha contato coordenado com comunidades internacionais e a OMS para compartilhar informações sobre identificação de casos, histórico de viagens de pacientes, tratamentos, estratégias de prevenção e escassez de suprimentos médicos
\item Planeje-se para o tratamento de pacientes com sintomas semelhantes que não estão infectados pelo COVID-19
\item Em áreas com transmissão ativa:
  \begin{itemize}
  \item Feche os locais de culto, igrejas, universidades, escolas e empresas
  \item Restrinja as pessoas às casas e forneça suporte para que suprimentos e produtos necessários sejam entregues sem contato físico
  \item Realize pesquisas de porta em porta para indivíduos com sintomas iniciais e necessidades de serviços básicos, com as precauções necessárias de EPIs e com o esforço e envolvimento da comunidade
  \end{itemize}
\end{itemize}

\end{multicols}

\vspace{2ex}
Para mais informações acerca de respostas sociais e dos sistemas de saúde locais ao Coronavírus, veja:
\begin{itemize}
\item Organização Mundial de Saúde: \url{https://www.who.int/emergencies/diseases/novel-coronavirus-2019/technical-guidance}
\item Diretrizes do governo de Cingapura contra o Coronavírus (em inglês): \url{https://www.moh.gov.sg/covid-19}
\end{itemize}






% \bibliography{MyCollection.bib}
\bibliography{references.bib}

\end{document}

