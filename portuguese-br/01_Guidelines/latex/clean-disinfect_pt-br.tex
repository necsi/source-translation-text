%\documentclass[twocolumn,journal]{IEEEtran}
\documentclass[onecolumn,journal]{IEEEtran}
\usepackage{amsfonts}
\usepackage{amsmath}
\usepackage{amsthm}
\usepackage{amssymb}
\usepackage{graphicx}
\usepackage[T1]{fontenc}
%\usepackage[english]{babel}
\usepackage{supertabular}
\usepackage{longtable}
\usepackage[usenames,dvipsnames]{color}
\usepackage{bbm}
%\usepackage{caption}
\usepackage{fancyhdr}
\usepackage{breqn}
\usepackage{fixltx2e}
\usepackage{capt-of}
%\usepackage{mdframed}
\setcounter{MaxMatrixCols}{10}
\usepackage{tikz}
\usetikzlibrary{matrix}
\usepackage{endnotes}
\usepackage{soul}
\usepackage{marginnote}
%\newtheorem{theorem}{Theorem}
\newtheorem{lemma}{Lemma}
%\newtheorem{remark}{Remark}
%\newtheorem{error}{\color{Red} Error}
\newtheorem{corollary}{Corollary}
\newtheorem{proposition}{Proposition}
\newtheorem{definition}{Definition}
\newcommand{\mathsym}[1]{}
\newcommand{\unicode}[1]{}
\newcommand{\dsum} {\displaystyle\sum}
\hyphenation{op-tical net-works semi-conduc-tor}
\usepackage{pdfpages}
\usepackage{enumitem}
\usepackage{multicol}
\usepackage[utf8]{inputenc}

\headsep = 5pt
\textheight = 730pt
%\headsep = 8pt %25pt
%\textheight = 720pt %674pt
%\usepackage{geometry}

\bibliographystyle{unsrt}

\usepackage{float}

% \usepackage{xcolor}

\usepackage[framemethod=TikZ]{mdframed}
%%%%%%%FRAME%%%%%%%%%%%
\usepackage[framemethod=TikZ]{mdframed}
\usepackage{framed}
    % \BeforeBeginEnvironment{mdframed}{\begin{minipage}{\linewidth}}
     %\AfterEndEnvironment{mdframed}{\end{minipage}\par}


%	%\mdfsetup{%
%	%skipabove=20pt,
%	nobreak=true,
%	   middlelinecolor=black,
%	   middlelinewidth=1pt,
%	   backgroundcolor=purple!10,
%	   roundcorner=1pt}

\mdfsetup{%
	outerlinewidth=1,skipabove=20pt,backgroundcolor=yellow!50, outerlinecolor=black,innertopmargin=0pt,splittopskip=\topskip,skipbelow=\baselineskip, skipabove=\baselineskip,ntheorem,roundcorner=5pt}

\mdtheorem[nobreak=true,outerlinewidth=1,%leftmargin=40,rightmargin=40,
backgroundcolor=yellow!50, outerlinecolor=black,innertopmargin=0pt,splittopskip=\topskip,skipbelow=\baselineskip, skipabove=\baselineskip,ntheorem,roundcorner=5pt,font=\itshape]{result}{Result}


\mdtheorem[nobreak=true,outerlinewidth=1,%leftmargin=40,rightmargin=40,
backgroundcolor=yellow!50, outerlinecolor=black,innertopmargin=0pt,splittopskip=\topskip,skipbelow=\baselineskip, skipabove=\baselineskip,ntheorem,roundcorner=5pt,font=\itshape]{theorem}{Theorem}

\mdtheorem[nobreak=true,outerlinewidth=1,%leftmargin=40,rightmargin=40,
backgroundcolor=gray!10, outerlinecolor=black,innertopmargin=0pt,splittopskip=\topskip,skipbelow=\baselineskip, skipabove=\baselineskip,ntheorem,roundcorner=5pt,font=\itshape]{remark}{Remark}

\mdtheorem[nobreak=true,outerlinewidth=1,%leftmargin=40,rightmargin=40,
backgroundcolor=pink!30, outerlinecolor=black,innertopmargin=0pt,splittopskip=\topskip,skipbelow=\baselineskip, skipabove=\baselineskip,ntheorem,roundcorner=5pt,font=\itshape]{quaestio}{Quaestio}

\mdtheorem[nobreak=true,outerlinewidth=1,%leftmargin=40,rightmargin=40,
backgroundcolor=yellow!50, outerlinecolor=black,innertopmargin=5pt,splittopskip=\topskip,skipbelow=\baselineskip, skipabove=\baselineskip,ntheorem,roundcorner=5pt,font=\itshape]{background}{Background}

%TRYING TO INCLUDE Ppls IN TOC
\usepackage{hyperref}


\begin{document}
\title{\color{Brown}  Diretrizes de limpeza e desinfecção para prevenir a transmissão do COVID-19
\vspace{-0.35ex}}
\author{Aaron Green, Chen Shen e Yaneer Bar-Yam \\ New England Complex Systems Institute \\
\vspace{+0.35ex}
\small{\textit{(traduzido por Lucas Pontes})}\\
 \today
  \vspace{-8ex} \\
%\bigskip
\textbf{}
 }

\maketitle

%\vspace{-1ex}
%\flushbottom % Makes all text pages the same height

%\maketitle % Print the title and abstract box

%\tableofcontents % Print the contents section

\thispagestyle{empty} % Removes page numbering from the first page

%----------------------------------------------------------------------------------------
%	ARTICLE CONTENTS
%----------------------------------------------------------------------------------------

%\section*{Introduction} % The \section*{} command stops section numbering

%\addcontentsline{toc}{section}{\hspace*{-\tocsep}Introduction} % Adds this section to the table of contents with negative horizontal space equal to the indent for the numbered sections

%\tableofcontents
%\section{ Introduction}

%\section*{Overview}


\begin{multicols}{2}

O COVID-19 é transmitido principalmente através de aerossóis e gotículas, e esse vírus pode manter, por dias, sua capacidade infectiva quando estão em fômites (fômites são qualquer objeto ou substância capaz de reter organismos infecciosos, como superfícies, equipamentos, utensílios, tecidos, cabelos, poeira e outras partículas). A doença pode se espalhar quando as pessoas tocam superfícies contaminadas e depois tocam o rosto. A limpeza e desinfecção regulares de superfícies que são locais freqüentes de contato físico ajudam a evitar a propagação da doença.

A limpeza remove germes, poeira, sujeira e impurezas das superfícies. Algumas formas de limpeza também matam germes. Mesmo quando a limpeza não mata germes, removê-los do ambiente em que as pessoas estão localizadas reduz o risco de propagação da infecção.

A desinfecção para neutralizar microrganismos deve ser realizada após a limpeza para reduzir ainda mais o risco de transmissão da infecção. O coronavírus que causa o COVID-19 pode ser neutralizado por sabão, álcool e água sanitária.

Mais especificamente, o SARS-CoV-2 pode ser neutralizado por solventes lipídicos, incluindo éter (75\%), etanol, desinfetantes contendo cloro, ácido peroxiacético e clorofórmio, exceto a clorexidina. Uma lista de desinfetantes para SARS-CoV-2 pode ser encontrada no site da EPA, em inglês (https://www.epa.gov/pesticideregistration/list-n-disinfectants-use-against-sars-cov-2).

O tempo também mata partículas virais. As informações atuais indicam que, durante um período de horas, as partículas virais de baixa densidade do papelão se tornam inativas e, durante um período de alguns dias, elas se tornam inativas em superfícies de plástico e metal duras. No entanto, a inativação temporal em uma variedade de fômites, e condições, ainda não é bem compreendida. Isso parece depender da quantidade de depósitos virais, da temperatura, da umidade e de outras condições ambientais. Para áreas de alta densidade ou grandes, a confiabilidade da inativação diminui e a desinfecção química é altamente recomendada. A lavagem e o uso de desinfetantes também devem dar tempo para que os efeitos ocorram. É importante aplicar desinfetante e deixá-lo por um tempo nas superfícies antes de enxaguar.

Em geral, existem dois tipos de superfícies que precisam ser limpas e requerem protocolos diferentes. Materiais macios e porosos incluem carpetes, tapetes, toalhas, roupas, sofás, cadeiras, roupas de cama, brinquedos de tecido macio (animais de pelúcia), etc. As superfícies duras não porosas incluem aço inoxidável, pisos, superfícies de cozinha, bancadas, mesas e cadeiras, pias, vasos sanitários, grades, placas de interruptores, maçanetas, brinquedos de metal/plástico, teclados de computador, controles remotos, e equipamentos de recreação.

\section*{Suprimentos de limpeza necessários}

\begin{itemize}
    \item Luvas impermeáveis, como látex, nitrilo ou para lavar louças
    \item Sabão/detergente, água morna, toalhas limpas, sacos de lixo plásticos à prova de vazamentos
    \item Batas descartáveis para tarefas relacionadas à limpeza, incluindo a remoção de lixo industrial
    \item Máscara de rosto
    \item Óculos de proteção (opcional para evitar reações adversas à limpeza e desinfecção de solventes)
    \item Desinfetantes
\end{itemize}

\section*{Guia geral de limpeza}

\begin{itemize}
    \item Se possível, ao invés de limpar ou desinfetar itens altamente contaminados, descarte-os.
    \item Jogue imediatamente fora todos os itens de limpeza descartáveis após usá-los.
    \item Lave as mãos com freqüência, inclusive após esvaziar cestos de lixo e tocar tecidos e resíduos semelhantes.
    \item Lave bem as mãos com água e sabão por pelo menos 20 segundos, ou use um desinfetante para as mãos à base de álcool que contenha pelo menos 60\% de álcool
\end{itemize}

\section*{Roupas e outros materiais porosos que podem ser lavados}

\begin{itemize}
    \item Coloque os materiais em um saco plástico selado até a lavagem
    \item Se possível, lave com água quente e detergente, de preferência contendo alvejante adequado para a cor das roupas
    \item Se possível, seque-as na máxima temperatura da secadora, ou ao sol.
\end{itemize}

\section*{Materiais porosos que não podem ser lavados (sofás, tapetes, etc.)}

\begin{itemize}
    \item Se possível, aspire a poeira dos materiais com aspirador para impedir que a poeira se espalhe no ar
    \item Limpe manchas resultantes de fluido corporal ou por comida logo que ocorrerem, seguindo procedimentos seguros
    \item Limpe tapetes integralmente, evitando espirrar o máximo possível
    \item Use limpadores à vapor para limpar tapetes e outras superfícies porosas, se possível
\end{itemize}

\section*{Superfícies duras e não porosas}

\begin{itemize}
    \item Siga as instruções rotuladas em todos os recipientes.
    \item Limpe superfícies com água e sabão, removendo todos os detritos e manchas visíveis.
    \item Lave a superfície com água limpa e limpe com uma toalha limpa.
    \item Aplique o desinfetante. Para efetivamente matar o vírus, garanta que a superfície fica molhada com o desinfetante por pelo menos 10 minutos antes de limpar com uma toalha limpa. Se não for possível que utilize um desinfetante registrado na EPA, uma solução de alvejante a cloro a 2\% pode ser usada. Tome cuidado com desinfetantes à base de álcool, pois eles tendem a evaporar rapidamente e podem não desinfetar completamente a superfície se as instruções não forem seguidas.
    \item Enxágue com água e deixe a superfície secar ao ar. A lavagem após o uso de um desinfetante é especialmente importante em áreas utilizadas para preparação de alimentos.
    \item As cabeças de rodos, vassouras e esfregões devem ser limpas com sabão e água quente, higienizadas com um desinfetante ou solução de água sanitária registrada na EPA, e deixadas para secar. Considere o uso de cabeças descartáveis de esfregão, ou panos descartáveis de uso único, como alternativa.
    \item Remova as luvas, coloque-as em uma lixeira, e descarte-as.
    \item Lave as mãos após remover as luvas, e após manusear qualquer material contaminado, resíduos ou lixo
\end{itemize}

\end{multicols}

% \bibliography{MyCollection.bib}
\bibliography{references.bib}

\end{document}
