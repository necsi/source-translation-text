%\documentclass[twocolumn,journal]{IEEEtran}
\documentclass[onecolumn,journal]{IEEEtran}
\usepackage{amsfonts}
\usepackage{amsmath}
\usepackage{amsthm}
\usepackage{amssymb}
\usepackage{graphicx}
\usepackage[T1]{fontenc}
%\usepackage[english]{babel}
\usepackage{supertabular}
\usepackage{longtable}
\usepackage[usenames,dvipsnames]{color}
\usepackage{bbm}
%\usepackage{caption}
\usepackage{fancyhdr}
\usepackage{breqn}
\usepackage{fixltx2e}
\usepackage{capt-of}
%\usepackage{mdframed}
\setcounter{MaxMatrixCols}{10}
\usepackage{tikz}
\usetikzlibrary{matrix}
\usepackage{endnotes}
\usepackage{soul}
\usepackage{marginnote}
%\newtheorem{theorem}{Theorem}
\newtheorem{lemma}{Lemma}
%\newtheorem{remark}{Remark}
%\newtheorem{error}{\color{Red} Error}
\newtheorem{corollary}{Corollary}
\newtheorem{proposition}{Proposition}
\newtheorem{definition}{Definition}
\newcommand{\mathsym}[1]{}
\newcommand{\unicode}[1]{}
\newcommand{\dsum} {\displaystyle\sum}
\hyphenation{op-tical net-works semi-conduc-tor}
\usepackage{pdfpages}
\usepackage{enumitem}
\usepackage{multicol}
\usepackage[utf8]{inputenc}


\headsep = 5pt
\textheight = 730pt
%\headsep = 8pt %25pt
%\textheight = 720pt %674pt
%\usepackage{geometry}

\bibliographystyle{unsrt}

\usepackage{float}

 \usepackage{xcolor}

\usepackage[framemethod=TikZ]{mdframed}
%%%%%%%FRAME%%%%%%%%%%%
\usepackage[framemethod=TikZ]{mdframed}
\usepackage{framed}
    % \BeforeBeginEnvironment{mdframed}{\begin{minipage}{\linewidth}}
     %\AfterEndEnvironment{mdframed}{\end{minipage}\par}


%	%\mdfsetup{%
%	%skipabove=20pt,
%	nobreak=true,
%	   middlelinecolor=black,
%	   middlelinewidth=1pt,
%	   backgroundcolor=purple!10,
%	   roundcorner=1pt}

\mdfsetup{%
	outerlinewidth=1,skipabove=20pt,backgroundcolor=yellow!50, outerlinecolor=black,innertopmargin=0pt,splittopskip=\topskip,skipbelow=\baselineskip, skipabove=\baselineskip,ntheorem,roundcorner=5pt}

\mdtheorem[nobreak=true,outerlinewidth=1,%leftmargin=40,rightmargin=40,
backgroundcolor=yellow!50, outerlinecolor=black,innertopmargin=0pt,splittopskip=\topskip,skipbelow=\baselineskip, skipabove=\baselineskip,ntheorem,roundcorner=5pt,font=\itshape]{result}{Result}


\mdtheorem[nobreak=true,outerlinewidth=1,%leftmargin=40,rightmargin=40,
backgroundcolor=yellow!50, outerlinecolor=black,innertopmargin=0pt,splittopskip=\topskip,skipbelow=\baselineskip, skipabove=\baselineskip,ntheorem,roundcorner=5pt,font=\itshape]{theorem}{Theorem}

\mdtheorem[nobreak=true,outerlinewidth=1,%leftmargin=40,rightmargin=40,
backgroundcolor=gray!10, outerlinecolor=black,innertopmargin=0pt,splittopskip=\topskip,skipbelow=\baselineskip, skipabove=\baselineskip,ntheorem,roundcorner=5pt,font=\itshape]{remark}{Remark}

\mdtheorem[nobreak=true,outerlinewidth=1,%leftmargin=40,rightmargin=40,
backgroundcolor=pink!30, outerlinecolor=black,innertopmargin=0pt,splittopskip=\topskip,skipbelow=\baselineskip, skipabove=\baselineskip,ntheorem,roundcorner=5pt,font=\itshape]{quaestio}{Quaestio}

\mdtheorem[nobreak=true,outerlinewidth=1,%leftmargin=40,rightmargin=40,
backgroundcolor=yellow!50, outerlinecolor=black,innertopmargin=5pt,splittopskip=\topskip,skipbelow=\baselineskip, skipabove=\baselineskip,ntheorem,roundcorner=5pt,font=\itshape]{background}{Background}

%TRYING TO INCLUDE Ppls IN TOC
\usepackage{hyperref}


\begin{document}
\title{\color{Brown} Uso massivo de testes diagnósticos pode parar o surto de Coronavírus \\
\vspace{-0.35ex}}
\author{Chen Shen e Yaneer Bar-Yam \\ New England Complex Systems Institute \\
\vspace{+0.35ex}
\small{\textit{(traduzido por Lucas Pontes})}\\
 \today
  \vspace{-14ex} \\


\bigskip
\bigskip

\textbf{}
 }

\maketitle


\flushbottom % Makes all text pages the same height

%\maketitle % Print the title and abstract box

%\tableofcontents % Print the contents section

\thispagestyle{empty} % Removes page numbering from the first page

%----------------------------------------------------------------------------------------
%	ARTICLE CONTENTS
%----------------------------------------------------------------------------------------

%\section*{Introduction} % The \section*{} command stops section numbering

%\addcontentsline{toc}{section}{\hspace*{-\tocsep}Introduction} % Adds this section to the table of contents with negative horizontal space equal to the indent for the numbered sections

%\tableofcontents
%\section{ Introduction}
\renewcommand{\thefootnote}{\fnsymbol{footnote}}




\begin{multicols}{2}

Para que um surto seja parado, a transmissão deve ser interrompida. Uma estratégia chave é identificar indivíduos que têm a doença e isolá-los, para que outros não sejam infectados. Se o teste não for perfeito, podemos acabar isolando outras pessoas que não estão infectadas - esses são os chamados "falsos-positivos". Isso gera custos sociais adicionais, mas ainda interrompe o surto. Por outro lado, também podemos acabar deixando passar um número pequeno de "falsos negativos" (ou seja, pessoas infectadas que acabam sendo erroneamente identificadas em testes como não infectadas). Enquanto a proporção de falsos negativos for pequena o suficiente, haverá menos casos novos ao longo do tempo e o surto desaparecerá exponencialmente. Quanto maior o número de indivíduos conhecidamente infectados por cada indivíduo doente (taxa de infecção ou taxa de reprodução sem teste), menor a taxa de falsos negativos permitidos.

Quanto mais específicos pudermos ser sobre a identificação de casos em potencial, melhor, porque menos pessoas precisam ser isoladas. Por outro lado, quanto menos casos "perdermos" (ou seja, quanto mais casos de infectados forem identificados), mesmo que isolemos mais pessoas, mais rápido o surto desaparece e menos pessoas ficam doentes e morrem.

Como as várias maneiras pelas quais paramos os surtos se relacionam com essa estrutura geral? Aqui estão alguns exemplos:

\begin{itemize}
    \item \textbf{Auto-relato e diagnóstico:} nessa forma de teste, um indivíduo deve primeiro identificar que possui sintomas que requerem cuidados médicos e, em seguida, reportar-se a um médico que realiza um diagnóstico. Se o diagnóstico determinar que eles têm essa doença específica (com falsos positivos e negativos) o indivíduo é isolado. Quem está doente e não se auto-denuncia é falso-negativo. Aqueles que são diagnosticados incorretamente com a doença são falsos positivos. Normalmente, a dificuldade mais importante é a falsa-negativa por causa da autorrelato: pessoas que estão doentes, mas não a reconhecem e não se autorrelatam, talvez porque os sintomas sejam genéricos/inespecíficos ou não sejam imediatamente fatais. Como alternativa, os indivíduos podem suspeitar que estão com a doença, mas por razões pessoais, financeiras, sociais ou profissionais não escolhem o diagnóstico ou não têm a oportunidade de serem diagnosticados e isolados (há outras questões, incluindo se o processo de fazer o teste, ser testado e isolado resulta em novos casos, por exemplo, infectando aqueles envolvidos durante o transporte ou nos consultórios médicos, e qual o êxito do isolamento).
    
    \item \textbf{Rastreamento de contato:} nessa forma de teste, os indivíduos que entraram em contato com um indivíduo diagnosticado (de acordo com o método de autorrelato e diagnóstico) são identificados e contatados para procurar sintomas ou isolar-se diretamente. Mesmo se não estiverem infectados, seu isolamento (incluindo muitas pessoas que não estão realmente infectadas, ou seja, falsos positivos) é usado para interromper o surto.

    \item \textbf{Bloqueio - identificação da comunidade geográfica:} nessa forma de teste, todos os membros de uma comunidade geográfica na área de indivíduos infectados são considerados potencialmente infectados e isolados. Isso inclui muitos falsos positivos e pode interromper o surto.
    
    \item \textbf{Teste sintomático genérico em vizinhanças:} dessa maneira, todos os membros de uma comunidade geográfica na área de indivíduos infectados são testados para sintomas como febre (que pode estar associada à doença), mas também para outras condições, e são considerados potencialmente infectados e isolados. A vantagem dessa abordagem em relação aos bloqueios é que menos indivíduos estão isolados, reduzindo o custo social. A vantagem sobre um diagnóstico mais específico é que muitos mais indivíduos infectados são isolados. Essa abordagem foi usada efetivamente para interromper o surto de Ebola na Libéria e na Serra Leoa [1].
    
    \item \textbf{Uso massivo de testes específicos:} dessa maneira, o DNA ou outros testes específicos são aplicados amplamente à população, talvez focados em uma área geográfica específica, a fim de identificar possíveis casos a serem isolados. Se o teste for específico o suficiente e puder ser aplicado amplamente, essa abordagem poderá interromper o surto.
    
    \item \textbf{Amostragem aleatória direcionada}: dessa forma de teste, diagnóstico ou testes de DNA, são aplicados a indivíduos de populações altamente conectadas, por exemplo, em comunidades confinadas, como prisões, dormitórios, albergues, asilos, centros de reabilitação, enfermarias psiquiátricas, médicos instalações ou comunidades de aposentados. Nos locais em que um indivíduo é infectado, é provável que muitos sejam infectados, mesmo que ainda não apresentem sintomas ou apresentem resultados positivos. Nesse caso, toda a comunidade pode ser isolada (como indivíduos e não como um grupo) para impedir a transmissão adicional.
\end{itemize}

De maneira geral, vemos que qualquer maneira de determinar os indivíduos a serem isolados pode ser utilizada para interromper um contágio, usando vários testes, incluindo sintomas, localização geográfica ou testes moleculares, que têm a capacidade de identificar aqueles que estão infectados, mesmo se houver falsos positivos e com número suficientemente baixo de falsos negativos.

Além de realizar o teste, uma pergunta importante é quão cedo podemos identificar se alguém é um membro do grupo que deve ser isolado e como isso afeta o número de pessoas que este infecta. Se elas forem identificadas antes de se tornarem igualmente contagiosas, ou se houver apenas uma pequena janela de tempo desde que elas tenham se tornado contagiosas, isso pode ser suficiente para impedir novas infecções, e assim, a propagação do surto. Essa fração de tempo em que esses indivíduos não testados estão contagiosos atua de maneira semelhante aos falsos negativos: isso contribui para o número de indivíduos infectados e reduz a eficácia do teste para interromper o surto. Isso faz com que a aplicação rápida e antecipada do teste, independentemente da forma adotada (sintomática, geográfica e/ou molecular) seja parte essencial do que determinará se o teste será eficaz em interromper o surto e em minimizar quantas pessoas adoecerão e morrerão.

No caso do COVID-19, o surto de Coronavírus que veio de Wuhan, há um teste de DNA específico usando um cotonete no nariz ou na garganta, que pode fornecer identificação suficientemente rápida dos casos, para reduzir drasticamente a taxa de contágio se aqueles que tiverem um resultado positivo estiverem isolados. Os testes levam vários dias para fornecer resultados. Testes mais rápidos estão em desenvolvimento.

No início do surto, havia um limite severo para o número de testes que poderiam ser feitos, e a necessidade de isolamento com base na comunidade era imperativa e foi usada na China com um sucesso notável, principalmente em Wuhan, com um esforço massivo de rastreamento tradicional de contato físico (num local de 670.000 pessoas) [2]. Posteriormente na Coréia do Sul, enquanto um bloqueio foi implementado [3], foram feitos testes em escala muito maior, incluindo unidades de testes feitos com o paciente ainda dentro do carro [4]. Muito recentemente, há indícios de que o surto na Coréia do Sul esteja sob controle [5].

Atualmente, em muitos lugares do mundo, incluindo os EUA, há testes insuficientes para alcançar a testagem generalizada. Isso limita nossa capacidade de usar essa abordagem. Ainda assim, é possível, em princípio, que o teste seja produzido de forma rápida e barata e, em seguida, aplicado em massa para identificar casos, limitando a necessidade de usar outras abordagens, como bloqueios. Quando um grande número de testes estiver disponível, testes específicos maciços podem alcançar o resultado desejado para interromper o surto.

O que deveria ser feito? Retardar ou interromper o surto de coronavírus pode envolver uma abordagem multifacetada. Indivíduos, famílias e comunidades devem tomar precauções para evitar contatos e limitar a probabilidade de serem infectados. Ao mesmo tempo, as autoridades médicas devem alavancar a testagem e seu fornecimento imediato para os demais locais geográficos. Onde é possível, empresas ou ONGs também podem fornecer serviços de teste em locais convenientes, talvez até de porta em porta, como foi feito no final do surto na China [6], para identificação rápida de casos.

\end{multicols}

\section*{Referências}

[1] \textbf{How community response stopped ebola}. Disponível em: <https://necsi.edu/how-community-response-stopped-ebola>.

[2] \textbf{Report of the WHO-China Joint Mission on Coronavirus Disease 2019 (COVID-19)}. Disponível em: <https://www.who.int/docs/default-source/coronaviruse/who-china-joint-mission-on-covid-19-final-report.pdf>.

[3] \textbf{Daegu in Lockdown as Coronavirus Infections Soar.} Disponível em: <http://english.chosun.com/site/data/html_dir/2020/02/24/2020022401353.html>.

[4] \textbf{South Korea pioneers coronavirus drive-through testing station}. Disponível em: <https://www.cnn.com/2020/03/02/asia/coronavirus-drive-through-south-korea-hnk-intl/index.html>.

[5] \textbf{BREAKING: Coronavirus Update. Significant decline in daily new cases in South Korea–positive sign of gaining control}. Disponível em: <https://twitter.com/yaneerbaryam/status/1235734017699430401?s=20>.

[6] \textbf{China Goes Door to Door in Wuhan, Seeking Infections}. Disponível em: <https://www.courthousenews.com/china-goes-door-to-door-in-wuhan-seeking-infections/>.

% \bibliography{MyCollection.bib}

\end{document}

