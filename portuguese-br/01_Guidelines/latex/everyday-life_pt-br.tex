%\documentclass[twocolumn,journal]{IEEEtran}
\documentclass[onecolumn,journal]{IEEEtran}
\usepackage{amsfonts}
\usepackage{amsmath}
\usepackage{amsthm}
\usepackage{amssymb}
\usepackage{graphicx}
\usepackage[T1]{fontenc}
%\usepackage[english]{babel}
\usepackage{supertabular}
\usepackage{longtable}
\usepackage[usenames,dvipsnames]{color}
\usepackage{bbm}
%\usepackage{caption}
\usepackage{fancyhdr}
\usepackage{breqn}
\usepackage{fixltx2e}
\usepackage{capt-of}
%\usepackage{mdframed}
\setcounter{MaxMatrixCols}{10}
\usepackage{tikz}
\usetikzlibrary{matrix}
\usepackage{endnotes}
\usepackage{soul}
\usepackage{marginnote}
%\newtheorem{theorem}{Theorem}
\newtheorem{lemma}{Lemma}
%\newtheorem{remark}{Remark}
%\newtheorem{error}{\color{Red} Error}
\newtheorem{corollary}{Corollary}
\newtheorem{proposition}{Proposition}
\newtheorem{definition}{Definition}
\newcommand{\mathsym}[1]{}
\newcommand{\unicode}[1]{}
\newcommand{\dsum} {\displaystyle\sum}
\hyphenation{op-tical net-works semi-conduc-tor}
\usepackage{pdfpages}
\usepackage{enumitem}
\usepackage{multicol}
\usepackage[utf8]{inputenc}

\headsep = 5pt
\textheight = 730pt
%\headsep = 8pt %25pt
%\textheight = 720pt %674pt
%\usepackage{geometry}

\bibliographystyle{unsrt}

\usepackage{float}

 \usepackage{xcolor}

\usepackage[framemethod=TikZ]{mdframed}
%%%%%%%FRAME%%%%%%%%%%%
\usepackage[framemethod=TikZ]{mdframed}
\usepackage{framed}
    % \BeforeBeginEnvironment{mdframed}{\begin{minipage}{\linewidth}}
     %\AfterEndEnvironment{mdframed}{\end{minipage}\par}


%	%\mdfsetup{%
%	%skipabove=20pt,
%	nobreak=true,
%	   middlelinecolor=black,
%	   middlelinewidth=1pt,
%	   backgroundcolor=purple!10,
%	   roundcorner=1pt}

\mdfsetup{%
	outerlinewidth=1,skipabove=20pt,backgroundcolor=yellow!50, outerlinecolor=black,innertopmargin=0pt,splittopskip=\topskip,skipbelow=\baselineskip, skipabove=\baselineskip,ntheorem,roundcorner=5pt}

\mdtheorem[nobreak=true,outerlinewidth=1,%leftmargin=40,rightmargin=40,
backgroundcolor=yellow!50, outerlinecolor=black,innertopmargin=0pt,splittopskip=\topskip,skipbelow=\baselineskip, skipabove=\baselineskip,ntheorem,roundcorner=5pt,font=\itshape]{result}{Result}


\mdtheorem[nobreak=true,outerlinewidth=1,%leftmargin=40,rightmargin=40,
backgroundcolor=yellow!50, outerlinecolor=black,innertopmargin=0pt,splittopskip=\topskip,skipbelow=\baselineskip, skipabove=\baselineskip,ntheorem,roundcorner=5pt,font=\itshape]{theorem}{Theorem}

\mdtheorem[nobreak=true,outerlinewidth=1,%leftmargin=40,rightmargin=40,
backgroundcolor=gray!10, outerlinecolor=black,innertopmargin=0pt,splittopskip=\topskip,skipbelow=\baselineskip, skipabove=\baselineskip,ntheorem,roundcorner=5pt,font=\itshape]{remark}{Remark}

\mdtheorem[nobreak=true,outerlinewidth=1,%leftmargin=40,rightmargin=40,
backgroundcolor=pink!30, outerlinecolor=black,innertopmargin=0pt,splittopskip=\topskip,skipbelow=\baselineskip, skipabove=\baselineskip,ntheorem,roundcorner=5pt,font=\itshape]{quaestio}{Quaestio}

\mdtheorem[nobreak=true,outerlinewidth=1,%leftmargin=40,rightmargin=40,
backgroundcolor=yellow!50, outerlinecolor=black,innertopmargin=5pt,splittopskip=\topskip,skipbelow=\baselineskip, skipabove=\baselineskip,ntheorem,roundcorner=5pt,font=\itshape]{background}{Background}

%TRYING TO INCLUDE Ppls IN TOC
\usepackage{hyperref}


\begin{document}
\title{\color{Brown}  Dia-a-dia e COVID-19
\vspace{-0.35ex}}
\author{Chen Shen e Yaneer Bar-Yam \\ New England Complex Systems Institute \\
\vspace{+0.35ex}
\small{\textit{(traduzido por Lucas Pontes})}\\
 \today
  \vspace{-8ex} \\
%\bigskip
\textbf{}
 }

\maketitle

%\vspace{-1ex}
%\flushbottom % Makes all text pages the same height

%\maketitle % Print the title and abstract box

%\tableofcontents % Print the contents section

\thispagestyle{empty} % Removes page numbering from the first page

%----------------------------------------------------------------------------------------
%	ARTICLE CONTENTS
%----------------------------------------------------------------------------------------

%\section*{Introduction} % The \section*{} command stops section numbering

%\addcontentsline{toc}{section}{\hspace*{-\tocsep}Introduction} % Adds this section to the table of contents with negative horizontal space equal to the indent for the numbered sections

%\tableofcontents
%\section{ Introduction}

%\section*{Overview}


\begin{multicols}{2}

Nos últimos meses, nossas vidas mudaram de muitas maneiras profundas. Um jogo de basquete, uma ida ao supermercado e até mesmo caminhar pelo saguão do prédio até o elevador estão repletos de perigos em potencial - a possibilidade de que o atacante do outro time, o último cliente a segurar a mesma laranja no setor de frutas, ou seu vizinho que apertou o botão do elevador estejam infectados pelo COVID19, e transmitindo o vírus para nós. Isso vale para tudo o que eles tocam e cada respiração que fazem nos espaços que compartilhamos, mesmo que momentaneamente. Enquanto contatos próximos dominam a transmissão, outros perigos também devem ser evitados.

Corridas, ciclismo e encontros via conversas de vídeo com os amigos são maus substitutos para esse jogo de basquete, mas eles fazem muito bem. No entanto, todos temos que comer e, mesmo quando estamos nos abrigando, às vezes precisamos sair de casa. O que deve ser feito sob essas circunstâncias? Estamos em um momento de grande medo, perigo e incerteza. Aqui estão algumas diretrizes para tornar nossas atividades da vida diária mais seguras.

\section*{Edifícios residenciais}

Casas com entradas compartilhadas nos expõem a todos os outros em nosso prédio: nas áreas comuns, entradas, corredores, lavanderias, às vezes até no ar que respiramos através de um sistema de ventilação comum.

\vspace{2ex}

\begin{itemize}
\item Passe o mínimo de tempo possível nas entradas e áreas comuns, mesmo quando não estiverem lotadas. O coronavírus gruda nas superfícies e no ar.
\item Pressuponha que qualquer superfície - sua caixa de correio, maçaneta da porta, botões do elevador - esteja contaminada e possa transmitir COVID19. Coloque algo descartável entre você e a superfície: luvas, um pedaço de pano ou até mesmo um pedaço de papel. Seja criativo.
\item Quando possível, use uma máscara, lenço ou bandana ao caminhar em espaços compartilhados. Máscaras reutilizáveis e laváveis estão disponíveis e há instruções simples disponíveis para você fazer a sua.
\item Quando chegar ao seu apartamento, lave bem as mãos antes de tocar em qualquer outra coisa (com sabão por pelo menos 20 segundos). Leve sanitizadores de mão para limpá-las quando sair.
\item Consulte a possibilidade de determinarem uma área separada do edifício para recebimento de encomendas (mais informações abaixo).
\item Mantenha as janelas do seu apartamento abertas quando o tempo permitir. Também pode ser útil o uso de um purificador de ar HEPA no ar condicionado de seu apartamento, nos sistemas de ventilação central, ou quando as janelas não puderem ser abertas.
\item Pressuponha que os elevadores estejam contaminados, pois pode haver muitas pessoas por dia compartilhando esse pequeno espaço. Evite compartilhar elevadores com outras pessoas. Quando possível, use as escadas: é um ótimo exercício para começar, embora ainda não seja tão divertido quanto aquele jogo de basquete.
\end{itemize}

\section*{Compras em mercados, farmácias, mercearias e outras lojas}

Obter e fornecer itens essenciais em um momento de crise requer atenção extraordinária. Muitos supermercados estão agora oferecendo várias maneiras de tornar as compras seguras para seus funcionários e clientes. Encontre e dê preferência às lojas da sua região que levam a sério a segurança e que ofereçam alguns ou todos os itens a seguir:

\vspace{2ex}

\begin{itemize}
    \item Desinfetante para as mãos e toalhas descartáveis na entrada, em toda a loja, e no check-out para caixas e ensacadeiras, além de clientes.
    \item Funcionários que lembrem compradores de manterem uma distância segura uns dos outros.
    \item Funcionários que usam um termômetro infravermelho digital sem contato para verificar a temperatura corporal externa, para garantir que qualquer pessoa que entre na loja não tenha sintomas.
    \item Pedidos on-line para serem retirados na calçada, estacionamento ou entrega em domicílio sem contato.
    \item Horas estendidas para evitar aglomeração.
    \item Abertura antecipada com acesso limitado a idosos e outros clientes vulneráveis.
    \item Limite para o número de pessoas na loja a qualquer momento.
    \item Algumas farmácias agora enviarão ou entregarão suas prescrições.
    \item Clientes, funcionários da loja e gerentes de loja devem estar abertos a se comunicar para ajudar a melhorar a experiência de compra, com ênfase na segurança da comunidade.
\end{itemize}

\vspace{2ex}

\textbf{Como decidir se deve ir às compras ou não?}

\vspace{2ex}

\begin{itemize}
    \item Indivíduos que suspeitam que possam ter COVID-19 ou apresentam sintomas semelhantes aos da gripe, não devem ir às compras. Eles devem providenciar soluções alternativas que não envolvam contato com outras pessoas, se possível.
    \item Da mesma forma, aqueles com mais de 50 anos ou com condições de saúde preexistentes devem providenciar soluções alternativas de compras sempre que possível. Caso contrário, eles devem aproveitar as compras restritas de madrugada, onde são fornecidas.
    \item Soluções alternativas para compras incluem, entre outras: compras on-line, aplicativos ou sites de entrega, ter um amigo ou membro da família que realize as compras por você, e perguntar à loja se eles têm opções de entrega. A entrega, seja através de familiar, amigo ou entregador, deve ser feita sem contato.
    \item As compras devem ser feitas com a menor frequência possível.
    \item Limite o tempo de compras. Se possível, apenas um membro de cada família deve fazer as compras. Se o esforço for demais para uma pessoa, planeje a viagem com cuidado para que cada uma faça uma parte da lista e o tempo de compra seja o mais curto possível.
    \item Para reduzir a tensão nos estoques de lojas, e garantir que todos obtenham os recursos de que precisam, sugerimos que cada indivíduo tenha duas semanas em alimentos e mercadorias a qualquer momento, e desencorajamos fortemente a compra de quantidades excessivas de qualquer produto.
\end{itemize}

\vspace{2ex}

\textbf{Antes de ir à loja:}

\vspace{2ex}

\begin{itemize}
    \item Tenha um conjunto designado de roupas e sapatos que você usa ao fazer compras. Antes e depois das compras, lave ou desinfete completamente essas roupas e mantenha-as separadas das roupas e sapatos do dia a dia.
    \item Use luvas e uma máscara facial (ou cachecol ou bandana) que cubram seu nariz e boca.
    \item Traga suas próprias sacolas de compras designadas: caso as tenha, desinfecte-as antes e depois das compras e mantenha-as afastadas dos itens do dia-a-dia. Uma cesta com rodinhas é uma ótima alternativa para sacolas de compras quando é necessário caminhar longas distâncias.
    \item Desinfetante para as mãos, toalhetes, sabão ou outros materiais desinfetantes devem ser levados à loja, caso você ou outras pessoas necessitem.
    \item Faça uma lista de compras de itens essenciais antes de ir à loja. Se você conhece o layout da loja, organize a lista para poder percorrer a loja sem voltar atrás. Se possível, inclua na lista itens substitutos, caso a loja não possua algo específico. Uma lista bem feita tornará desnecessário ligar para casa, pois o uso de um telefone celular na loja deve ser evitado.
    \item Se possível, configure uma área de recebimento em sua garagem, alpendre ou área de entrada, onde você possa deixar itens não perecíveis por 2 a 3 dias, período em que qualquer vírus nas superfícies não estará ativo. Se você precisar trazê-los para sua casa, configure uma área marcada para eles perto da entrada.
\end{itemize}

\vspace{2ex}

\textbf{Alguns itens a serem considerados na sua lista de compras:}

\vspace{2ex}

\begin{itemize}
    \item Onde disponível, máscaras de respiração N95 ou máscara cirúrgica, se os respiradores não estiverem acessíveis.
    \item Desinfetantes, incluindo desinfetante para as mãos, sabonete, detergente para a roupa e sabonete.
    \item Luvas, casaco e revestimentos para sapatos
    \item Suprimentos médicos, incluindo termômetros, remédios para resfriado e medicamentos prescritos
    \item Alimentos não perecíveis
\end{itemize}

\vspace{2ex}

\textbf{Transporte:}

\vspace{2ex}

\begin{itemize}
    \item Evite o transporte público, se possível. Se necessário, tome precauções extras - máscaras (ou lenços ou bandanas) e luvas são essenciais. Os automóveis de passageiros são úteis não apenas para diminuir a exposição, mas também para trazer quantidades maiores de mantimentos para casa.
    \item Uber / 99 / Táxis e outros são melhores opções do que o transporte público, mas também têm riscos - máscaras e luvas são essenciais.
    \item Se você não possui um carro, mas tem a capacidade de conduzir um e tem condições financeiras, alugue um carro pelas próximas semanas.
\end{itemize}

\vspace{2ex}

\textbf{Quando você estiver na loja:}

\vspace{2ex}

\begin{itemize}
    \item Os protocolos de distanciamento social devem ser seguidos, inclusive mantendo uma distância de 2 metros da pessoa mais próxima o tempo todo.
    \item Higienize seu carrinho ou cesto antes e depois de usá-lo e coloque sua sacola de compras dentro dela.
    \item Use luvas e/ou sacos de plástico para levar itens da prateleira para o carrinho. Como as luvas podem ser contaminadas ao tocar em qualquer coisa da loja, é melhor usar luvas e sacos de plástico descartáveis sempre que possível.
    \item Evite tocar em qualquer coisa da loja, a menos que seja necessário, e tome cuidado extra para evitar tocar em seu rosto.
    \item Retire os itens da parte de trás da prateleira, onde é menos provável que tenham sido manuseados por vários clientes.
    \item O checkout deve ser realizado com o mínimo de contato. É preferível usar auto-checkout e pagamento eletrônico.
    \item Peça aos caixas e ensacadeiros que usem luvas ou desinfetantes para as mãos enquanto examinam seus itens.
\end{itemize}

\vspace{2ex}

\textbf{Ao chegar das compras em casa:}

\vspace{2ex}

\begin{itemize}
    \item Ao chegar em casa, coloque todos os itens em uma área de recebimento, onde possam ser deixados por 2-3 dias (veja acima).
    \item Lembrete: lave as mãos quando terminar.
    \item Para itens perecíveis ou urgentemente necessários, lave-os com sabão e enxágue com cuidado antes de guardar.
    \item Ao retornar de qualquer passeio com outras pessoas, mesmo que você tenha mantido um distanciamento seguro, coloque as roupas em uma bolsa para lavar roupa e tomar banho.
\end{itemize}

\section*{Recebendo correio e encomendas}

O coronavírus pode permanecer em papel e papelão por um dia. Para plásticos e outros materiais, ele pode permanecer por 3-4 dias, dependendo das condições. Os pacotes e envelopes foram tocados por muitas pessoas antes de chegarem a sua casa.

\vspace{2ex}

\begin{itemize}
    \item Se possível, deixe a embalagem fechada na sua área de recepção: garagem, varanda ou outra área semelhante por 2-4 dias.
    \item Se precisar abrir imediatamente, ou se não houver espaço disponível, lave a caixa com água e sabão, desinfetantes ou toalhetes com sanitizantes antes de abrir. Remova o item ou itens com cuidado e descarte a caixa externa.
    \item Lembrete: lave as mãos quando terminar.
\end{itemize}


\end{multicols}

\vspace{2ex}







% \bibliography{MyCollection.bib}
\bibliography{references.bib}

\end{document}
