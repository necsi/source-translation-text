%\documentclass[twocolumn,journal]{IEEEtran}
\documentclass[onecolumn,journal]{IEEEtran}
\usepackage{amsfonts}
\usepackage{amsmath}
\usepackage{amsthm}
\usepackage{amssymb}
\usepackage{graphicx}
\usepackage[T1]{fontenc}
%\usepackage[english]{babel}
\usepackage{supertabular}
\usepackage{longtable}
\usepackage[usenames,dvipsnames]{color}
\usepackage{bbm}
%\usepackage{caption}
\usepackage{fancyhdr}
\usepackage{breqn}
\usepackage{fixltx2e}
\usepackage{capt-of}
%\usepackage{mdframed}
\setcounter{MaxMatrixCols}{10}
\usepackage{tikz}
\usetikzlibrary{matrix}
\usepackage{endnotes}
\usepackage{soul}
\usepackage{marginnote}
%\newtheorem{theorem}{Theorem}
\newtheorem{lemma}{Lemma}
%\newtheorem{remark}{Remark}
%\newtheorem{error}{\color{Red} Error}
\newtheorem{corollary}{Corollary}
\newtheorem{proposition}{Proposition}
\newtheorem{definition}{Definition}
\newcommand{\mathsym}[1]{}
\newcommand{\unicode}[1]{}
\newcommand{\dsum} {\displaystyle\sum}
\hyphenation{op-tical net-works semi-conduc-tor}
\usepackage{pdfpages}
\usepackage{enumitem}
\usepackage{multicol}
\usepackage[utf8]{inputenc}


\headsep = 5pt
\textheight = 730pt
%\headsep = 8pt %25pt
%\textheight = 720pt %674pt
%\usepackage{geometry}

\bibliographystyle{unsrt}

\usepackage{float}

 \usepackage{xcolor}

\usepackage[framemethod=TikZ]{mdframed}
%%%%%%%FRAME%%%%%%%%%%%
\usepackage[framemethod=TikZ]{mdframed}
\usepackage{framed}
    % \BeforeBeginEnvironment{mdframed}{\begin{minipage}{\linewidth}}
     %\AfterEndEnvironment{mdframed}{\end{minipage}\par}


%	%\mdfsetup{%
%	%skipabove=20pt,
%	nobreak=true,
%	   middlelinecolor=black,
%	   middlelinewidth=1pt,
%	   backgroundcolor=purple!10,
%	   roundcorner=1pt}

\mdfsetup{%
	outerlinewidth=1,skipabove=20pt,backgroundcolor=yellow!50, outerlinecolor=black,innertopmargin=0pt,splittopskip=\topskip,skipbelow=\baselineskip, skipabove=\baselineskip,ntheorem,roundcorner=5pt}

\mdtheorem[nobreak=true,outerlinewidth=1,%leftmargin=40,rightmargin=40,
backgroundcolor=yellow!50, outerlinecolor=black,innertopmargin=0pt,splittopskip=\topskip,skipbelow=\baselineskip, skipabove=\baselineskip,ntheorem,roundcorner=5pt,font=\itshape]{result}{Result}


\mdtheorem[nobreak=true,outerlinewidth=1,%leftmargin=40,rightmargin=40,
backgroundcolor=yellow!50, outerlinecolor=black,innertopmargin=0pt,splittopskip=\topskip,skipbelow=\baselineskip, skipabove=\baselineskip,ntheorem,roundcorner=5pt,font=\itshape]{theorem}{Theorem}

\mdtheorem[nobreak=true,outerlinewidth=1,%leftmargin=40,rightmargin=40,
backgroundcolor=gray!10, outerlinecolor=black,innertopmargin=0pt,splittopskip=\topskip,skipbelow=\baselineskip, skipabove=\baselineskip,ntheorem,roundcorner=5pt,font=\itshape]{remark}{Remark}

\mdtheorem[nobreak=true,outerlinewidth=1,%leftmargin=40,rightmargin=40,
backgroundcolor=pink!30, outerlinecolor=black,innertopmargin=0pt,splittopskip=\topskip,skipbelow=\baselineskip, skipabove=\baselineskip,ntheorem,roundcorner=5pt,font=\itshape]{quaestio}{Quaestio}

\mdtheorem[nobreak=true,outerlinewidth=1,%leftmargin=40,rightmargin=40,
backgroundcolor=yellow!50, outerlinecolor=black,innertopmargin=5pt,splittopskip=\topskip,skipbelow=\baselineskip, skipabove=\baselineskip,ntheorem,roundcorner=5pt,font=\itshape]{background}{Background}

%TRYING TO INCLUDE Ppls IN TOC
\usepackage{hyperref}


\begin{document}
\title{\color{Brown} Diretrizes para famílias \\
\vspace{-0.35ex}}
\author{Chen Shen e Yaneer Bar-Yam \\ New England Complex Systems Institute \\
\vspace{+0.35ex}
\small{\textit{(traduzido por Lucas Pontes})}\\
 \today
  \vspace{-14ex} \\


\bigskip
\bigskip

\textbf{}
 }

\maketitle


\flushbottom % Makes all text pages the same height

%\maketitle % Print the title and abstract box

%\tableofcontents % Print the contents section

\thispagestyle{empty} % Removes page numbering from the first page

%----------------------------------------------------------------------------------------
%	ARTICLE CONTENTS
%----------------------------------------------------------------------------------------

%\section*{Introduction} % The \section*{} command stops section numbering

%\addcontentsline{toc}{section}{\hspace*{-\tocsep}Introduction} % Adds this section to the table of contents with negative horizontal space equal to the indent for the numbered sections

%\tableofcontents
%\section{ Introduction}
\renewcommand{\thefootnote}{\fnsymbol{footnote}}




\begin{multicols}{2}


Em áreas de risco elevado em que o governo não está tomando as medidas adequadas, proteger uma família ou grupo é um desafio. A propagação do fogo requer uma trilha de combustíveis: da mesma forma, o contágio do COVID-19 requer uma cadeia de indivíduos suscetíveis. A solução é: (1) Reduzir o contato entre a família e outras pessoas, e prover-se com produtos essenciais, à medida em que o risco aumentar; (2) criar um espaço seguro para a proteção daqueles que estão nele, através de acordo entre todos para não entrarem em contato físico desprotegido com outras pessoas, ou com superfícies tocadas por outras pessoas.

A existência de um "espaço seguro" também reduz o contágio, porque os que estão no espaço seguro não participam da transmissão da doença. Os membros de um espaço seguro podem combinar-se com outros para expandirem cuidadosamente o espaço seguro ou criar novos. 

Abaixo estão nossas diretrizes para as famílias.


\section*{Reduzindo o contato entre a família e outras pessoas}
\begin{itemize}
\item Leia atentamente as nossas diretrizes para indivíduos e compartilhe-as com os membros da família. Discuta com eles como reduzir o contato com outras pessoas.Transforme as reuniões de família para encontros virtuais. O surto atual será derrotado ou se espalhará. Se a primeira opção for o caso, daqui a alguns meses a rotina voltará ao normal. Se a segunda opção for o caso, serão necessárias ações diferentes.
\item Certifique-se de que você e seus familiares possuem os suprimentos necessários, incluindo medicamentos prescritos. Leve sempre em conta, quanto ao risco de contato com outras pessoas, os membros vulneráveis da família, incluindo idosos, mas também pessoas acima de 50 anos e pessoas com problemas de saúde crônicos. Reduza o contato deles com outras pessoas, provendo a eles suporte de modo que eles não precisem sair de casa, nem acessar espaços públicos.
\item Considere mudar temporariamente os indivíduos que estão em moradias coletivas (asilos, instalações de vida assistida etc.) para acomodações mais isoladas, incluindo residências particulares ou instalações para pequenos grupos.
\item Onde não for possível reduzir os contatos, converse com os responsáveis por instalações coletivas para que se aumente o nível de precauções contra a transmissão.
\item Evite aglomerações de pessoas e locais públicos, incluindo eventos e restaurantes, especialmente aqueles em espaços fechados.
\end{itemize}


\section*{Criando espaços seguros sob condições de alto risco}
\begin{itemize}

\item O objetivo principal de um "espaço seguro" é que um grupo de pessoas forme uma unidade solitária que reduza ao mínimo os contatos físicos com indivíduos externos, e que também seja capaz de sustentar-se e dar suporte.
\item Os indivíduos não precisam esperar por ações de segurança de cima para baixo, guiadas pelo governo. Na ausência de intervenção sistemática e agressiva, os "espaços seguros" auto-organizados também ajudam os indivíduos. Ao serem aumentadas progressivamente, as zonas seguras podem diminuir ou até parar os surtos locais.
\item Os "espaços seguros" podem começar pela família ou grupo de pessoas que compartilham uma única hospedagem. Várias habitações podem ser combinadas, incluindo idas e vindas entre elas (por exemplo, andando ou dirigindo), se e somente se protocolos seguros forem estabelecidos e respeitados. Para estabelecer com sucesso um Espaço Seguro, TODOS os participantes devem concordar com os princípios de minimizar o contato físico externo e aderir a ele. Também deve haver instruções claras sobre como agir e cooperar. Os membros do mesmo espaço seguro devem ser sinceros sobre o histórico de viagens e as condições de saúde, e serem responsáveis pela saúde um do outro.
\item Para que os indivíduos se comprometam com esse espaço compartilhado, pode ser necessário fazer acordos com o trabalho, escolas, família e amigos. Talvez seja necessário trabalhar de casa com a aprovação de um empregador, ou tirar uma licença.
\item Planejar-se para um período prolongado de tempo (por uma ou mais semanas) em um espaço seguro deve ser feito com antecedência, incluindo a obtenção de suprimentos, mas enquanto estiver no trabalho de obtê-los, tome cuidado extra, dada à potencial exposição a aglomerados de pessoas. Estratégias de sobrevivência podem ser de ajuda nesse contexto. Planejar-se acerca dos produtos que precisará é essencial, dado que cada viagem que fará para comprá-los envolverá riscos.
\item Sempre que possível, providencie entregas de itens, incluindo alimentos, para que as viagens ao supermercado sejam limitadas. \item Alguns cuidados devem ser tomados, pois qualquer item entregue deve ser manuseado por alguém. A menos que haja acordo com o fornecedor para o uso de luvas, é aconselhável lavar ou desinfetar os itens em áreas de transmissão ativa.
\item Para atividades essenciais, incluindo compras, durante as quais é inevitável o contato físico externo, os membros devem planejar-se com antecedência para agir com eficiência e minimizar a duração e extensão do contato. Sair e retornar ao espaço seguro envolvem precauções. Use proteção pessoal apropriada, incluindo luvas ou itens descartáveis (toalhas de papel) para agarrar ou manipular itens que não devem ser tocados, antissépticos ou álcool para uso nas mãos, e máscaras [1]. O retorno ao espaço requer lavagem ou desinfecção antes (de preferência) ou logo após da entrada.

\item Promova a comunicação interna e o cuidado mútuo para manter os membros do espaço em relacionamentos positivos e mentalmente saudáveis. Embora seja essencial reconhecer que a emergência atual exige ações e sacrifícios extraordinários, tal pensamento não substitui a importância do suporte mútuo.
\item Os membros do espaço seguro devem obter informações sobre as ações a serem tomadas no caso de um ou mais membros apresentarem sintomas de infecção. As ações variam de acordo com países/estados/locais e também são dinâmicas. Os membros devem informar a todos no grupo os últimos planos de contingência e informações de contato. Caso um membro mostre sintomas típicos, os demais devem agir rapidamente para ajudá-lo a fazer o teste e realizar o isolamento preventivo antes que os resultados sejam obtidos.
\end{itemize}

À medida em que o surto avançar, decisões difíceis surgirão inevitavelmente sobre sair de um espaço seguro para ajudar a família ou os amigos que não estão em um espaço seguro. Os indivíduos devem estar preparados para tomar esse tipo de decisão.

Em um momento de risco como esse, haverá ações tomadas por engano que podem comprometer a segurança do "espaço seguro". Para evitar reação exagerada a um evento individual, é importante perceber que qualquer ato individual tem uma baixa probabilidade de dano. No entanto, quando várias ações errôneas são tomadas, o risco aumenta dramaticamente. Garantir que as lições sejam aprendidas é mais importante do que acusação, culpa ou punição.

\section*{Observações adicionais}

[1] Apesar do uso de máscaras ser debatido, observamos que: (1) A Qualquer pessoa que tenha sintomas leves deve evitar o contato com outras pessoas e usar máscara enquanto estiver em contato público ou privado com outras pessoas. (2) O uso da máscara deve ser aceito em locais públicos para impedir que os doentes hesitem ou se sintam estigmatizados usando uma máscara. (3) Embora as máscaras não garantam segurança para um indivíduo saudável, e sua disponibilidade possa ser limitada devido à maior prioridade nas condições médicas, o uso de máscaras nos casos em que não se pode evitar o contato físico com outras pessoas reduz drasticamente o risco de infecção. (4) Para os indivíduos com alta suscetibilidade à doença (acima de 50 anos ou com condições de saúde preexistentes), assim como para aqueles que estão em áreas de risco elevado, o alto custo de ser infectado justifica o uso da máscara.



\end{multicols}



% \bibliography{MyCollection.bib}


\end{document}

