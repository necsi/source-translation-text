%\documentclass[twocolumn,journal]{IEEEtran}
\documentclass[onecolumn,journal]{IEEEtran}
\usepackage{amsfonts}
\usepackage{amsmath}
\usepackage{amsthm}
\usepackage{amssymb}
\usepackage{graphicx}
\usepackage[T1]{fontenc}
%\usepackage[english]{babel}
\usepackage{supertabular}
\usepackage{longtable}
\usepackage[usenames,dvipsnames]{color}
\usepackage{bbm}
%\usepackage{caption}
\usepackage{fancyhdr}
\usepackage{breqn}
\usepackage{fixltx2e}
\usepackage{capt-of}
%\usepackage{mdframed}
\setcounter{MaxMatrixCols}{10}
\usepackage{tikz}
\usetikzlibrary{matrix}
\usepackage{endnotes}
\usepackage{soul}
\usepackage{marginnote}
%\newtheorem{theorem}{Theorem}
\newtheorem{lemma}{Lemma}
%\newtheorem{remark}{Remark}
%\newtheorem{error}{\color{Red} Error}
\newtheorem{corollary}{Corollary}
\newtheorem{proposition}{Proposition}
\newtheorem{definition}{Definition}
\newcommand{\mathsym}[1]{}
\newcommand{\unicode}[1]{}
\newcommand{\dsum} {\displaystyle\sum}
\hyphenation{op-tical net-works semi-conduc-tor}
\usepackage{pdfpages}
\usepackage{enumitem}
\usepackage{multicol}
\usepackage[utf8]{inputenc}

\headsep = 5pt
\textheight = 730pt
%\headsep = 8pt %25pt
%\textheight = 720pt %674pt
%\usepackage{geometry}

\bibliographystyle{unsrt}

\usepackage{float}

 \usepackage{xcolor}

\usepackage[framemethod=TikZ]{mdframed}
%%%%%%%FRAME%%%%%%%%%%%
\usepackage[framemethod=TikZ]{mdframed}
\usepackage{framed}
    % \BeforeBeginEnvironment{mdframed}{\begin{minipage}{\linewidth}}
     %\AfterEndEnvironment{mdframed}{\end{minipage}\par}


%	%\mdfsetup{%
%	%skipabove=20pt,
%	nobreak=true,
%	   middlelinecolor=black,
%	   middlelinewidth=1pt,
%	   backgroundcolor=purple!10,
%	   roundcorner=1pt}

\mdfsetup{%
	outerlinewidth=1,skipabove=20pt,backgroundcolor=yellow!50, outerlinecolor=black,innertopmargin=0pt,splittopskip=\topskip,skipbelow=\baselineskip, skipabove=\baselineskip,ntheorem,roundcorner=5pt}

\mdtheorem[nobreak=true,outerlinewidth=1,%leftmargin=40,rightmargin=40,
backgroundcolor=yellow!50, outerlinecolor=black,innertopmargin=0pt,splittopskip=\topskip,skipbelow=\baselineskip, skipabove=\baselineskip,ntheorem,roundcorner=5pt,font=\itshape]{result}{Result}


\mdtheorem[nobreak=true,outerlinewidth=1,%leftmargin=40,rightmargin=40,
backgroundcolor=yellow!50, outerlinecolor=black,innertopmargin=0pt,splittopskip=\topskip,skipbelow=\baselineskip, skipabove=\baselineskip,ntheorem,roundcorner=5pt,font=\itshape]{theorem}{Theorem}

\mdtheorem[nobreak=true,outerlinewidth=1,%leftmargin=40,rightmargin=40,
backgroundcolor=gray!10, outerlinecolor=black,innertopmargin=0pt,splittopskip=\topskip,skipbelow=\baselineskip, skipabove=\baselineskip,ntheorem,roundcorner=5pt,font=\itshape]{remark}{Remark}

\mdtheorem[nobreak=true,outerlinewidth=1,%leftmargin=40,rightmargin=40,
backgroundcolor=pink!30, outerlinecolor=black,innertopmargin=0pt,splittopskip=\topskip,skipbelow=\baselineskip, skipabove=\baselineskip,ntheorem,roundcorner=5pt,font=\itshape]{quaestio}{Quaestio}

\mdtheorem[nobreak=true,outerlinewidth=1,%leftmargin=40,rightmargin=40,
backgroundcolor=yellow!50, outerlinecolor=black,innertopmargin=5pt,splittopskip=\topskip,skipbelow=\baselineskip, skipabove=\baselineskip,ntheorem,roundcorner=5pt,font=\itshape]{background}{Background}

%TRYING TO INCLUDE Ppls IN TOC
\usepackage{hyperref}


\begin{document}
\title{\color{Brown}  Diretrizes para Instituições Sujeitas à Risco Elevado em Resposta ao Coronavírus
\vspace{-0.35ex}}
\author{Aaron Green, Chen Shen e Yaneer Bar-Yam \\ New England Complex Systems Institute \\
\vspace{+0.35ex}
\small{\textit{(traduzido por Lucas Pontes})}\\
 \today
  \vspace{-8ex} \\
%\bigskip
\textbf{}
 }

\maketitle

%\vspace{-1ex}
%\flushbottom % Makes all text pages the same height

%\maketitle % Print the title and abstract box

%\tableofcontents % Print the contents section

\thispagestyle{empty} % Removes page numbering from the first page

%----------------------------------------------------------------------------------------
%	ARTICLE CONTENTS
%----------------------------------------------------------------------------------------

%\section*{Introduction} % The \section*{} command stops section numbering

%\addcontentsline{toc}{section}{\hspace*{-\tocsep}Introduction} % Adds this section to the table of contents with negative horizontal space equal to the indent for the numbered sections

%\tableofcontents
%\section{ Introduction}

%\section*{Overview}


\begin{multicols}{2}

% \section

As comunidades de aposentados, dormitórios, asilos, instalações de reabilitação e prisões são instituições de alto risco para transmissão de doenças. O COVID-19 é uma doença de transmissão rápida que requer hospitalização em cerca de 20\% dos casos e resulta em morte em 2-4\%. As complicações aumentam rapidamente para pessoas com 50 anos ou mais, com comorbidades como insuficiência cardíaca e doença arterial coronariana, aumentando ainda mais o risco. O COVID-19 pode ser transmitido por pacientes com sintomas leves (tosse, espirros ou temperatura elevada), e talvez até antes que os sintomas apareçam. Reduzir a probabilidade de transmissão é imprescindível em ambientes institucionais de alto risco, dado o que ocorreu na prisão de Rencheng, na China, e no hospital Cheongdo Daenam, na Coréia do Sul. Aqui estão as diretrizes para prevenção de transmissão em Instituições Sujeitas a Risco Elevado, baseada na introdução de barreiras à transmissão advinda do ambiente externo.

\section*{Regras Gerais}

\textit{Sobre visitantes}

\begin{itemize}
\item Desestimule visitas não essenciais.
\item Posicione alguém nas entradas para consultar o objetivo de
visita e pergunte se os visitantes apresentam algum sintoma, viajou recentemente para áreas de transmissão ativa ou foi exposto a pessoas com sintomas.
\item As visitas devem ser espaçadas em intervalos para evitar aglomeração.
\item As diretrizes de melhores práticas devem ser postadas facilmente em formato legível em espaços públicos para que funcionários, residentes e visitantes vejam.
\end{itemize}

\textit{Higiene}

\begin{itemize}
\item Garanta a limpeza e higienização de todos os objetos e locais frequentemente tocados por mãos. Isso inclui maçanetas, botões de elevador, pias, tampos de mesa e máquinas freqüentemente manipuladas, equipamentos, dispositivos eletrônicos e outros itens.
\item Verifique e garanta que os dispensadores de sabão e papel-toalha nos banheiros permaneçam sempre supridos durante todo o dia.
\item Forneça desinfetantes para as mãos em entradas, saídas e locais de alto tráfego.
\item Forneça lenços esterilizantes à base de álcool.
\end{itemize}

\textit{Funcionários e ambiente corporativo}
\begin{itemize}
\item Eduque funcionários e suas famílias, bem como residentes e suas famílias, acerca das formas de transmissão e prevenção contra o COVID-19.
\item Garanta que os funcionários saibam que, quando apresentarem sintomas, eles não devem comparecer ao local de trabalho ou à reuniões pessoais, e que serão pagos e não penalizados nos dias de doença. Elabore um sistema de registro e relatório para esses casos.
\item Garanta que os funcionários tenham apólices de plano de saúde adequadas, para que não tenham medo de procurar assistência quando apresentarem sintomas, mesmo os leves.
\item Prepare-se para a substituição de funcionários, caso eles adoeçam.
\item Fique a par das informações e recomendações atuais acerca da epidemia.
\item Substitua reuniões pessoais de trabalho por virtuais.
\end{itemize}

\section*{Regras mais rígidas para áreas de risco elevado de transmissão}

É essencial que as instituições de alto risco sigam as práticas de Espaços Seguros e permaneçam livres de doenças.

\vspace{1ex}

\textit{Para visitantes}

\begin{itemize}
\item Se possível, evite o contato externo e incentive o uso de texto, telefone e videoconferência para comunicação.
\item Caso seja necessária alguma visita, considere a criação de uma área separada para reuniões de visitantes, incluindo espaço suficiente para que todos permaneçam a uma distância segura (6 pés), links de vídeo para contato virtual, ou divisórias de vidro.
\item O uso de máscaras (se possível, N95) pode ser incentivado
mesmo se não houver sinais de surto.
\item Entregas de suprimentos e itens devem ser feitas por um único motorista, que não apresente sintomas da doença e que não tenha tido exposição à possíveis portadores da doença no intervalo de 14 a 21 dias anteriores.
\item Sempre que possível, as entregas de suprimentos, feitas sem contato físico direto com os entregadores, devem ser feitas de modo a não ser necessária a entrada dos entregadores na área da Instituição, e de preferência sendo deixadas em um espaço intermediário.
\item É altamente recomendável que as pessoas autorizadas a entrar no ambiente da Instituição Sujeita a Risco Elevado tenham sido recentemente testadas com resultados negativos.
\end{itemize}

\textit{Restaurantes e outras atividades de alto contato}
\begin{itemize}
\item Desinfete as áreas de contato após cada uso individual, incluindo superfícies de mesas, apoio de braço de cadeiras, cardápios, ou utilize toalhas e cardápios laváveis.
\item A equipe de garçons e dos demais serviços deve evitar contato e proximidade.
\item Segmente os horários das refeições entre grupos de pessoas, para evitar aglomerações e arranjos de mesas que obriguem as pessoas a ficarem fisicamente próximas uma das outras.
\item Onde o contato é essencial para o serviço prestado, devem ser elaborados protocolos de manuseio, incluindo ventilação, luvas, equipamentos descartáveis e máscaras.
\end{itemize}

\section*{Empregados, Instalações e Ambientes Corporativos}
\begin{itemize}
\item Enfatize aos funcionários que suas ações fora do trabalho podem levar à transmissão de infecções, arriscando a vida dos residentes. Mesmo que a doença apresente baixo risco, qualquer contato com indivíduos ou superfícies em áreas não seguras é extremamente perigoso para aqueles que estão nas Instalações Sujeitas a Risco Elevado. Eles devem assumir a responsabilidade e limitar ao mínimo o contato não seguro fora do trabalho.
\item Incentive a adoção de protocolos de Espaços Seguros por funcionários em suas próprias casas, limitando seu contato e o contato dos demais que moram com ele com outros indivíduos, e o contato com superfícies não são seguras, e mantenha registro de quais funcionários estão seguindo esses protocolos.
\item Mantenha contato com instalações médicas locais para coordenar testes rápidos de residentes e funcionários para o Coronavírus.
\item Divida as instalações para separar zonas, limitando funcionários e residentes de cruzar de uma zona para outra, de modo que, no caso de uma zona ser infectada, os funcionários e residentes de outra zona não sejam igualmente contaminados antes da devida detecção da contaminação na outra zona.
\end{itemize}




\end{multicols}

\vspace{2ex}

% \bibliography{MyCollection.bib}
\bibliography{references.bib}

\end{document}

