%\documentclass[twocolumn,journal]{IEEEtran}
\documentclass[onecolumn,journal]{IEEEtran}
\usepackage{amsfonts}
\usepackage{amsmath}
\usepackage{amsthm}
\usepackage{amssymb}
\usepackage{graphicx}
\usepackage[T1]{fontenc}
%\usepackage[english]{babel}
\usepackage{supertabular}
\usepackage{longtable}
\usepackage[usenames,dvipsnames]{color}
\usepackage{bbm}
%\usepackage{caption}
\usepackage{fancyhdr}
\usepackage{breqn}
\usepackage{fixltx2e}
\usepackage{capt-of}
%\usepackage{mdframed}
\setcounter{MaxMatrixCols}{10}
\usepackage{tikz}
\usetikzlibrary{matrix}
\usepackage{endnotes}
\usepackage{soul}
\usepackage{marginnote}
%\newtheorem{theorem}{Theorem}
\newtheorem{lemma}{Lemma}
%\newtheorem{remark}{Remark}
%\newtheorem{error}{\color{Red} Error}
\newtheorem{corollary}{Corollary}
\newtheorem{proposition}{Proposition}
\newtheorem{definition}{Definition}
\newcommand{\mathsym}[1]{}
\newcommand{\unicode}[1]{}
\newcommand{\dsum} {\displaystyle\sum}
\hyphenation{op-tical net-works semi-conduc-tor}
\usepackage{pdfpages}
\usepackage{enumitem}
\usepackage{multicol}
\usepackage[utf8]{inputenc}

\headsep = 5pt
\textheight = 730pt
%\headsep = 8pt %25pt
%\textheight = 720pt %674pt
%\usepackage{geometry}

\bibliographystyle{unsrt}

\usepackage{float}

% \usepackage{xcolor}

\usepackage[framemethod=TikZ]{mdframed}
%%%%%%%FRAME%%%%%%%%%%%
\usepackage[framemethod=TikZ]{mdframed}
\usepackage{framed}
    % \BeforeBeginEnvironment{mdframed}{\begin{minipage}{\linewidth}}
     %\AfterEndEnvironment{mdframed}{\end{minipage}\par}


%	%\mdfsetup{%
%	%skipabove=20pt,
%	nobreak=true,
%	   middlelinecolor=black,
%	   middlelinewidth=1pt,
%	   backgroundcolor=purple!10,
%	   roundcorner=1pt}

\mdfsetup{%
	outerlinewidth=1,skipabove=20pt,backgroundcolor=yellow!50, outerlinecolor=black,innertopmargin=0pt,splittopskip=\topskip,skipbelow=\baselineskip, skipabove=\baselineskip,ntheorem,roundcorner=5pt}

\mdtheorem[nobreak=true,outerlinewidth=1,%leftmargin=40,rightmargin=40,
backgroundcolor=yellow!50, outerlinecolor=black,innertopmargin=0pt,splittopskip=\topskip,skipbelow=\baselineskip, skipabove=\baselineskip,ntheorem,roundcorner=5pt,font=\itshape]{result}{Result}


\mdtheorem[nobreak=true,outerlinewidth=1,%leftmargin=40,rightmargin=40,
backgroundcolor=yellow!50, outerlinecolor=black,innertopmargin=0pt,splittopskip=\topskip,skipbelow=\baselineskip, skipabove=\baselineskip,ntheorem,roundcorner=5pt,font=\itshape]{theorem}{Theorem}

\mdtheorem[nobreak=true,outerlinewidth=1,%leftmargin=40,rightmargin=40,
backgroundcolor=gray!10, outerlinecolor=black,innertopmargin=0pt,splittopskip=\topskip,skipbelow=\baselineskip, skipabove=\baselineskip,ntheorem,roundcorner=5pt,font=\itshape]{remark}{Remark}

\mdtheorem[nobreak=true,outerlinewidth=1,%leftmargin=40,rightmargin=40,
backgroundcolor=pink!30, outerlinecolor=black,innertopmargin=0pt,splittopskip=\topskip,skipbelow=\baselineskip, skipabove=\baselineskip,ntheorem,roundcorner=5pt,font=\itshape]{quaestio}{Quaestio}

\mdtheorem[nobreak=true,outerlinewidth=1,%leftmargin=40,rightmargin=40,
backgroundcolor=yellow!50, outerlinecolor=black,innertopmargin=5pt,splittopskip=\topskip,skipbelow=\baselineskip, skipabove=\baselineskip,ntheorem,roundcorner=5pt,font=\itshape]{background}{Background}

%TRYING TO INCLUDE Ppls IN TOC
\usepackage{hyperref}


\begin{document}
\title{\color{Brown}  COVID-19: Como vencê-lo
\vspace{-0.35ex}}
\author{Chen Shen e Yaneer Bar-Yam \\ New England Complex Systems Institute \\
\vspace{+0.35ex}
\small{\textit{(traduzido por Lucas Pontes})}\\
 \today
  \vspace{-8ex} \\
%\bigskip
\textbf{}
 }

\maketitle

%\vspace{-1ex}
%\flushbottom % Makes all text pages the same height

%\maketitle % Print the title and abstract box

%\tableofcontents % Print the contents section

\thispagestyle{empty} % Removes page numbering from the first page

%----------------------------------------------------------------------------------------
%	ARTICLE CONTENTS
%----------------------------------------------------------------------------------------

%\section*{Introduction} % The \section*{} command stops section numbering

%\addcontentsline{toc}{section}{\hspace*{-\tocsep}Introduction} % Adds this section to the table of contents with negative horizontal space equal to the indent for the numbered sections

%\tableofcontents
%\section{ Introduction}

%\section*{Overview}


\begin{multicols}{2}


O COVID-19 é um oponente difícil: furtivo, enganoso, rápido e cruel. As quarentenas são úteis, mas não são suficientes. A taxa de casos nos EUA ainda está crescendo. A Itália, em quarentena nacional, tem obtido um declínio muito lento nos casos. A Áustria tem tido mais sucesso, mas suas ações começaram cedo. As coisas pioram antes de melhorar, mas podemos vencer ao nos anteciparmos a essa piora. Quando tomarmos as medidas certas, podemos pará-lo em 5 semanas. Após perdas difíceis e com importantes lições aprendidas, podemos voltar a uma vida normal. Aqui está como.

\begin{enumerate}
    \item Coloque todos a bordo: Todos os níveis/setores do governo, comunidades, empresas e indivíduos precisam se esforçar ao máximo para parar esta doença, e mesmo "pequenos vazamentos podem afundar o navio". Faça com que todos estejam focados, adaptáveis e criativos para maximizar seu papel. Esse vírus nos ataca exponencialmente, mas quando revidamos, fazemos o mesmo com o vírus.
    \item Bloqueio: Separe os indivíduos para impedir a transmissão. Não se trata de manter as pessoas em casa, mas de separar as pessoas. Estar ao ar livre é bom se as pessoas estiverem separadas. Cerca de 80\% da transmissão ocorre dentro de grupos familiares. Se houver risco de infecção, os colegas de casa podem reduzir o risco se separando temporariamente. A quarentena pode ser concluída em cinco semanas, porque o declínio exponencial pode ser tão rápido quanto o crescimento exponencial. Vai ser doloroso, mas não tema o pior: sofremos a pandemia quando subimos, mas podemos esmagá-la.
    \item Quarentenas: organize instalações de saúde específicas para casos leves e moderados, para que esses não infectem seus colegas de casa ou outras pessoas. Verifique casos porta a porta para encontrar todos os casos. Designe hospitais exclusivos para tratar os casos de COVID-19, e outros para casos de não-COVID, para minimizar os danos colaterais e a infecção cruzada.
    \item Use máscaras em espaços compartilhados: tosses, espirros e respiração comum: todas essas ações espalham os vírus. As máscaras bloqueiam a transmissão em caso de pessoas precisarem estar no mesmo espaço. Limpe as superfícies, e use luvas ou outras formas de evitar tocá-las.
    \item Restrições de viagem: Isso interrompe o início de novos surtos, viabiliza o rastreamento de contatos e preserva a conquista local para um "efeito catraca" - mantendo o progresso para a frente, impedindo a retirada para trás. Caso contrário, isso nunca acaba, como drenar uma banheira com uma torneira aberta. Nações e estados (veja o exemplo de muitas políticas de países e estados dos EUA: MA, RI, Flórida, Texas, Ohio) devem impor quarentenas de 14 dias aos viajantes, assim como separar regiões dentro de um estado. O transporte essencial de mercadorias e pessoas deve seguir protocolos de não-contato dentro de uma zona, ou de uma zona para outra.
    \item Serviços essenciais devem ser seguros: As empresas que fornecem itens essenciais devem desenvolver e fornecer entregas, retiradas de compras na calçada ou outros métodos para reduzir o risco para funcionários e clientes.
    \item Teste, teste, teste: tente lutar contra algo que você não pode ver. Se soubermos quem está infectado, podemos isolar somente esses indivíduos, e não qualquer pessoa com sintomas de resfriado. Os indivíduos podem transmitir mesmo antes e depois dos sintomas, enquanto os testes nos dirão se há infecção ou não. Ao testar pessoas em áreas específicas, podemos identificar as zonas nas quais focar e aquelas onde podemos relaxar as restrições. Nota: não podemos fazer isso se os viajantes puderem trazer a doença. para que isso funcione, são necessárias restrições de viagem.
    \item Diretrizes de saúde: concentre-se em manter as pessoas saudáveis para evitar que casos leves se tornem graves. O sistema imunológico se fortalece com hidratação, boa alimentação, bom sono, exercícios físicos e respirando ar fresco ou filtrado. Para febres normais de 101◦F / 38◦C ou um pouco mais, deixe seu corpo usar a febre para combater a doença. Acima de 103 ° F / 39,5 ° C, use medicamentos para reduzir a febre e procure atendimento médico (https://www.ncbi.nlm.nih.gov/pmc/articles/PMC4786079/).
\item Dê suporte ao atendimento médico: Hospitais e profissionais de saúde ficam sobrecarregados, e isso vai piorar até que nossas ações parem a transmissão. Dê a eles as ferramentas necessárias para fazer o melhor possível para salvar vidas: instalações, equipamentos médicos, equipamentos de proteção individual (EPIs), itens essenciais da vida cotidiana. Reconheça suas realizações e mostre apoio para ajudá-los a superar os desafios atuais.
\end{enumerate}

\end{multicols}


% \bibliography{MyCollection.bib}
\bibliography{references.bib}

\end{document}
