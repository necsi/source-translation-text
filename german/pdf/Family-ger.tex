%\documentclass[twocolumn,journal]{IEEEtran}
\documentclass[onecolumn,journal]{IEEEtran}
\usepackage{amsfonts}
\usepackage{amsmath}
\usepackage{amsthm}
\usepackage{amssymb}
\usepackage{graphicx}
\usepackage[T1]{fontenc}
%\usepackage[english]{babel}
\usepackage{supertabular}
\usepackage{longtable}
\usepackage[usenames,dvipsnames]{color}
\usepackage{bbm}
%\usepackage{caption}
\usepackage{fancyhdr}
\usepackage{breqn}
\usepackage{fixltx2e}
\usepackage{capt-of}
%\usepackage{mdframed}
\setcounter{MaxMatrixCols}{10}
\usepackage{tikz}
\usetikzlibrary{matrix}
\usepackage{endnotes}
\usepackage{soul}
\usepackage{marginnote}
%\newtheorem{theorem}{Theorem}
\newtheorem{lemma}{Lemma}
%\newtheorem{remark}{Remark}
%\newtheorem{error}{\color{Red} Error}
\newtheorem{corollary}{Corollary}
\newtheorem{proposition}{Proposition}
\newtheorem{definition}{Definition}
\newcommand{\mathsym}[1]{}
\newcommand{\unicode}[1]{}
\newcommand{\dsum} {\displaystyle\sum}
\hyphenation{op-tical net-works semi-conduc-tor}
\usepackage{pdfpages}
\usepackage{enumitem}
\usepackage{multicol}
\usepackage[utf8]{inputenc}


\headsep = 5pt
\textheight = 730pt
%\headsep = 8pt %25pt
%\textheight = 720pt %674pt
%\usepackage{geometry}

\bibliographystyle{unsrt}

\usepackage{float}

 \usepackage{xcolor}

\usepackage[framemethod=TikZ]{mdframed}
%%%%%%%FRAME%%%%%%%%%%%
\usepackage[framemethod=TikZ]{mdframed}
\usepackage{framed}
    % \BeforeBeginEnvironment{mdframed}{\begin{minipage}{\linewidth}}
     %\AfterEndEnvironment{mdframed}{\end{minipage}\par}


%	%\mdfsetup{%
%	%skipabove=20pt,
%	nobreak=true,
%	   middlelinecolor=black,
%	   middlelinewidth=1pt,
%	   backgroundcolor=purple!10,
%	   roundcorner=1pt}

\mdfsetup{%
	outerlinewidth=1,skipabove=20pt,backgroundcolor=yellow!50, outerlinecolor=black,innertopmargin=0pt,splittopskip=\topskip,skipbelow=\baselineskip, skipabove=\baselineskip,ntheorem,roundcorner=5pt}

\mdtheorem[nobreak=true,outerlinewidth=1,%leftmargin=40,rightmargin=40,
backgroundcolor=yellow!50, outerlinecolor=black,innertopmargin=0pt,splittopskip=\topskip,skipbelow=\baselineskip, skipabove=\baselineskip,ntheorem,roundcorner=5pt,font=\itshape]{result}{Result}


\mdtheorem[nobreak=true,outerlinewidth=1,%leftmargin=40,rightmargin=40,
backgroundcolor=yellow!50, outerlinecolor=black,innertopmargin=0pt,splittopskip=\topskip,skipbelow=\baselineskip, skipabove=\baselineskip,ntheorem,roundcorner=5pt,font=\itshape]{theorem}{Theorem}

\mdtheorem[nobreak=true,outerlinewidth=1,%leftmargin=40,rightmargin=40,
backgroundcolor=gray!10, outerlinecolor=black,innertopmargin=0pt,splittopskip=\topskip,skipbelow=\baselineskip, skipabove=\baselineskip,ntheorem,roundcorner=5pt,font=\itshape]{remark}{Remark}

\mdtheorem[nobreak=true,outerlinewidth=1,%leftmargin=40,rightmargin=40,
backgroundcolor=pink!30, outerlinecolor=black,innertopmargin=0pt,splittopskip=\topskip,skipbelow=\baselineskip, skipabove=\baselineskip,ntheorem,roundcorner=5pt,font=\itshape]{quaestio}{Quaestio}

\mdtheorem[nobreak=true,outerlinewidth=1,%leftmargin=40,rightmargin=40,
backgroundcolor=yellow!50, outerlinecolor=black,innertopmargin=5pt,splittopskip=\topskip,skipbelow=\baselineskip, skipabove=\baselineskip,ntheorem,roundcorner=5pt,font=\itshape]{background}{Background}

%TRYING TO INCLUDE Ppls IN TOC
\usepackage{hyperref}


\begin{document}
\title{\color{Brown} Richtlinien für Familien \\
\vspace{-0.35ex}}
\author{Chen Shen und Yaneer Bar-Yam \\ New England Complex Systems Institute \\
\vspace{+0.35ex}
\small{\textit{(übersetzt von A., Bergmann, N., Bellic, V. Brunsch})}\\
 \today
  \vspace{-14ex} \\


\bigskip
\bigskip

\textbf{}
 }

\maketitle


\flushbottom % Makes all text pages the same height

%\maketitle % Print the title and abstract box

%\tableofcontents % Print the contents section

\thispagestyle{empty} % Removes page numbering from the first page

%----------------------------------------------------------------------------------------
%	ARTICLE CONTENTS
%----------------------------------------------------------------------------------------

%\section*{Introduction} % The \section*{} command stops section numbering

%\addcontentsline{toc}{section}{\hspace*{-\tocsep}Introduction} % Adds this section to the table of contents with negative horizontal space equal to the indent for the numbered sections

%\tableofcontents
%\section{ Introduction}
\renewcommand{\thefootnote}{\fnsymbol{footnote}}




\begin{multicols}{2}


In Gebieten, in denen ein erhöhtes Ansteckungsrisiko besteht und die örtlichen Behörden nicht die entsprechenden Maßnahmen ergreifen, ist es schwierig die eigene Familie oder soziale Gruppe zu schützen. Die Ausbreitung von Feuer benötigt brennbare Materialien. Ebenso benötigt die Ausbreitung von COVID-19 Menschen, die sich infizieren können. Um die Ausbreitung von COVID-19 zu verhindern, müssen wir deswegen

\begin{itemize}

\item den Kontakt zwischen Familienmitgliedern und anderen Personen verringern und die Befriedigung von Grundbedürfnissen sicherstellen.

\item sichere Orte (Safe Spaces) für diejenigen bereitstellen, die sich auf freiwilliger Basis gemeinsam dazu entscheiden, den Kontakt mit anderen Menschen und möglicherweise infizierten Oberflächen zu vermeiden.

\end{itemize}

\bigskip

Die Safe Spaces verhindern die Ausbreitung von COVID-19, weil sie die Zahl der Ansteckungen minimieren. Mitglieder eines Safe Spaces können sich mit anderen zusammentun, um ihn schrittweise zu vergrößern oder neue zu schaffen. Im Folgenden stellen wir unsere Leitlinien für Familien dar. \\

Den Kontakt zwischen Familienmitgliedern und anderen Personen verringern:

\begin{itemize}

\item Lies dir unsere Leitlinien aufmerksam durch und informiere deine Familienmitglieder. Diskutiert, wie ihr den Kontakt zu anderen Personen reduzieren könnt.

\item Lasst Familienzusammenkünfte wenn möglich digital stattfinden.

\item Stellt sicher, dass ihr Zugang zu Nahrungsmitteln, Produkten des täglichen Bedarfs und benötigten Medikamenten habt. Sorgt euch vor allem um ältere Familienmitglieder, Familienmitglieder über 50 Jahre und Familienmitglieder mit chronischen Krankheiten hinsichtlich des Kontaktrisikos mit anderen. Sie sollten den Kontakt mit anderen Menschen vermeiden und nicht an öffentliche Orte gehen müssen.

\item Versucht Menschen in Gemeinschaftseinrichtungen (bspw. Pflegeheimen, Altersheimen, …) vorübergehend in abgeschiedeneren Unterkünften unterzubringen, zum Beispiel zu Hause oder in Einrichtungen mit kleineren Gruppengrößen.

\item Wenn es nicht möglich ist, den Kontakt mit anderen Menschen zu reduzieren, sprecht die Verantwortlichen in den Gemeinschaftseinrichtungen an, so dass diese die Schutzmaßnahmen gegen Ansteckung erhöhen.

\item Vermeidet Menschenansammlungen an öffentlichen Orten, größere Veranstaltungen und Restaurants, insbesondere innerhalb von Gebäuden.

\end{itemize}

\bigskip

Safe Spaces unter Risikobedingungen schaffen:

\begin{itemize}

\item Mit Hilfe von Safe Spaces können kleine Gruppen von Menschen eine unabhängige Einheit bilden und den physischen Kontakt zu Menschen außerhalb ihrer Gruppe vermeiden. Währenddessen können sie sich gegenseitig unterstützen.

\item Ihr müsst nicht auf die Sicherheitsmaßnahmen der Behörden warten. Wenn es keine deutliche top-down Reaktion durch die Behörden gibt, sind selbstorganisierte Safe Spaces eine notwendige Maßnahme. Durch langsame Vergrößerung der Safe Spaces können lokale Ausbrüche eingedämmt oder verhindert werden.

\item Safe Spaces können von Familien ausgehen oder auch einfach von Menschen, die zusammenwohnen. Wenn Sicherheitsregeln formuliert und eingehalten werden, dann können mehrere Wohnungen einen gemeinsam Safe Space bilden. Um einen Safe Space zu bilden, muss JEDE Person zustimmen, den Kontakt zu Außenstehenden zu minimieren. Es muss darüber hinaus klare Regeln für die Zusammenarbeit geben. Mitglieder eines Safe Spaces sollten offen über ihre Aufenthaltsorte sprechen und sich um die Gesundheit der anderen Mitglieder sorgen.

\item Damit Menschen einen gemeinsamen Safe Space einrichten können, müssen möglicherweise spezielle Vereinbarungen mit Arbeitgebern, Schulen, Familien und Freunden getroffen werden.

\item Es sollte im Vorfeld für eine oder mehrere Wochen geplant werden, was die Mitglieder des Safe Spaces benötigen. Bei den Einkäufen sollten große Menschengruppen vermieden werden. Genaue Planung ist nötig, da jede Besorgung ein potentielles Infektionsrisiko birgt.

\item Wenn möglich sollten Lieferservices genutzt werden, so dass persönliche Einkäufe vermieden werden. Aber auch bestellte Waren müssen von einem Menschen gebracht werden. Wenn der Zulieferer keine Handschuhe benutzt, ist es ratsam, Waren zu reinigen und zu desinfizieren, falls man in einer Gegend mit vielen erkrankten Personen wohnt.

\item Alle Aktivitäten außerhalb des Safe Spaces sollten sorgfältig vorbereitet werden und so kurz wie möglich sein. Es müssen für das Verlassen des Safe Spaces und das Zurückkehren Vorsichtsmaßnahmen getroffen werden. Benutzt persönliche Schutzausrüstung, bspw. Handschuhe, um Gegenstände anzufassen und Desinfektionsmittel sowie Schutzmasken. Beim Zurückkehren sollten auf jeden Fall die Hände gereinigt und desinfiziert werden.

\item Sorgt füreinander und achtet auf eure Gesundheit und euer Wohlbefinden. Ausnahmesituationen wie diese erfordern außergewöhnliche Maßnahmen und auch persönliche Opfer. Umso wichtiger ist es, dass ihr euch gegenseitig unterstützt.

\item Die Mitglieder eines Safe Spaces sollten genau absprechen, wie vorgegangen werden soll, sobald ein Mitglied Infektionssymptome aufzeigt. Die Maßnahmen unterscheiden sich von Region zu Region und werden sich auch im Zeitverlauf noch verändern. Jeder sollte auf dem neuesten Stand bezüglich Kontingenzplänen und Kontaktinformationen sein. Sollte ein Mitglied Symptome zeigen, so müssen die anderen Mitglieder dafür sorgen, dass ein Test durchgeführt wird. Vorsorglich sollte sich das Mitglied mit Symptomen selbst isolieren.

\end{itemize}

Wenn der Ausbruch voranschreitet kann es sein, dass schwere Entscheidungen getroffen werden müssen, beispielsweise wenn der Safe Space verlassen werden muss, um anderen Familienmitgliedern zu helfen, die nicht im Safe Space sind. Alle Mitglieder sollten darauf vorbereitet werden, solche Entscheidungen zu treffen.\\

In risikoreichen Zeiten wird manchmal versehentlich auf eine Art und Weise gehandelt, die die Sicherheit gefährdet. Um Überreaktionen auf ein Ereignis zu vermeiden, bedenkt, dass jede einzelne Handlung nur ein geringes Gefahrenpotential bietet. Dennoch, mehrere Handlungen hintereinander können das Risiko drastisch erhöhen. Es ist wichtiger gemeinsam in solchen Situationen zu lernen, anstatt sich Vorwürfe zu machen oder zu bestrafen.


\end{multicols}



% \bibliography{MyCollection.bib}


\end{document}
