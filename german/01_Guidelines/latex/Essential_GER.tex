\documentclass{article}

\usepackage[utf8]{inputenc}

\title{Grundsätzliche Leitlinie zum Coronavirus}
\author{Chen Shen und Yaneer Bar-Yam \\ New England Complex Systems Institute \\ (übersetzt von K. Weisser, V. Brunsch)}

\date{März 2020}



\begin{document}

\maketitle


\begin{center}I. AN ALLE:\end{center}
\begin{itemize}
\item Wir müssen anerkennen, dass dies keine normale Situation ist. Normales Verhalten muss angepasst werden, um zu überleben und das Leben anderer zu schützen und aufrecht zu erhalten.
\item In 80 Prozent der Fälle ist der Krankheitsverlauf mild, 20 Prozent der Fälle müssen im Krankenhaus behandelt werden, 10 Prozent der Fälle werden intensivpflichtig und in zwei bis vier Prozent der Fälle sterben Infizierte. Auch milde Fälle sind ansteckend und können zu schweren Verläufen bei anderen Menschen führen. Jeder hat die Verantwortung, Familie, Freunde und die Gesellschaft vor mehr Schaden zu bewahren.
\item Minimieren Sie physischen Kontakt. Erhalten sie notwendige Tätigkeiten aufrecht und andere nur dort wo möglich. Helfen Sie anderen so viel Sie können.
\item Es kann sein, dass Verhaltensweisen nicht sofort ideal der Situation angepasst sind, aber jeder Schritt, den Sie unternehmen, um sich weniger einer potentiellen Infektion auszusetzen, reduziert sowohl Ihr eigenes Risiko als auch den Ausbruch. Jegliche Maßnahmen können mit der Zeit und durch Zusammenarbeit verbessert werden.
\end{itemize}




\begin{center}II. SELBSTISOLATION:\end{center}
\begin{itemize}
\item Halten Sie zwei Meter Abstand zu anderen Menschen und versuchen Sie, unsichere öffentliche oder gemeinschaftlich genutzte Oberflächen nicht zu berühren.
\item Helfen Sie anderen Menschen in Ihrer Umgebung ohne körperlichen Kontakt. Bleiben Sie online im Austausch.
\item Benutzen Sie Seife, Desinfektionsmittel, Handschuhe und Plastiktüten.
\item Bleiben Sie informiert über typische Symptome und örtliche Anweisungen rund um Testen und Behandlung. Seien Sie in der Lage, unverzüglich zu handeln, wenn sich irgendwelche Symptome einstellen.
\item Vereinbaren Sie einen regelmäßigen Austausch mit einem verlässlichen Verwandten oder Freund (in manchen Fällen treten Symptome sehr rasch auf).
\end{itemize}




\begin{center}III. FAMILIE UND FREUNDE:\end{center}
\begin{itemize}
\item Teilen Sie einen Safe Space mit Familie / Freunden, die damit einverstanden sind, sich unter gemeinsamen Regeln von anderen zu isolieren.
\item Koordinieren Sie Hilfe mit anderen innerhalb dieses Safe Spaces.
\item Individuen mit Symptomen oder einem positiven Test sollten sich getrennt von Familie und Freunden isolieren.
\end{itemize}




\begin{center}IV. GEMEINDE:\end{center}
\begin{itemize}
\item Identifizieren sie Zonen und bewachen sie, insofern das möglich ist, deren Grenzen, indem Sie nur notwendigen Verkehr erlauben.
\item Organisieren sie die Lieferung von lebensnotwendigen Gütern ohne direkten Kontakt.
\item Untersuchen Sie alle Mitglieder der Gemeinde auf Symptome. Testen und isolieren Sie wenn nötig. Machen Sie Individuen kenntlich, die wahrscheinlich nicht infiziert sind.
\item Stellen Sie Pflege ohne direkten Kontakt für isolierte Personen zur Verfügung.
\item Erweitern Sie nach und nach Safe Space-Einrichtungen und -Dienstleistungen.
\end{itemize}




\begin{center}V. FIRMEN/ORGANISATIONEN:\end{center}
\begin{itemize}
\item Maximieren Sie Heimarbeit, um Selbstisolation zu ermöglichen und Safe Spaces zu fördern.
\item Erhalten Sie alle wesentlichen Funktionen aufrecht und reduzieren Sie den Einfluss auf alle Funktionen, indem Sie den Arbeitsplatz zum Safe Space machen.
\end{itemize}



\begin{center}VI. GESUNDHEITSWESEN:\end{center}
\begin{itemize}
\item Trennen Sie Einrichtungen, um Kontaminierungen zu vermeiden. Schaffen Sie temporär große, voneinander getrennte Räume für verschiedene Versorgungslevel.
\end{itemize}



\begin{center}VII. GEMEINDE/REGIERUNG/FIRMEN:\end{center}
\begin{itemize}
\item Organisieren Sie schnelles Testen, um "sichere" und zu isolierende Individuen zu identifizieren.
\item Isolieren Sie Gruppen mit aktiver Übertragung und helfen Sie mit der Grundversorgung.
\end{itemize}

\end{document}
