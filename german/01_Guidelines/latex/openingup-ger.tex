%\documentclass[twocolumn,journal]{IEEEtran}
\documentclass[onecolumn,journal]{IEEEtran}
\usepackage{amsfonts}
\usepackage{amsmath}
\usepackage{amsthm}
\usepackage{amssymb}
\usepackage{graphicx}
\usepackage[T1]{fontenc}
%\usepackage[english]{babel}
\usepackage[ngerman]{babel}
\usepackage{supertabular}
\usepackage{longtable}
\usepackage[usenames,dvipsnames]{color}
\usepackage{bbm}
%\usepackage{caption}
\usepackage{fancyhdr}
\usepackage{breqn}
\usepackage{fixltx2e}
\usepackage{capt-of}
%\usepackage{mdframed}
\setcounter{MaxMatrixCols}{10}
\usepackage{tikz}
\usetikzlibrary{matrix}
\usepackage{endnotes}
\usepackage{soul}
\usepackage{marginnote}
%\newtheorem{theorem}{Theorem}
\newtheorem{lemma}{Lemma}
%\newtheorem{remark}{Remark}
%\newtheorem{error}{\color{Red} Error}
\newtheorem{corollary}{Corollary}
\newtheorem{proposition}{Proposition}
\newtheorem{definition}{Definition}
\newcommand{\mathsym}[1]{}
\newcommand{\unicode}[1]{}
\newcommand{\dsum} {\displaystyle\sum}
\hyphenation{op-tical net-works semi-conduc-tor}
\usepackage{pdfpages}
\usepackage{enumitem}
\usepackage{multicol}
\usepackage[sort&compress]{natbib}

%\headsep = 5pt
%\textheight = 730pt
\headsep = 25pt
\textheight = 674pt
%\usepackage{geometry}

\bibliographystyle{unsrt}

\usepackage{float}

\usepackage{xcolor}
 
\usepackage[framemethod=TikZ]{mdframed}
%%%%%%%FRAME%%%%%%%%%%%
\usepackage[framemethod=TikZ]{mdframed}
\usepackage{framed}
    % \BeforeBeginEnvironment{mdframed}{\begin{minipage}{\linewidth}}
     %\AfterEndEnvironment{mdframed}{\end{minipage}\par}
% \usepackage[document]{ragged2e}

%	%\mdfsetup{%
%	%skipabove=20pt,
%	nobreak=true,
%	   middlelinecolor=black,
%	   middlelinewidth=1pt,
%	   backgroundcolor=purple!10,
%	   roundcorner=1pt}

\mdfsetup{%
	outerlinewidth=1,skipabove=20pt,backgroundcolor=yellow!50, outerlinecolor=black,innertopmargin=0pt,splittopskip=\topskip,skipbelow=\baselineskip, skipabove=\baselineskip,ntheorem,roundcorner=5pt}

\mdtheorem[nobreak=true,outerlinewidth=1,%leftmargin=40,rightmargin=40,
backgroundcolor=yellow!50, outerlinecolor=black,innertopmargin=0pt,splittopskip=\topskip,skipbelow=\baselineskip, skipabove=\baselineskip,ntheorem,roundcorner=5pt,font=\itshape]{result}{Result}


\mdtheorem[nobreak=true,outerlinewidth=1,%leftmargin=40,rightmargin=40,
backgroundcolor=yellow!50, outerlinecolor=black,innertopmargin=0pt,splittopskip=\topskip,skipbelow=\baselineskip, skipabove=\baselineskip,ntheorem,roundcorner=5pt,font=\itshape]{theorem}{Theorem}

\mdtheorem[nobreak=true,outerlinewidth=1,%leftmargin=40,rightmargin=40,
backgroundcolor=gray!10, outerlinecolor=black,innertopmargin=0pt,splittopskip=\topskip,skipbelow=\baselineskip, skipabove=\baselineskip,ntheorem,roundcorner=5pt,font=\itshape]{remark}{Remark}

\mdtheorem[nobreak=true,outerlinewidth=1,%leftmargin=40,rightmargin=40,
backgroundcolor=pink!30, outerlinecolor=black,innertopmargin=0pt,splittopskip=\topskip,skipbelow=\baselineskip, skipabove=\baselineskip,ntheorem,roundcorner=5pt,font=\itshape]{quaestio}{Quaestio}

\mdtheorem[nobreak=true,outerlinewidth=1,%leftmargin=40,rightmargin=40,
backgroundcolor=yellow!50, outerlinecolor=black,innertopmargin=5pt,splittopskip=\topskip,skipbelow=\baselineskip, skipabove=\baselineskip,ntheorem,roundcorner=5pt,font=\itshape]{background}{Background}

%TRYING TO INCLUDE Ppls IN TOC
\usepackage{hyperref}


\begin{document}
\title{\color{Brown} Wieder aufmachen \\
\vspace{-0.35ex}}
\author{\large Zvi Bar-Yam, Chen Shen, Yaneer Bar-Yam \\ New England Complex Systems Institute \\
\vspace{+0.35ex}
\small{\textit{(übersetzt von U. Renger, V. Brunsch})}\\
 \today 
  \vspace{-10ex} \\ 

   
\bigskip
\bigskip

\textbf{}
 }
    
\maketitle


\flushbottom % Makes all text pages the same height

%\maketitle % Print the title and abstract box

%\tableofcontents % Print the contents section

\thispagestyle{empty} % Removes page numbering from the first page

%----------------------------------------------------------------------------------------
%	ARTICLE CONTENTS
%----------------------------------------------------------------------------------------

%\section*{Introduction} % The \section*{} command stops section numbering

%\addcontentsline{toc}{section}{\hspace*{-\tocsep}Introduction} % Adds this section to the table of contents with negative horizontal space equal to the indent for the numbered sections

%\tableofcontents 
%\section{ Introduction}
\renewcommand{\thefootnote}{\fnsymbol{footnote}}

\large
\begin{multicols}{2}

Bevor man die Wirtschaft wieder in die Gänge bringt, muss man sich vergewissern, dass es nicht zu einem erneuten wirtschaftlichen Zusammenbruch kommt. Eine vorzeitige Lockerung der Einschränkungen garantiert den Verlust all dessen, was erreicht worden ist. Selbst eine kurze vorzeitige Lockerung würde neue Übertragungen bewirken, die man nicht in ein paar Wochen rückgängig machen kann.

Bedingungen und zu befolgendes Vorgehen:
\begin{enumerate}
\item Einschränkungen regional nach geografisch isolierten Gebieten lockern (nicht nach Industriekonzernen, Handel oder Arbeitsplätzen).
\item Dafür sorgen, dass durch Reisebeschränkungen keine neuen Krankheitsfälle importiert werden. Bußgelder oder das Zurückschicken ins Herkunftsland können helfen, den Reiz zur Einreise zu vermindern.
\item Stoppen von lokalen Übertragungen (Werden Reisende oder andere Menschen, die Kontakt mit Erkrankten hatten, während ihrer Zeit in Quarantäne ebenfalls krank, verhindert das nicht die Öffnung).
\item Dafür sorgen, dass durch genügend Tests Gegenden identifiziert werden können, die frei von Viren sind. Selbst wenn die Zahl der Fälle deutlich sinkt, sollte mindestens 2 Wochen lang eine breitflächige Testung fortgesetzt werden, um Gruppenansteckungen zu verhindern, die von Personen verursacht werden, die eine lange Inkubationszeit haben oder falsch-negativ getestet wurden.
\item Innerhalb der letzten 14 Tage (Inkubationszeit) sollte es keine lokalen Neuinfizierungen geben.
\item Einrichtungen für die Isolierung und medizinische Versorgung von positiv Getesteten bereitstellen.
\item Einrichten einer Kontaktverfolgung.
\item Zahlreiche Schritte unternehmen, um stufenweise Einschränkungen zu lockern und neue Fälle zu kontrollieren.
\item Sicherstellen, dass für mehrere Wochen nach der Öffnung Masken getragen werden.
\item Öffentliche Verkehrsmittel und große Treffen können erst im letzten Schritt der Öffnung wieder erlaubt werden, um Superspreader-Ereignisse zu vermeiden. Dann können auch Einschränkungen an Hochrisiko-Einrichtungen und für gefährdete Menschen gelockert werden.
\end{enumerate}

Solange es Einschränkungen gibt, sind immer noch einige Dinge möglich:
\begin{enumerate}
\item In einem schwach besiedelten Gebiet kann man ins Freie gehen.
\item Man kann 1 oder 2 Leute draußen treffen, muss aber 6-9m voneinander entfernt bleiben (2m ist nicht genug). Ein kürzerer Abstand ist nur dann möglich, wenn Wind weht.
\item In nicht dicht besiedelten Gebieten kann man Auto fahren und im Auto bleiben.
\end{enumerate}

Öffnung von Schulen
\begin{enumerate}
\item Mit Treffen im Freien anfangen, ohne Kontakt zwischen Lehrer und Schüler.
\item In Gegenden mit ausgezeichneter Durchlüftung kleine Gruppentreffen im Freien ohne Kontakt regeln.
\item Spielverabredungen zwischen 2 Schülern organisieren, vorzugsweise im Freien. Wenn drinnen, dann nur auf Kontakte zwischen 2 Familien beschränken, die 14 Tage sicher in Quarantäne isoliert waren. 
\end{enumerate}

\end{multicols}
%\begin{thebibliography}{20}

%\bibitem{r1} a


%\end{thebibliography}

% \bibliography{MyCollection.bib}



\end{document}